\subsubsection{双作用叶片泵的流量}

双作用叶片泵的流量与两叶片间密闭工作腔的容积在半转中的变化量 $\Delta V$ 有关,而 $\Delta V$ 等于两叶片处于定子长度半径 $R$ 圆弧(见图2-14)上的工作腔容积 $V_{1}$ 减去短半径 $r$ 圆弧上的工作腔容积 $V_{2}$ 。由于叶片有厚度 $s$ ,叶片所占据的容积并不起输油作用,因此在计算 $V_{1}$ 和 $V_{2}$ 时,应扣除叶片所占的容积。

由图可得

\begin{align}
V_{1}&=\frac{\pi (R^{2}-r_{0}^{2})}{z} b- \frac{R-r_{0}}{\cos \theta} sb \notag \\
V_{2}&=\frac{\pi (r^{2}-r_{0}^{2})}{z} b- \frac{r-r_{0}}{\cos \theta} sb \notag \\
\Delta V&=V_{1}-V_{2}=\frac{\pi (R^{2}-r^{2})}{z} b- \frac{R-r}{\cos \theta} sb \notag
\end{align}

\noindent 式中, 
\begin{tabular}[t]{ll}
 $R$ & ————定子长半径; \\
 $r$ & ————定子短半径; \\
 $r_{0}$ & ————转子半径; \\
 $b$ & ————叶片宽度; \\
 $s$ & ————叶片厚度; \\
 $\theta$ & ————叶片倾角;\\
 $z$ & ————叶片数。
\end{tabular}

双作用叶片泵的排量$q$和流量$Q$分别为
\begin{align}
q&=2z\Delta V=2b[\pi (R^{2}-r^{2})-\frac{R-r}{\cos \theta}sz] \notag \\
Q&=qn\eta _{V_{p}}=2bn[\pi(R^{2}-r^{2})-\frac{R-r}{\cos \theta}sz]\eta _{V_{p}}
\end{align}

一般双作用叶片泵的叶片底部都与压油腔相连通,叶片在压油区时,叶片向里运动,叶片从压油腔让出的那部分容积刚好由叶片底部挤出的那部分压力油来补偿。因此在压油区,叶片的存在并不影响流量,也就不会影响流量的均匀性。但叶片在吸油区时,因为这时叶片的底部仍通压力油,当叶片向外伸出时,叶片底部容积增大,需由压力油来补充,使输出流量减少。如果处于吸油区的叶片,其底部容积变化率的总和是一个常数,那么也不会影响流量的均匀性。为此,对等加速度过渡曲线来讲,应使在过渡线段上始终保持有两个叶片,它们之间的夹角为$\alpha/2$(即当叶片数$z=12,\alpha/2=30^{\circ}$时),这两个叶片运动速度$dp/dt$的总和就是一个常数 (参见图2-12(b)),从而保证流量平稳。国产双作用叶片泵的叶片数为12,故其流量脉动较其他液压泵(除螺杆泵外)小得多。

\subsubsection{高压叶片泵的结构特点}

一般双作用叶片泵的叶片底部通压力油,这就使得处于吸油区的叶片顶部和底部的液压作用力不平衡,因为这时叶片的顶部是低压油,而底部是压力油。叶片顶部以很大的压紧力抵在定子吸油区的内表面上,使磨损加剧,影响了油泵的使用寿命,尤其是工作压力较高时,磨损更严重。因此吸油区叶片两端压力不平衡,限制了双作用叶片泵工作压力的提高。所以,要提高叶片泵的压力,则必须减小吸油区叶片对定子表面的压紧力。目前一般常采用减小叶片底部受压面积及通到吸油区叶片底部油液压力的办法来减小吸油区叶片对定子的压力。

\subsection{单作用叶片泵}

单作用叶片泵的构造和工作原理可用图2-15来说明。它与双作用叶片泵相似,也是由转子1、定子2、叶片3以及侧面两个配油盘等零件组成。不同之处是定子2的内表面是圆的,且转子1和定子2并不是同心安装,而是有一个偏心量e,当转子转动时,转子径向槽中的叶片在离心力的作用下伸出,使叶片顶部紧靠在定子内表面上。在两侧配油盘上开有吸油和压油窗口,分别与吸、压油口连通,在吸油窗口和压油窗口之间的区域(其夹角应等于或稍大于两个叶片间的夹角)就是封油区,它把吸油腔和压油腔隔开。处在封油区的两个叶片a,b与转子外圆、定子内孔以及侧面两个配油盘形成左、右两个密封工作腔。当转子按图示方向旋转时,右边密封工作腔的容积逐渐增大,通过配油盘上的吸油窗口将油液吸人,而左边密封工作腔的容积逐渐减小通过压油窗口将油液压出。 转子每转一转,每两叶片间的密封工作腔实现一次吸油和压油,故称单作用叶片泵。由图可看出转子受到压油腔的单向液压作用力,使转子轴承承受很大的径向载荷,所以也称为非卸荷式叶片泵。通常这类泵的叶片底部通过配油盘上的通油槽与叶片所在的工作腔相连,因此叶片在压油区时,叶片底部通高压,叶片在吸油区时,叶片底部通低压,从而使叶片顶端和底端因径向运动而对流量产生的影响互相抵消,故叶片的厚度对泵的流量无影响,但由于封油区定子内表面和转子外表面不是同心圆弧,因而会产生流量脉动且困油现象也难以避免,故一般不宜用在高压系统中。单作用叶片泵的优点是它的流量可以通过改变转子和定子之间的偏心距e来调节,当加大e时,密封工作腔的容积变化大,因而输出流量增大。随着e的减小,输出流量相应减小,当e减小到零时,转子和定子同心,密封容积不产生容积变化,因而输出流量为零。此外,还可以通过改变偏心的方向来调换泵的进出油口,从而改变泵的输油方向。调节流量的方式可以是手动的,也可以自动进行。

\section{柱塞泵和柱塞液压马达}

为了提高泵的工作压力,必须改善泵的密封性能和机件的受力情况。柱塞泵和柱塞液压马达是利用柱塞在油缸中作往复运动实现密封容积变化来进行工作的,由于它们的主要构件——柱塞和油缸的密封面形状是圆柱形,易于准确加工,达到很精密的配合,能保证严格的间隙和良好的密封性,因而保证了在高压下工作仍有较高的容积效率。并且其主要零件都承受压力,充分发挥了材料的强度性能,所以柱塞泵可承受的工作压力很高,一般可达30MPa以上。

根据柱塞-油缸排列方式,柱塞泵可分为径向柱塞泵和轴向柱塞泵两大类。

\subsection{径向柱塞泵的工作原理和流量计算}

径向柱塞泵的工作原理如图2-16所示。它是由定子1、转子(缸体)2、配油轴3、衬套4和柱塞5等主要零件构成。沿转子的半径方向均匀分布有若干个柱塞缸,柱塞可在其中灵活滑动。衬套4与转子内孔是紧配合,随转子一起转动。配油轴3是固定不动的,其结构如图中右半部所示,当转子转动时,由于定子内圆中心和转子中心之间有偏心距e,于是柱塞在定子内表面的作用下,在转子的油缸中作往复运动,实现密封容积变化。图示转向,在上半部柱塞向外伸出(柱塞的伸出是靠本身的离心力及吸油腔中低压油的压力或者借助于机械联结装置),缸内密封容积逐渐增大,通过配油轴的油孔c将油液吸入。在下半部柱塞向里推入,缸内密封容积逐渐缩小,通过配油轴上的孔d将油液压出。为了配油,在配油轴与衬套4接触处加工出上下两个缺口,形成吸、压油口a和b,留下的部分形成封油区,封油区的宽度应适当,既能保证封住衬套上的孔,使a和b两油口不通,又能避免产生困油现象。转子每转一转,每个柱塞往复一次,完成一次吸油和压油。柱塞在吸油区除靠本身离心力向外伸出外,往往采用辅助泵向吸油口供低压油(压力一般为$4\times 10^{5} Pa$),使柱塞在低压油液的作用下伸出,以改善泵的吸油条件。沿水平方向移动定子,改变偏心距e的大小,便可改变柱塞移动的行程长度,从而改变密封容积变化的大小,达到改变其输出流量的目的。若改变偏心距e的偏移方向,则泵的输油方向亦随之改变,即成为双向的变量径向柱塞泵了。径向柱塞泵的平均理论流量Q。计算如下:

\begin{equation}
Q_{0}=nq=nq_{z}z=\frac{\pi d^{2}ezn}{2}
\end{equation}

\noindent 式中,
\begin{tabular}[t]{ll}
 $n$ & ————泵的转速;\\ 
 $q$ & ————泵的每转排量; \\
 $q_{z}$ & ————每个柱塞的排量; \\
 $z$ & ————柱塞数目; \\
 $e$ & ————偏心距; \\
 $d$ & ————柱塞直径。
\end{tabular}

实际上径向柱塞泵的输出流量是不均匀的,这是因为每个柱塞径向移动速度是变化的。实践证明柱塞数越多,流量脉动越小,并且当柱塞是单数时,流量比较均匀,所以柱塞数一般为5,7,9,11等。

径向柱塞泵由于柱塞缸按径向排列,造成径向尺寸大,结构较复杂。柱塞和定子间不用机械联结装置时,自吸能力差。配油轴受到很大的径向载荷,易变形,磨损快,且配油轴上封油区尺寸小,易漏油。因此限制了泵的工作压力和转速的提高。尤其是作为液压马达使用时,因其惯量大,对于快速系统、速度频繁变换的系统以及自动调节系统是十分不利的,因而在机床液压系统中较少采用。

\subsection{轴向柱塞泵的工作原理和流量计算}

轴向柱塞泵的柱塞缸是轴向排列的,因此它除了具有径向柱塞泵良好的密封性和较高的容积效率等优点外,它的结构紧凑,尺寸小,惯性小。在机床及其他工业上应用较多,尤其是作为液压马达更是如此。

图2-17所示为轴向柱塞泵的工作原理。它是由倾斜盘1、柱塞2、转子(缸体)3、配油盘4等主要零件组成。缸体上沿圆周均匀分布若干轴向排列的柱塞缸,柱塞可在其中灵活滑动,倾斜盘和配油盘固定不动,传动轴5带动转子3和柱塞2一起转动,柱塞在低压油(由辅助泵供给)的作用下或靠机械联结装置使柱塞紧压在倾斜盘上。由于倾斜盘1相对转子3的轴5倾斜了一个角度$\delta $,当传动轴按图示方向转动时,柱塞在从下到上回转的半周内逐渐向外伸出,使缸内密封容积不断增大,将油液从配油盘上的吸油窗口a吸入。柱塞在从上到下回转的半周内,逐渐向里推入,使缸内密封容积不断减小,将油液从配油盘上的压油窗口b压出。转子每转一转,每个柱塞往复移动一次,完成一次吸油和压油。改变倾斜盘倾斜角度$\delta $的大小,可以改变柱塞往复运动的行程长度,从而改变泵的排量。

由图2-18可看出,轴向柱塞泵的平均理论流量$Q_{0}$的计算式

\begin{equation}
Q_{0}=nq=q_{z}zn=\frac{\pi}{4}d^{2}Dnz\tan \delta  
\end{equation}

\noindent 式中, 
\begin{tabular}[t]{ll}
 $q_{z}$ & ————每个柱塞的排量; \\
 $z$ & ————柱塞数目; \\
 $n$ & ————泵的转速; \\
 $d$ & ————柱塞直径; \\
 $D$ & ————柱塞在缸体上的分布圆直径; \\
 $\delta $ & ————倾斜盘倾角。
\end{tabular}

实际上由于柱塞轴向移动的瞬时速度不是常数,所以泵的输出流量是脉动的,这一点用图2-18来说明。假设柱塞从最高位置开始转动,当缸体转过$\theta $角度时,柱塞的轴向位置,由图可看出,有

