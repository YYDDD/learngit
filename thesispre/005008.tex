子链被剪切,黏度会逐步下降,因此必须定期换油。液压传动中的各种元件和工作液体都在封闭的油路内工作,故障原因一般较难查找。

总的说来,液压传动的优点较多,随着生产的发展,缺点正在逐步加以克服因此液压传动有着广阔的发展前途。

液压传动的优点很多,在工程中的应用很广。在现代飞机的襟翼、尾舵、起落架等的操纵中,采用液压传动是为了获得大的力和力矩,并且单位功率重量轻。在机床中,采用液压传动主要是为了在工作过程中便于无级变速实现自动化和实现换向频繁的往复运动。液压传动在机床上的应用如下:

(1)进给运动。液压传动在机床上的进给运动中应用最为广泛,例如:车床六角车床、自动车床的刀架及转塔刀架的进给;组合机床的动力头、动力滑台的进给等,要求有较大的调整范围,且在工作中能无级调节;C7120车床的纵向进给,最小工作进给量为25mm/min,而纵向快进可达5000mm/min;磨床、刨床工作台往复一次,用液压控制,周期地实现定量进给次,进给量可进行无级调节。

(2)主体运动。龙门刨床的工作台、牛头刨床或插床的滑枕,都可采用液压传动实现所需的高速往复运动,并可减少换向冲击,缩短换向时间。液压传动也可用于自动车床、数控机床等的主轴旋转运动。

(3)仿形装置。车床、铣床、刨床上的仿形加工可以采用液压伺服系统来实现。液压仿形精度可达0.01\textasciitilde0.02mm,灵敏性好,靠模接触力小,寿命长。

(4)辅助装置。工件与刀具的装卸、输送转位、变速操纵,垂直移动的部件平衡等都可采用液压传动来实现。采用液压传动可以简化机床结构,提高机床自动化程度。

(5)数控机床。在数控机床的拖动系统中广泛地采用液压传动,如电液脉冲马达及电液伺服阀等的电液伺服装置。

(6)静压支承。在重型机床、高速机床、高精度机床上采用液体静压轴承液体静压导轨及液体静压丝杆,可以使其工作平稳,运动精度高,是近年来的一项新技术。

随着液压技术的发展,液压传动在机床上的应用将得到不断的扩大和完善。

\chapter{液压油及液压流体力学基础}

液压传动是以油液作为工作介质来传递动力的,为此必须了解油液的物理性质研究油液的运动规律。这一章主要介绍这两方面的内容,着重介绍液压流体力学的一些基础知识。

由于流体力学只研究流体的宏观运动,因此假设流体是连续的,即假设流体是由无限多个一个紧挨着一个的流体质点组成的,流体质点之间没有任何间隙,这种假定称做连续介质假定。根据这一假定就可以把油液的运动参数看做是时间和空间的连续函数,从而可用解析数学去描述这种流体的运动规律,以解决工程实际问题。

液压油同其他流体一样,没有确定的几何形状,它在受切应力作用时,会产生连续不断的变形,即表现出流动性。另外,当流体四周同时受到压力作用时,它具有弹性的性质,即流体能承受压应力。相反,由于流体分子间内聚力很小,基本上不能承受拉应力。

\section{液压油}

下面要介绍的液压油的物理性质(密度、比容、压缩性、黏性等)都是与流体的力学特性关系很密切的性质。

\subsection{流体的密度和比容}

单位体积内所含有的流体质量称为(质量)密度,用符号$\rho $表示。设有一均质流体的体积为$\mathit{V}$,所含有的质量为$\mathit{m}$,则其密度

\begin{equation}
    \rho =\frac{m}{V}
\end{equation}

密度的倒数称为比容,用符号$\mathit{v}$表示,它是单位质量流体所占的体积,即

\begin{equation}
    v=\frac{1}{\rho }
\end{equation}

流体的密度和比容将随着它们所在处的压力和温度而变化,而压力和温度又都是空间点坐标和时间的函数,即

$$\rho =\rho (x,y,z,t) $$
$$v=v(x,y,z,t) $$

由于液体的密度随压力和温度的变化改变极小,一般情况下可忽略不计,因此,常令$\rho $为常数;同样比容也是如此。

\subsection{流体的压缩性及液压弹簧刚性系数}

流体受压力作用其体积减小的性质称为压缩性。流体压缩性的大小用体积压缩系数$\kappa $来表征。一定体积V的流体,当压力增大d$\mathit{p}$时,体积减小了d$\mathit{V}$,则体积压缩系数

\begin{equation}
    \kappa =-\frac{{\rm d}V}{V}\frac{1}{{\rm d}p}=-\frac{1}{V}\frac{{\rm d}V}{{\rm d}p}
\end{equation}

式中,d$\mathit{V}$/$\mathit{V}$表示流体的体积相对变化量,负号表示d$\mathit{V}$与d$\mathit{p}$的变化方向相反,即压力增加时,体积是减少的,反之亦然。

压缩系数$\kappa $的倒数称为体积弹性模量,用符号$\mathit{K}$表示,即

\begin{equation}
    K=\frac{1}{\kappa }=-V\frac{{\rm d}p}{{\rm d}V}
\end{equation}

流体的压缩系数和体积弹性模量的值都是随压力和温度而变化的。对液体来说,它们的变化是很小的,一般忽略不计。

纯液体的压缩系数很小,即其体积弹性模量很大,例如,当压力为$(1\sim 500)\times 10^5\ $Pa时,纯水的平均体积弹性模量$K\approx 2.1\times 10^3\ $MPa,纯液压油的平均体积弹性模量K值则在$(1.4\sim 2)\times 10^3\ $MPa范围内。如果液体中含有非溶解的气体,则其体积弹性模量就会下降较多。在定压力下,油液中混有1$\%$的气体时,其体积弹性模量将降低为纯油的30$\%$左右;如果混有4$\%$的气体,则其体积弹性模量仅为纯油的10$\%$左右。由于油液在使用中很难避免不混入气体,因此工程上常将油液的K值取为700MPa。

如不特殊指明,一般K值都是表示等效体积弹性模量,也即是综合考虑了盛放液压油的封闭容器(包括管道)受压变形引起的容积变化、液压油本身的可压缩性以及混入油中的气体的可压缩性。为了叙述简单,将K值就叫液体的体积弹性模量。

液体的压缩性在液压机械中会产生“液压弹簧效应”。如图1-1所示,当对活塞一端施加的外力变化$\Delta F$时,由于液体是可压缩的,活塞便会沿受力方向产生一个位移量$\Delta l$,使容器中的液体受到压缩。外力消除后,被压缩的液体就会膨胀活塞就会向反方向移动$\Delta $,回复到原来位置。这一现象与机械弹簧受力变形的情况类似,被称之为“液压弹簧效应”。液压弹簧的刚性系数按如下方法计算。

由式(1-4)得出

$${\rm d}p=\frac{K{\rm d}V}{V}=\frac{KA{\rm d}l}{V}$$

\noindent 又

$${\rm d}F={\rm d}pA=\frac{KA^2}{V}{\rm d}l$$

\noindent 故有

\begin{equation}
    K_{\rm h}=\frac{{\rm d}F}{{\rm d}l}=\frac{KA^2}{V}
\end{equation}

\noindent 式中\quad %注意别换行,要和下面的tabular列表连起来
\begin{tabular}[t]{p{1mm}l}
    $\ $\textit{A} &—— 活塞的有效面积;\\
    d\textit{l} &—— 活塞的微小位移量;\\
    d\textit{F} &—— 作用在活塞上外力的变化量;\\
    $K_{\rm h}$ &—— 液压弹簧刚性系数。
\end{tabular}

一般在作液压系统静态分析和计算时,可以不考虑液体的压缩性。但在进行动态分析和计算时,例如液压系统动态性能计算和液压冲击最大压力峰值的计算等,必须重视油的可压缩性这一因素的影响。“液压弹簧效应”还是造成液压传动装置产生低速爬行的一个重要原因。

\subsection{流体的黏性}
\subsubsection{黏性及其表示方法}
液体在外力作用下流动时,液体分子间的内聚力阻碍分子间的相对运动而产生内摩擦力的性质,就是液体的黏性。

以图1-2所示的两块平行平板流动情况为例,观察黏性的作用。上平板以速度$u_0$相对于下平板向右运动,下平板固定不动。经测量平板某法线$y$上各点的流速发现,紧贴在上平板上极薄的一层液体,在流体分子与平板表面的附着力作用下,以相同的速度$u_0$随上平板一起向右运动。紧贴在下平板上极薄的一层液体黏附在下平板上而保持静止。中间各层液体流速则由零逐渐增加,流动快的流层会拖动流动慢的流层,而流动慢的流层又阻止流动快的流层流动,这样层与层之间就因为存在黏性而产生了内摩擦力。这种摩擦力是产生在两流层接触表面之间的剪切力因此,流体的黏性又可说成是决定流体反抗剪切力程度的一种性质。

实验还表明,流体层相对运动时产生的内摩擦力的大小,与流体黏性的大小和接触面积的大小以及流速沿法线的变化率(即速度梯度)有关。其数学表达式为

\begin{equation}
    F_f=\mu A\frac{{\rm d}u}{{\rm d}y}
\end{equation}

\noindent 式中\quad %注意别换行,要和下面的tabular列表连起来
\begin{tabular}[t]{p{1mm}l}
    $F_f$&——流体层相对运动时的内摩擦力;\\
    $\ \mu $&——液体黏性的比例系数;\\
    $\ A $&——流层之间的接触面积;\\
    $\displaystyle\frac{{\rm d}u}{{\rm d}y}$&——流层相对运动时的速度梯度。
\end{tabular}

内摩擦力$F_f$除以接触面积$A$,即得液体内的切应力

\begin{equation}
    \tau =\frac{F_f}{A}=\mu \frac{{\rm d}u}{{\rm d}y}
\end{equation}

\noindent 式(1-7)又称为牛顿液体内摩擦定律。

表示液体黏性大小程度的参数称为黏度,流体的黏度有三种表示方法:

(1)动力黏度(又称绝对黏度)。动力黏度以$\mu $表示,这就是式(1-6)中的黏性比例系数。
它直接表示了流体内摩擦力的大小,其物理意义为:两相邻流体层以单位速度梯度流动时,在单位接触面积上所产生的内摩擦力的大小,即