
\noindent 不变,使执行元件的运动速度保持稳定。实际上由于缸与阀的泄露、减压阀阀芯弹簧力的微小变动与液压力的变化等原因。负载的变化会对速度产生一定影响 。在全负载下,这种回路的速度波动值一般不会超过$\pm$4\%。

图6-10所示是采用调速阀与采用节流阀的进出口节流调速(见图6-10(a))与 旁路节流调速(见图6-10(b))的速度-负载特性曲线比较。曲线1为采用调速阀,曲线2为采用节流阀的特性曲线。由图可见,采用调速阀的节流调速回路其机械特性要硬得多。

这种调速工作回路工作时也有节流损失(包括减压阀和节流阀两者的节流损失)与溢流损失,因调速阀的最小压差要比普通简式节流阀大些,在相同条件下,供油压力也需调得高些,故功率损失也大些。负载恒定时,回路的功率与速度间的关系和简式节流阀的调速回路相同。当负载变化时,由于调速阀使流量不随负载而变,有效功率及回路效率只是负载的函数。在调定节流阀通流截面积下,其溢流量保持不变,泵输出功率与溢流损失都是常量,而有效功率随负载增加而线性上升,节流损失则随负载增加而线性下降。其功率-负载关系曲线如图6-11所示。因此,在变负载情况下,调速阀进出口节流调速回路都是利用节流损失的变化来适应有效功率的变化。

如果采用一个普通定值减压阀后面串接一个简式节流阀装在回油路上,由于定值减压阀出口压力恒定,节流阀两端压力差也能保持不变,从而同样可以达到稳速。显然这种方法不能用于进口调速回路的速度稳定。

(2)采用溢流节流阀的节流调速回路。图6-12是采用溢流节流阀的进口节流调速回路。由差压式溢流阀a和节流阀b组成的溢流节流阀也是一种速度稳定器。定量泵输出的流量,一部分($Q_\text{1}$)经节流阀b进入液压缸,其余流量($\Delta$ $Q_\text{y}$)经差压式溢流阀a流回油箱,c是安全阀,用来防止过载。由式(4-23)可知,当负载 发生变化时,差压式溢流阀自动调节开口量,保持其阀芯两端的压力差(同时也是 节流阀两端的压力差)基本不变,即
\begin{equation}
p_\text{p}-p_\text{1}=\Delta p_\text{j}=\frac{F_\text{s}}{A_\text{y}}\approx\text{常数}
\end{equation}
\noindent 式中\  
\begin{tabular}[t]{ll}
$F_\text{s}$ ——\hspace{1mm} 差压式溢流阀中弹簧力;\\
$A_\text{y}$ ——\hspace{1mm} 差压式溢流阀阀芯截面积。
\end{tabular}\\
从而保证了通过节流阀进入液压缸的流量和活塞的运动速度基本不变。如果负载增大时,工作压力$p_\text{1}$增大,差压式溢流阀内弹簧腔一侧的压力大于无弹簧一侧,使阀芯下移致溢流口关小,泵的供油压力随之增大。反之,当$p_\text{1}$减小时,$p_\text{p}$亦随之减小,使节流阀两端的压力差基本不变。这种回路在全负载下的速度波动值也不大于$\pm$4\%。

由于供油压力随负载的增减而增减,故功率损耗较小,效率较采用普通节流阀或调速阀的节流调速回路为高。但这种溢流节流阀只能在进油路上使用,适用于对运动平稳性要求较高、功率较大的系统。

节流调速系统,无论是采用节流阀或调速阀,其功率损失较大,效率较低是一个共同的缺点,尤其是调速范围较大时,能量的利用率很低,发热很大。为了提高效率,可采用多泵供油、分级调速的方法,即采用二个或三个不同流量的液压泵组成供油系统。工作时 根据速度(即所需流量)的大小,分别由一个、二个或三个泵供油,不供油的泵进行卸荷,同时采用节流阀或调速阀进行无级调速。

图6-13所示为双泵分级节流调速回路。换向阀的四种不同工作位置,对应于泵1供油、泵2供油、双泵供油和双泵供油加差动连接四种回路工作状态,可使液压缸获得四种不同的运动速度,再利用装在旁油路上的调速阀,就可在四种速度之间获得无级调速。但这钟回路的换向阀结构复杂,泵的数量也较多,一般仅用于调速范围较大的中等功率液压系统。
\section{容积调速回路}
容积调速回路由变量泵或变量马达及安全阀等元件组成,它通过改变变量泵的输油量或变量马达的每转排量来实现运动速度的调节。

这种调速回路仅有泵和马达的泄露损失,没有节流元件和溢流量,故没有节流损失和溢流损失,效率高,发热小,一般用于功率较大或对发热要求严格的系统。但变量泵与变量马达的结构比较复杂,成本较高。

根据调节对象的不同,容积调速方法可有三种:\ding{192}变量泵和定量执行元件(定量液压马达或液压缸)组成的容积调速回路;\ding{193}定量泵和变量液压马达组成的容积调速回路;\ding{194}变量泵和变量液压马达组成的容积调速回路。
\subsection{变量泵和定量执行元件组成的调速回路}
如图6-14所示,依靠改变变量泵1的输出流量来调节定量液压马达或液压缸2的运动速度。3是安全阀,只在系统过载时才打开。回路为闭式并通过单向阀4从副油箱补油。

在这种调速回路中,变量泵的流量是根据执行元件的运动速度要求来调节的,需要多少流量就供给多少流量,没有多余流量从溢流阀溢走。当不考虑管路损失时,液压泵的供油压力等于执行元件的工作压力并由负载决定,随负载的增减而增减,允许最大工作压力由安全阀调定。

这种调速回路具有如下特性:

(1)当不计漏损时,液压马达或液压缸的最高与最低运动速度决定于变量泵的最大与最小流量$Q_\text{max}$和$Q_\text{min}$,即\\
液压马达
\begin{equation}
\left.
\begin{aligned}
n_\text{max}&=\frac{Q_\text{max}}{q_\text{m}}\\
n_\text{min}&=\frac{Q_\text{min}}{q_\text{m}}\\
\text{液压缸}
v_\text{max}&=\frac{Q_\text{max}}{A_\text{1}}\\
v_\text{min}&=\frac{Q_\text{min}}{A_\text{1}}
\end{aligned}
\right\}
\end{equation}
变速泵调速范围一般可达40。实际上,调速范围受容积效率的限制。

(2)在各种速度下,液压马达能产生的转矩和液压缸能产生的推力分别为
\begin{equation}
\left.
\begin{aligned}
T_\text{m}=\frac{p_\text{1}q_\text{m}}{2\pi}\\
F=p_1A_1
\end{aligned}
\right\}
\end{equation}
\noindent 式中\  
\begin{tabular}[t]{ll}
$p_\text{1}$ ——\hspace{1mm} 液压马达或液压缸的工作压力,大小由负载决定,最大工作压力由安全阀调定;\\
$q_\text{m}$ ——\hspace{1mm} 液压马达每转排量;\\
$A_\text{1}$ ——\hspace{1mm} 液压缸有效工作面积。
\end{tabular}

当负载转矩或负载一定时,在整个调速范围内,液压马达的输出转矩或液压缸产生的推力不变;由于安全阀的调定压力一定,故其最大输出转矩或最大推力亦不变。因此,这种调速方式称为恒扭矩或恒推力调速。

(3)忽略系统的损失,液压马达或液压缸的有效功率等于泵的输出功率。当负载一定时,执行元件的功率随液压泵输油量呈线性变化。这种调速回路的输出特性如图6-15所示。

(4)液压泵和执行元件的容积效率随负载的增加而下降,泄露增加,因而执行元件的速度将随之下降,故这种回路也有速度随负载增加而下降的特性,速度低时,负载增加,转速容易变成零。影响这一特性的主要因素是泵和执行元件的质量。加大执行元件的有效工作面积,减少元件的泄露,可以提高回路的速度刚性。
\subsection{定量泵和变量液压马达组成的调速回路}
如图6-16所示,定量泵1输油量不变,改变变量液压马达2的排量$q_\text{m}$就可改变液压马达的转速。3是安全阀,4是辅助阀,用以向系统补油。5为辅助泵的溢流阀,其压力调得较低,使主泵吸油腔保持一定的压力,防止空气侵入,改善吸油特性。

这种调速回路有如下特性:

(1)液压马达的最高转速与最低转速,相应于其最小排量与最大排量,即
\begin{equation}
\left.
\begin{aligned}
n_\text{max}&=\frac{Q_\text{p}}{q_\text{m\ min}}\\
n_\text{min}&=\frac{Q_\text{p}}{q_\text{m\ max}}
\end{aligned}
\right\}
\end{equation}

由$T_m=\frac{pq_\text{m}}{2\pi}$,故液压马达的最小排量q$_\text{m min}$不能调得太小,否则输出转矩太小,带不动负载,所以调速范围较小(约为4)。

(2)在各种转速下,泵的供油量不变,且其最大工作压力由安全阀调定,故泵的最大输出功率恒定。如不考虑系统效率,则液压马达的输出功率在整个调速范围内亦恒定,故称恒功率调速。当外负载所要求的工作压力低于调定的最大工作压力时,液压马达的输出功率与输出转矩亦低于其可能输出的最大功率与最大转矩。减少排量,转速提高,输出转矩下降。图6-17所示为这种回路在其安全阀允许的最大工作压力下的输出特性曲线。

(3)不宜采用双向液压马达在运转中实现换向,因为换向时,双向液压马达的偏心量(或倾斜角)必须要经历一个变小$\rightarrow$为零$\rightarrow$反向增大的过程,也就是马达的排量变小$\rightarrow$为零$\rightarrow$变大的过程。输出转矩就要经历转速变高$\rightarrow$输出转矩太小带不动负载转矩而使转速为零$\rightarrow$反向高转速的过程。调节很不方便,

