图4-43所示几种典型结构形式,分别通过轴向移动或旋转阀芯来调节通道截面的大小以调节流量。由于节流口的结构形式不同,在调节过程中,节流口变化规律差异较大,因而调节性能的差别也比较明显。对于图4-43(a),(b),(c)所示节流口形式,结构简单,制造比较方便,但由于通道长,水力半径小,容易堵塞,工作性能较差,只适用于要求不高的场合。而图4-43(d),(e)所示节流口形式,结构较复杂,但它们接近于薄壁小孔式,节流通道短,不易堵塞,工作性能较好,多用于精密调速设备或低速调节稳定性较高的机床。

\section*{二、流量稳定性的分析}

在液压系统工作时,希望节流口大小调节好后,流量稳定不变。但实际上会有变化,特别是流量小时变化较大。影响流量稳定的因素有下列几方面:

\subsection*{1.节流阀前后的压力差$\Delta p$对流量稳定性的影响}

从第一章中式(1-50)、式(1-51)及表1-4可以将节流阀的流量公式综合为

\begin{equation}
\begin{split}
Q & =CA_\text{j}(p_1 - p_2)^\varphi\\
Q & =CA_\text{j}\Delta p^\varphi
\end{split}
\end{equation}

\noindent
式中 $Q$ ——通过节流口孔道的流量;

\ $C$ ——由节流口形式、液体流态、油液性质等因素决定的系数;

\ $A_\text{j}$ ——节流口通流截面积;

\ $\Delta p$ ——节流口前、后的压力差;

\ $\varphi$ ——节流阀指数。对于细长孔$\varphi$=1,对于薄壁小孔$\varphi$=0.5,介于二者之间的$\varphi$=0.5~1。

节流阀的流量特性曲线如图4-44所示。

从式(4-11)可以看出,当节流阀通流截面$A_\text{j}$,一定时,如果节流阀进出口的压力差$(p_\text{1} - p_\text{2})$发生变化,将影响通过节流阀的流量,从而影响它所控制的执行元件的运动速度。

为了深入分析压差变化对流量的影响,我们引用节流阀刚性$k_\text{T}$,它定义为节流阀通流截面$A_\text{j}$一定时,节流阀前后压力差$\Delta p$发生的变化量,与由此而引起通过节流阀流量变化量之比。用数学表达式表示即

\begin{equation}
\begin{split}
k_\text{T} & =\frac{\partial(\Delta p)^{1-\varphi}}{\partial Q}\\ 
\end{split}
\end{equation}

将式(4-11)对$Q$求导数并整理后得

\begin{equation}
\begin{split}
k_\text{T} & =\frac{(\Delta p)^{1-\varphi}}{CA_\text{j}\varphi}\\ 
\end{split}
\end{equation}

从式(4-13)可知,$\varphi$值越小,越接近薄壁小孔,其刚性亦越大;同一节流阀,阀前、后压力差$\Delta p$相同时,开口小的刚性较大;同一节流阀,在节流口开度一定时,其前后压差$\Delta p$越大,则节流阀刚性越大。因此,为了保持节流阀具有一定的刚性,必须保证阀前后具有一定的压差。

不同开口时的流量特性曲线如图4-45所示。

\subsection*{2.温度对流量稳定性的影响}

液压传动的工作介质是矿物油。矿物油的性质,特别是黏性,受温度的影响最大。黏性变化,就引起节流阀的系数$C$发生变化,从而影响通过节流阀的流量。

另外,油液由于温度的变化会加速自身的氧化,生成胶状沉淀物,如沥青等物质,它们与油中的其他机械杂质混合,极易堵塞节流口。而这些杂质对节流口的堵塞往往又是随机的,它将随着因温度的变化所产生沉淀物的多少以及在高温高速下杂质附着与冲刷的情况而变化,其结果就导致通过节流阀的流量时多时少,影响了流量的稳定性。特别是在低速运动时,最突出的现象就是执行元件的“爬行”和周期性的波动。

为了保持液压系统执行元件的运动平稳性,关键在于改善节流阀的流量稳定性。从以上分析可知,我们通常采用薄壁小孔节流口形式,同时控制液压系统的温升和提高油液的过滤精度,减少杂质以改善杂质对节流口的堵塞现象。

\subsection*{3.流量调节范围和最小稳定流量}

节流阀的流量调节范围$R_\text{Q}$是指节流阀最大开口量时的流量$Q_\text{max}$与最小开口量时的最小稳定流量$Q_\text{min}$的比值,即

\begin{equation}
\begin{split}
R_\text{Q} & =\frac{Q_\text{max}}{Q_\text{min}} = \frac{CA_\text{max}\Delta p^\varphi}{CA_\text{min}\Delta p^\varphi} = \frac{A_\text{max}}{A_\text{min}}
\end{split}
\end{equation}

式(4-14)表示流量调节范围是最大开口时通流截面积与最小开口时通流截面积的比值。根据节流口结构形式的不同,通流截面的开口有的是轴向位移的函数,有的是转动角度的函数。目前国产元件的流量调节范围可以很大,如使用轴向三角槽式节流口的节流阀其流量调节范围在100以上。

所谓最小稳定流量就是节流阀在最小的开口量和一定的压差下能够长期保持其调节的流量恒定。目前国产轴向三角槽式节流阀的最小稳定流量在30~50 mL/min ,而薄壁小孔式节流阀的最小稳定流量在20 mL/ min左右。

节流阀的最小稳定流量是节流阀的一项重要性能指标。有些液压系统的执行元件在低速时出现“爬行”现象。所谓“爬行”是指液压传动中,当液压执行元件在低速下运转时可能产生时断时续的运动现象。爬行现象实质上是当一物体在滑动面上作低速相对运动时,在一定条件下产生的停止与滑动相交替的现象,是一种不连续的振动。究其原因是多方面的,如摩擦力的不均匀,负载的变化,环境温度的变化,油液的弹性变形,系统的泄漏,供油量的不稳定等都能产生低速“爬行”现象,而在液压调速系统中,最小供油量的稳定程度将对其是否产生“爬行”起着很大的作用。因此采用节流阀进行流量控制,在小流量时受节流阀最小稳定流量限制。如果采用一种叫计量阀的流量控制元件,就能比较稳定地控制小流量而不受负载、温度以及堵塞的影响。这种计量阀相当于一个柱塞泵,它利用改变柱塞行程大小来改变流量的大小。

\section*{三、调速阀}

调速阀可调节流量,并在调节后起稳定流量的作用。图4-46所示为它的工作原理图和符号图。

从原理图上可以看到,调速阀是由一个定差式减压阀串联一个普通节流阀组成的。压力油以压力$p_1$进入减压阀,其出端压力$p_2$作为节流阀的入端压力,节流阀出端压力$p_3$,也就是调速阀的出口压力。现以调速阀安装在液压缸的进油路上为例说明其工作原理。

$p_1$是由液压泵提供,由溢流阀调定的压力,基本上维持恒定值。$p_3$是由外部负载所决定的调速阀出端压力,其值为

\begin{equation}
\begin{split}
p_3 = \frac{\Sigma F}{A_1}
\end{split}
\end{equation}

调速阀两端的压差$p_1 - p_3$,将式(4-15)代入则得

\begin{equation}
\begin{split}
p_1 - p_3= p_1-\frac{\Sigma F}{A_1}
\end{split}
\end{equation}

\noindent
式中 $p_1$ ——调速阀入端压力;

\ $p_3$ ——调速阀出端压力;

\ $\Sigma F$ ——作用在活塞上的全部外载荷;

\ $A_1$ ——活塞的有效工作面积。

在节流阀一段中,我们已分析了当节流阀两端压差变化时,其调节的流量亦相应发生变化,使速度不稳定。调速阀两端的压差发生变化时是如何保证它所调节的流量恒定的呢?在原理图中,我们标出了油液流经调速阀的压力变化。压力油$p_1$进入调速阀,首先通过其中的减压阀,使压力降为$p_2$,然后通过节流阀使压力变为$p_3$与外部载荷相适应。节流阀两端的压差$\Delta p_\text{j}=p_2-p_3$,现在的问题是如何保持节流阀的压差$\Delta p_\text{j}$恒定。

下面我们来分析一下调速阀中减压阀的作用。从图4-46上可以看到,减压阀阀芯1的顶端弹簧腔b经孔道a与节流阀2的出油端($\text{p}_3$)相通;阀芯1的肩部c和下端d经孔道f,e与节流阀2的入端($\text{p}_2$)相连。当外部载荷增加时,从式(4-15)知$p_3$亦增加,这时$p_3$通过a孔道把$p_3$作用在减压阀的阀芯1的顶端,使顶端作用力增大,破坏阀芯原来的平衡状态,使阀芯下移。减压阀的开口加大,通过减压阀的液阻减小,使$p_2$也增大,而使$\Delta p_\text{j} = p_2-p_3$基本上能保持原来的数值不变。当外部载荷减小时,$p_3$亦减小,同理阀芯1又失去平衡而上移,此时减压阀的开口减小,液流通过减压阀的液阻增大,使$p_2$也跟随降低,同样使$\Delta p_\text{j} = p_2-p_3$仍保持不变,由于减压阀可保持节流阀两端压差为常数(故称定差式减压阀),因而流过节流阀的流量也就稳定不变了。减压阀阀芯上所受力的平衡方程式为

\begin{equation}
\begin{split}
p_2 A_\text{j} = p_3 A_\text{j} + F_\text{s} + G + F\text{f}
\end{split}
\end{equation}

\noindent
式中 $p_2$ ——节流阀人端压力,即减压阀的出端压力;

\ $p_3$ ——节流阀出端压力;

\ $A_\text{g}$ ——减压阀阀芯顶端面积;

\ $F_\text{s}$ ——减压阀恢复弹簧的作用力;

\ $G$ ——减压阀阀芯自重(滑阀垂直安放时考虑);

\ $F_\text{f}$ ——阀芯移动时的摩擦力。

如略去$G$和$F_\text{f}$的影响,可得

\begin{equation}
\begin{split}
\Delta p_\text{j} = p_2 - p_3 = \frac{F_\text{s}}{A_\text{g}}
\end{split}
\end{equation}

考虑到$F_\text{s}$,是作为恢复作用的,该弹簧的刚性较小,当阀芯移动时,由于弹簧的压缩量的变化所附加的弹簧作用力的变化是很小的,即$F_\text{s}$,近似为常数,因而可认为$p_2-p_3$是一常数,则通过节流阀的流量也是个常数,亦即通过调速阀的流量是个常数,这就保证了执行元件运动速度的稳定性。

调速阀正常工作时,要求调速阀两端的压差至少为(4~5)$\times 10^5$ Pa。这是因为压差过小,调速阀中的减压阀阀芯在弹簧力作用下,使减压阀开口全部打开,不能起到调节和稳定节流阀前压力的缘故。这种调速阀亦可用在回油路上,用相同的原理保持回油流量不变。

调速阀与普通节流阀一样,对温度和堵塞现象都是敏感的,为了弥补温度对流量稳定性的影响,可以采用带温度补偿装置的调速阀。所谓温度补偿装置的原理,就是采用一温度膨胀系数较大的材料附加控制节流开口的大小。我们知道,温度升高后,黏度降低,通过节流口的流量将增大,而受热膨胀的热敏元件推动节流阀阀芯,使节流开口减小,限制流量的增大。反之,若温度降低,黏度增加,流量将减小,此时热敏元件收缩拉回节流阀芯,使节流开口增大,使流量维持在温度变化前的数值。利用这种方法,可部分地补偿由于温度的变化而造成流量的变化。如要根本解决问题,则必须控制温度的变化。温度补偿调速阀的工作原理与调速阀相同。其最小稳定流量为20 mL/ min,其节流口形式多采用薄壁缝隙式,壁厚在0.07~0.09 mm ,缝隙的最小部分为0.13~0.16 mm,结构形式如图4-43(e)所示。

由上分析,我们可用图4-47来对比节流阀与调速阀的性能。

\section*{四、溢流节流阀}

除了上述调速阀可以比较稳定地控制流量以外,还可采用一种定差溢流阀与节流阀并联组成的溢流节流阀来控制流量,同样可以达到稳定流量的效果,而这种阀仅安装在进油管道上。图4-48所示为它的工作原理及符号图。

压力油$\text{p}_1$进入溢流节流阀后,一路经节流阀4从出口流出进入主油路系统($\text{p}_2$),一路经溢流阀的溢流口流回油箱。溢流阀阀芯大端的弹簧腔α与节流阀4的出口($\text{p}_2$)相连,而其肩部b与小端部的c腔接通入口压力油$\text{p}_1$。当出口压力$\text{p}_2$增大时,溢流阀阀芯α腔压力增加,阀芯3下移,溢流口减小,液阻加大,使液压泵提供的压力油$\text{p}_1$增加,因而使节流阀前后的压差$\Delta p_\text{j} = $
