%本部分页码为126-129页

\section{油箱和热交换器}
\subsection{油箱}

油箱的作用是保证供给系统充分的工作油液,同时具有沉淀油液中的污物、逸出油中的空气和散热等作用。为此,它需要有一定大小的容积。通常油箱的有效容积取为液压泵每分钟流量的$3\thicksim 6$倍。液压泵流量大、压力低或允许的油温升高时,取下限,反之取上限。如有必要,油箱有效容积应根据散热需要来确定(见第八章中的算法)。

油箱的结构形式有总体式和分离式两种。总体式油箱是利用机床床身的内腔作为油箱,这种油箱结构紧凑,不占地面,各处漏油易于回收,但增加了床身结构的复杂性,维护不便,散热不良,由于油温升高引起床身热变形,会降低机床的精度。分离式油箱是设置一个与机床分开的单独油箱,可减少油的温升和电机、液压泵的振动对机床工作精度的影响,精密机床一般都采用这种形式。

图$5-12$所示为一分离式油箱的结构简图。图中$1$为吸油管,$4$为回油管,中间有两个隔板$7$ 和$9$,隔板$7$用作阻挡沉淀杂物进入吸油管,隔板$9$用来阻挡泡沫进入吸油管。沉淀污物可从油阀$8$放出。加油滤油网$2$设在回油管一侧的上部。盖$3$上有通气孔。$6$是油面指示器。当彻底清洗油箱时可将上盖$5$卸开。

进行油箱的结构设计时应注意几个问题:

(1)油箱应有足够的刚度和强度。油箱一般用$2.5 \thicksim 4$ mm的钢板焊接而成,尺寸高大的油箱要加焊角板、筋条以增加刚度。油箱上盖板若安装电击传动装置、液压泵和其他液压元件,则盖板不仅要适当加厚,而且还要采取措施局部加强。液压泵和电机直立安装时,振动一般比横放安装时要好。

(2)吸油管和回油管之间的距离应尽量远些,两管最好要用隔板隔开,以增加油液循环流动的距离,提高散热效果,并使油液有足够长的时间放出气泡和沉淀杂质。隔板的高度约为最低油面高度的$2/3$。

吸油管离油箱底面的距离应不小于管径的$2$倍,距油箱侧面应不小于管径的$3$倍,以便油流畅通。回油管应插入最低油面一下,以防回油冲入液面使油中混入气泡。回油管管端切成$45^\circ$角,以增大排油口面积,排油口应面向箱壁,利于散热。泄油管不应插入油中,以免增大元件泄露腔处的背压。

(3)要采取措施保护箱内油液清洁。油箱上盖板与油箱四周都严密密封,盖板上的各种安装孔也都要密封,以防灰尘杂物进入油箱污染油液。加油口上要装滤油器,通气孔上须装空气滤清器。吸油管入口处最好装粗滤油器,它的额定通过流量应为液压泵流量的$2$倍以上。

(4)要便于清洗和为维护。为便于排放污油,油箱箱底应做成倾斜形,且与地面保持一定距离。在箱底最低处安装放油阀或放油塞。油箱结构还应考虑能方便地拆装滤油器和清洗内部。油箱侧壁应安装观察油面高低地油面指示器,以便适时补充油液。

(5)油箱内壁应涂上耐油地防锈涂料,以延长油箱寿命和减少油液污染。

(6)如有必要安装热交换器、温度计等附加装置,需要合理确定它们的安放位置。

\subsection{热交换器}

为了提高液压系统的工作稳定性,应使系统在允许的温度下工作并保持平衡。液压系统的油液工作温度一般希望保持在30 $ \thicksim$ 50$^\circ$C范围内,最高不超过60$^\circ$C ,最低不低于15$^\circ$C。油温过高将使油液变质,加速其污染,同时油的黏性和润滑能力降低,增加油液的泄露,缩短液压元件的寿命。油温过低,则液压泵启动时吸油有困难,系统的压力损失也增大。

如果液压系统单靠自然散热不能使油温限制在允许值以下,就必须安装冷却器;反之,如果环境温度太低无法使液压泵正常启动,就必须安装加热器。冷却器和加热器统称为热交换器。
\subsubsection{冷却器}
冷却器按其使用冷却介质的不同分为风冷、水冷和氨冷等多种形式。

风冷式冷却器构造比较简单, 它通常由许多带散热片的管子所组成的油散热器和风扇两部分构成。油散热器也可用汽车散热器来代替。风冷式冷却器可节约用水,但它的冷却效果较差。

水冷式冷却器有多种式样。最简单的一种是在油箱中安置蛇形水冷管,冷水从蛇形管里通过,把油的热量带走。这种冷却器的散热效率低,耗水量大,运转费用高。

液压系统中采用得较多的是多管式水冷却器,其结构如图$5-13$所示。油从右端上部油口c进入冷却器,经由左端上部油口b流出。冷却水从右端盖$4$中央的孔d进入,经过多根水管$3$的内部,从左端盖$1$上的孔a流出。油在水管外面流过,三块隔板$2$用来增加油的循环路线长度,以改善热交换的效果。


近来出现一种翅片管式冷却器,即在水管外面增加横向或纵向的散热翅片,使传热面积增加,其传热效率比直管式提高数倍。

冷却器一般应安装在回油路或在溢流阀的溢流管路上,图$5-14$所示是其正确的安装位置。液压泵输出的压力油直接进入液压系统,已经发热的回油和溢流阀溢出的热油一起通过冷却器1进行冷却后,回到油箱。单向阀$2$是保护冷却器用的。当不需要进行冷却时可将截止阀$3$打开,使油直接回油箱。

\subsubsection{加热器}
液压系统中的加热器一般都采用电加热器。这种加热器结构简单,使用方便,可根据所需的最高和最低温度进行自动调节。电加热器外形呈长管状,常横装在油箱侧壁上,用法兰盘固定。由于油液是热的不良导体,因此单个加热器的容量不能太大,以免周围油温过高,使油质发生变化。如有需要,可在油箱内多装几个加热器,使加热均匀。

\begin{center}
 \textbf{思考题和习题}
\end{center}

5-1\ \ 设蓄能器预充压力为$9$ MPa,并在绝对压力$10 \thicksim 20 $ MPa中间工作,若要求供油量为$5$ L,试求该蓄能器的尺寸。

5-2\  \ 在调整阀和液压伺服阀的入口油路上应安装什么样的滤油器?

5-3\  \ 设管道流量$Q = 25 $ L/min,若限制管内流速$v\leq 5$m/min,问应选用多大内径的油管?

5-4\ \ 确定油箱的容积应考虑哪些因素?

\chapter{液压传动系统的速度调节}

液压传动系统中的速度调节是液压系统中的核心部分,它的工作性能优劣对系统起着决定性的作用。速度调节包括调速回路、速度换接回路、快速运动回路等。

\section{调速回路}

调速回路用于工作过程中调节执行元件的运动速度,它对液压传动系统的性能好坏起决定性作用,故在机床液压系统中占有突出地位,往往是机床液压系统的核心部分。

调速回路应能满足如下基本要求:

(1)在规定的调速范围内能灵敏、平稳地实现无级调速,具有良好的调节特性。

(2)负载变化时,工作部件调定速度的变化要小(在允许范围内),即具有良好的速度刚性(或速度-负载特性)

(3)效率高,发热少,具有良好的功率特性。

液压缸的速度$v$与输入流量$Q_{1}$及缸有效工作面积$A_{1}$间的关系为
\[
v = \frac {Q_{1}} {A_{1}}
\]

液压马达的转速$n$与输入流量$Q_{1}$及马达排量$q_{m}$之间的关系为
\[
n = \frac {Q_{1}} {q_{m}}
\]

可见,改变输入执行元件的流量$Q_{1}$,或改变液压缸有效工作面积$A_{1}$和液压马达每转排量$q_{m}$,都可以达到调速的目的。改变液压缸有效工作面积$A_{1}$较困难,改变排量$q_{m}$在变量液压马达上则容易做到,而最易实现和广泛应用的是改变输入流量$Q_{1}$。

目前在机床液压系统的调速回路中,主要有以下三种基本调速形式:

(1)节流调速。采用定量泵供油,由流量控制阀调节进入执行元件的流量来实现调速。

(2)容积调速。通过改变变量泵或变量马达的排量来实现调速。

(3)容积节流调速。采用压力反馈式变量泵供油,配合流量控制阀进行节流来实现调速,又称联合调速。

就油路的循环形式而言,调速回路又有开式与闭式之分。开式回路是液压泵从油箱吸油,执行元件的回油直接通油箱(见图$6-1$、图$6-2$、图$6-3$)。这种回路形式结构简单,油液在油箱中能得到较好冷却和沉淀杂质故应用最广。但油箱尺寸大,油液与空气接触易使空气混入















