
液体在直管中作紊流运动时,沿程损失仍按式(1-45)计算,但如何取$ \lambda  $值就相当复杂了,只能按经验公式或实验曲线得到。

当Re较低时,由于在管道的管壁附近有一层层流的边界层,把管壁的粗糙度掩盖住,因而管壁粗糙度将不影响液体的流动,这时似乎液体流过一根光滑管,或称水力光滑管。这时$ \lambda  $仅和Re有关,和粗糙度无关,即 $ \lambda =f\left( \text{Re} \right)  $ 。

当Re増大时,层流边界层厚度减薄,小于管壁粗糙度,管壁粗糙度就突出在层流边界层以外,对液体的紊流压力损失产生影响,这时的$ \lambda  $ 将和Re以及管壁的相对粗糙度$ \bigtriangleup /d $ ($ \bigtriangleup  $ 为管壁的绝对粗糙度,d为管的内径)有关,即$ \lambda =f\text{(Re,}\varDelta /d\text{)} $。

在不同的雷诺数范围内,$ \lambda  $ 值也可按下列经验公式求出
$$ \lambda =0.032+0.221\text{Re}^{-0.237}\ \ \ \ \left( 3\times 10^6>\text{Re}>10^5 \right)  $$ 
$$ \lambda =0.316\text{Re}^{-0.25}\ \ \ \ \left( 10^5>\text{Re}>4000 \right)  $$ 
$$ \lambda =\text{(2}\lg \frac{d}{2\bigtriangleup}+1.74\text{)}^{-2}\ \ \ \ \left( \text{Re}>900\frac{d}{\varDelta} \right)  $$ 

2.局部压力损失

局部压力损失是液体流经如阀口、弯头及通流截面变化等局部阻力处所引起的压力损失。 流体通过这些局部阻力处时流速大小和方向会产生急剧变化,流体质点间产生撞击,形成旋涡区,从而产生了能量损失。

局部损失除少数几种能在理论上作一定的分析计算外,一般都依靠实验方法求得。

下面以截面突然扩大时的局部损失为例进行计算。如图1-23所示,假设是理想流体不可压缩且作恒定流动,因为是紊流,动能修正系数和动量修正系数均取1, 列截面1-1和2-2的伯努利方程。
$$ \frac{p_1}{\rho g}+\frac{v_{1}^{2}}{2g}=\frac{p_2}{\rho g}+\frac{v_{2}^{2}}{2g}+h_{\zeta} $$ 
式中$ h_{\zeta} $ 为单位质量液体的局部压力损失(由于路程短不计沿程损失)

将选截面1-1和2-2间的核心区I为控制体,根据动量方程,有
$$ p_1A_1+P_0\left( A_2-A_1 \right) -p_2A_2=\rho Q\left( V_2-V_1 \right)  $$
\noindent 由实验得知$ p_0\approx p_1 $ ,则上式可化简为
$$ p_1-p_2=\rho v_2\left( v_2-v_1 \right)  $$ 
\noindent 将式(b)带入式(a)中可求得
$$ h_{\zeta}=\frac{v_2\left( v_2-v_1 \right)}{g}+\frac{v_{1}^{2}-v_{2}^{2}}{2g} $$ 
\noindent 化简上式,并将$$ v_2=\frac{A_1}{A_2}v_1 $$ 代入,得
$$ h_{\zeta}=\frac{\left( v_1-v_2 \right) ^2}{2g}=\left( 1-\frac{A_1}{A_2} \right) ^2\frac{v_{1}^{2}}{2g} $$ 
\noindent 令
$$ \zeta =\left( 1-\frac{A_1}{A_2} \right) ^2 $$ 
\noindent 称为突然扩大时的局部损失系数,则
$$ h_{\zeta}=\zeta \frac{v_{1}^{2}}{2g} $$
由式(1-46)不难看出,局部损失系数仅与通流面积A1与A2比值有关,而与速度,黏性(或与雷诺数)无关。常见的局部损失系数如图1-24所示。

当$ A_2\gg A_1 $ 时,$ \zeta =1 $ ,因此突然扩大截面处的局部能量损失为 ,这说明进入突然扩大截面处液体的全部动能会因液体扰动而全部损失掉,变为热能而散失。

由于各种局部损失的实质是一样的,因此,可以将突然扩大的局部压力损失公式(1-47)作为普遍的局部压力损失计算公式

3.管路系统总能量损失

管路系统中总能量损失等于系统中所有直管沿程能量损失之和与局部能量损失之和的叠加,即
$$ h_w=\varSigma \lambda \frac{l}{d}\frac{v^2}{2g}+\varSigma \zeta \frac{\nu ^2}{2g} $$ 
$$ \varDelta p=\varSigma \lambda \frac{l}{d}\frac{\rho v^2}{2}+\varSigma \zeta \frac{\rho v^2}{2} $$ 
上式仅在两相邻局部损失之间的距离大于管道内径10〜20倍时才是正确的,否则液流受前一个局部阻力的干扰还没有稳定下来,就又经历后一个局部阻力,它所受扰动将更为严重,因而会使式(1-48)算出的压力损失值比实际数值小。

由前推导的计算压力损失的公式中可以看出,层流直管中的沿程损失与流速v呈一次方关系,局部损失则与流速v的平方成正比,因此,为了减少系统中的压力损失,管道中液体的流速不应过高

为了减少压力损失,还应尽量减少截面变化和管道弯曲,管道内壁力求光滑,油液黏度适当。

二、流量公式

1.孔口流量公式

在液压传动中,经常装有断面突然收缩的装置,称为节流装置(如节流阀)。突然收缩处的流动叫节流。一般均采用各种形式的孔口来实现节流。液体流过节流口时要产生局部损失,使系统发热,油液黏度减小,系统的泄漏增加,这是不利的一面。但是这种节流装置能实现对压力和流量的控制。

液体流经小孔的情况,可分为薄壁小孔和细长小孔,介于二者之间的孔叫短孔。它们的流量计算和流量压力特性有相同之处,也有区别。下面将分别进行分析。

(1)薄壁小孔的流量公式。所谓薄壁小孔是指小孔的长度l与直径d 之比$ l/d\le 0.5 $ 的孔。如流量阀中的节流口,静压支承中的小孔节流器都是薄壁孔,一般都将孔口边缘作为刃口形式,如图1-25所示“液流在小孔上游大约d/2处开始加速并从四周流向小孔,贴近管壁的液体由于惯性不会作直角转弯而是向管轴中心收缩,从而形成收缩断面,大约在小孔出口d/2的地方,形成最小收缩截面$ A_e $,通常把最小收缩面积与孔口截面积之比称为收缩系数,即
$$ C_e=\frac{A_e}{A_0} $$
截面收缩的程度取决于Re、孔口及边缘形状、孔口离管道及容器侧壁的距离等因素。如圆形小孔,当管道直径与小孔直径之比$ d/d_0\ge 7 $ 时,称完全收缩,此时流束的收缩不受大孔侧壁的影响。反之,当$ d/d_0<7 $ 时,称为不完全收缩,由于这时管壁与小孔较近,侧壁对收缩的程度有影响。

如图1-25所示,小孔前截面1-1,其相应参数为$ A_1 $ ,$ p_1 $ ,$ v_1 $ ;小孔后截面2-2,其相应参数为$ A_2 $ ,$ p_2 $ ,$ v_2 $ ,收缩处的参数为$ A_e $ ,$ p_e $ ,$ v_e $ 。

选取轴心线为参考基准,列写截面1-1及2-2的伯努利方程,则有

