力作用下静止液体中的等压面是一个水平面。

为了更清晰地说明静压力的分布规律,将式(1-14)按坐标$Z$变换,即以$h=Z_0-Z$代入式(1-15)整理后得
$$Z+\frac{p}{\rho g}=Z_0+\frac{p_0}{\rho g}$$

对于某一基准面来说,自由液面的高度$Z_0$及压力$p_0$均是常数,因此
$$Z+\frac{p}{\rho g}=\text{常数}$$

\section{压力的表示方法及单位}

液体压力通常有绝对压力、相对压力(表压力)和真空度三种表示方法(见图1-5)。

在地球表面上,一切物体都受大气压力的作用,而且是自成平衡的,因此绝大多数的压力表测得的压力值均为高于大气压力的那部分压力,即相对压力,又称表压力。绝对压力是以绝对真空为基准来进行度量的,由式(1-14)所表示的压力即是绝对压力。

如果液体中某点的绝对压力小于大气压力,就说这一点具有真空,而其不足大气压力的那部分数值称为该点的真空度。由此可知,真空度就是负的相对压力,其最大值不超过1个大气压($1.013 \times 10^5$Pa)。

绝对压力、相对压力及真空度的三者之间的关系为
$$ \mbox{绝对压力}=\mbox{相对压力}+\mbox{大气压力}$$
$$ \mbox{真空度}=\mbox{绝对压力}-\mbox{大气压力}=\mbox{负的相对压力}$$

压力的单位在国际制(SI)中为$\mbox{牛}/\mbox{米}^2(N/m^2)$,称为帕斯卡,简称帕(Pa)。

\section{帕斯卡原理——静压传递原理}
由静力学基本方程式(1-14)可知,盛放在密闭容器内的液体,其外加压力$p_0$发生变化时,只要流体仍然保持原来的静止状态,液体中任一点的压力,均将发生同样大小的变化。也就是说,在密闭的容器内,施加于静止液体上的压力将以等值同时传到液体各点。这就是静压传递原理或帕斯卡原理。

在液压系统中,外力作用所产生的压力远远大于由液体自重所产生的压力,因此常将液体自重产生的压力忽略不计,而认为在密闭容器中静止液体的压力处处相等。

根据帕斯卡原理可推导出推力与负载的关系。如图1-6所示,图中垂直液压缸、水平液压缸的截面积分别为$A_1$和$A_2$,活塞上作用的负载与推力为$F_1$和$F_2$。由于两缸互相连通,构成一个密闭容器,按帕斯卡原理,缸内压力处处相等,$p_1=p_2$,于是
\begin{equation}
F_2=\frac{A_2}{A_1}F_1
\end{equation}

只要$F_2$满足公式(1-16)就可推动负载$F_1$ ,而如果没有负载$F_1$,不计其他各种阻力,不论怎样推动水平液压缸的活塞,也不能在液体中形成压力,说明液压系统中的压力是负载决定的,这是液压传动中的一个基本概念。

\section{液体静压力作用在固体壁面上的力}
静止液体和固体壁面相接触时,固体壁面上各点在某一方向上所受静压作用力的总和,便是液体在该方向上作用于固体壁面上的力。

固体壁面为一平面,如不计重力作用,即忽略$\rho gh$项,平面上各点处的静压力大小相等,则作用在固体壁面上的力等于静压力与承压面积的乘积,即$F=pA$,其作用方向垂直于壁面。

如果承受压力的表面为曲面,由于压力总是垂直于承受压力的表面,因此作用在曲面上各点的压力互相间是不平行的,但大小仍然是相等的,要计算在曲面上的合力,就必须明确要计算的是哪一个方向上的力。下面以图1-7所示液压缸为例计算静压力作用在液压缸缸筒右半壁上$x$方向的力。

设$r$为液压缸的内半径,$l$为液压缸有效长度,在液压缸上取一微小窄条面积$dA$,则$dA=lds=lrd\theta$,静压力作用在这微小面积上的力$dF$在$x$方向的投影
$$dF_x=dF\cos \theta=pdA\cos \theta=plr\cos d\theta$$
液压缸右半壁上$x$方向的总作用力
$$F_x=\int_{+\frac{\pi}{2}}^{-\frac{\pi}{2}}  \,dF_x=\int_{+\frac{\pi}{2}}^{-\frac{\pi}{2}}  plr\cos \theta d\theta$$
其值等于静压力与曲面在垂直面上投影面积$2lr$的乘积。由此可以得出结论:曲面上液压作用力在某一方向上的分力等于静压力与曲面在该方向投影面积的乘积。


\chapter{流动液体的基本力学性能}
本节讨论液体在流动时的运动规律、能量转换和流动液体对固体壁面的作用力等问题,主要讨论三个基本方程——连续方程、能量方程和动量方程。这三个方程是刚体力学中质量守恒、能量守恒及动量守恒在流体力学中的具体体现。前两个用来解决压力、流速及流量之间的关系问题,后一个则用来解决液体与固体壁面之间的相互作用力问题。
\section{基本概念}
\subsection{理想液体、恒定流动和一维流动}
所谓理想液体是一种假想的没有黏性、不可压缩的液体。事实上,液体是既有黏性也可压缩的。之所以作这种假设是由于液体在流动时考虑黏性的影响会使问题变得相当复杂,而液体的可压缩性又很小。为了分析问题方便,先作这样的假设以推导出一些基本方程,然后再通过实验来修正或补充这些方程,这是实际工程中最常用的方法。

按液体运动时液体中任意一点处的参数与时间的关系来区分,可分为恒定流动(稳定流动、定常流动或非时变流动)和非恒定流动。所谓恒定流动是指液体运动参数仅是空间坐标的函数,不随时间变化,即在任何时间内,通过空间某一固定点的各液体质点的速度、压力和密度等参数都保持某一常数。否则就称为非恒定流动。研究液压系统静态性能时,可以认为液 体作恒定流动,但在研究其动态性能时则必须按非恒定流动来考虑。

一般地说,流体的运动都是在三维空间内进行的,运动参数是三个坐标的函数,称这种流动为三维流动或三元流动。依此类推即有二维流动和一维流动。一维流动最简单,但是严格地说一维流动要求液流截面上各点处的速度矢量完全相同,这种情况在现实中不存在。但当管道截面积变化很缓慢,管道轴心线的曲率不大,管道每个截面取液流速度平均值时,一般都可近似地按一维流动处理。
\subsection{流线、流束和通流截面}
流线是某一瞬时液流中一条条标志其质点运动状态的曲线,在流线上各点处的瞬时液流方向与该点的切线方向重合(见图1-8)。对于恒定流动,流线形状不随时间变化。由于液流中每一点处每一瞬时只能有一个速度,因而流线不能相交,也不能转折,它是一条光滑的曲线。

如果通过某截面$A$上所有各点画出流线,这些流线的集合就构成流束如图1-9所示。因为流线不能相交,所以流束内外的流线均不能穿越流束表面。当面积$A$无限小时,这个流束称为微小流束。微小流束截面上各点处的运动速度可以认为是相等的。

流束中与所有流线正交的截面称为通流截面(见图1-9中的$A$面和$B$面),截面上每点处的流动速度都垂直于这个面。

\subsection{流量及平均流速}
单位时间内流过某通流截面的液体体积称为流量。对微小流束而言,通流截面$dA$上的各点流速$u$认为是相等的,则通过$dA$的微小流量为
$$dQ=udA$$
对此进行积分,可得流经通流截面$A$的总流量为
$$Q=\int_{A}^{}  udA$$
要求得$Q$的值必须先知道流速$u$在整个通流截面上的分布规律,这实际上是很难求得的,为便于解决问题,在液压传动中,常采用一个假想的平均流速来求流量。认为通流截面上所有各点的流速均等于平均流速,即
\begin{equation}
Q=\int_{A}^{}  udA=vA
\end{equation}
故平均流速
\begin{equation}
 v=\frac{Q}{A}
\end{equation}

有了上述基本概念就能方便地理解复杂的流体力学和解决实际的工程问题。
\section{流体的流动状态、雷诺数}
实际流体是有黏性的,其流动情况如何,这要涉及流体运动的物理本质。19世纪末,英国物理学家雷诺通过大量实验发现,液体的流动具有两种基本的状态,即层流和紊流。其实验装置如图1-10所示,水箱4由进水管2不断供水,多余的水由隔板1上部流出,以使实验过程中保持恒定水位。在水箱下部装有玻璃管6和开关(水龙头)7,在玻璃管进口处放置与颜色水箱3相连的小导管5。

实验时首先将开关7打开,然后打开颜色水导管的开关,并用开关7来调节玻璃管6中水的流速。当流速较低时,颜色水的流动是一条与管轴平行的清晰的线状流,和大玻璃管中的清水互不混杂(如图1-l0(a)所示),这说明管中的水流是分层的,这种流动状态叫层流。逐渐开大开关7,当玻璃管中的流速增大至某一值时,颜色水流便开始抖动而呈波纹状态(如图(1-l0b)所示),这表明层流开始破坏。再进一步增大水的流速,颜色水流便和清水掺混在一起(如图1-l0(c)所示),这种流动状态叫紊流。

如果将开关7逐渐关小,则玻璃管中的流动状态便又从紊流向层流转变,只是其流速的临界值并不相同。

由层流过渡到紊流液体的速度叫上临界速度;由紊流过渡到层流的速度称为下临界速度;在上、下临界速度之间,液流处于过渡状态,或称变流,变流是一种不稳定的流态,一般按紊流处理。

由相似理论可以得出:层流与紊流是两种性质不同的流动状态。 层流时黏性力起主导作用,惯性力与黏性力相比不大,液体质点受黏性的约束,不能随意运动;紊流时惯性力起主导作用,液体质点在高速流动时黏性对它的约束就大为减小了。

实验证明,液体在圆管中流动是层流还是紊流与管内平均流速、管径及液体黏度有关。雷诺从一系列的实验发现:不论平均流速$v$、管径$d$及液体运动黏度$\nu $如何变化,液流状态仅与无量纲组合数$vd/\nu $有关,这个组合数叫雷诺数,以$Re$表示,即

a try for git