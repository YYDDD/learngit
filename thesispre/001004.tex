\chapter{绪论}

    液压传动技术早在18世纪末就已开始应用,从1795年英国制成第一台水压机,至今已有200多年的历史。液压传动技术的研究被各国普遍重视,并应用于各个工业部门,只是近60年的事,因此液压传动与机械传动相比,还是比较年轻的技术。

    第二次世界大战以后,液压元件迅速发展,性能也日趋完善,因而,液压传动技术开始得到广泛的应用。自从出现了精度高及快速响应的伺服阀和伺服控制系统以后,液压传动技术的应用就更为大家所重视。液压传动具有许多突出的优点,目前已广泛应用在机械制造、工程建筑、交通运输、矿山、冶金、航空、航海、军事、轻工、农机等工业部门,也被应用到宇宙航行、海洋开发、预测地震等方面。在机床行业中,液压传动的应用更为普遍,如应用在磨床、车床、拉床刨床、镗床、锻压机床、组合机床、数控机床、仿形机床、单机自动化、机械手和自动线等机械加工设备中。

    从发展趋势来看,液压传动正向着高压化、高速化、集成化、大流量、大功率、高效率、长寿命、低噪声方向发展。为此,一些主要液压元件生产国,注意了在下列几方面进行理论性研究:液压回路中的动态特性;元件的噪声、振动和气蚀;液压油的难燃性、充气性、压缩性和污染;阀的稳定性、流量系数、液动力;元件的内、外泄漏;提高元件的低温特性;提高元件的寿命;微电子与数字计算机在电液自动控制系统的应用等。

\section{机床液压传动系统}
\subsection{液压传动的工作原理}

    液压传动在机床上应用很广,具体的结构也比较复杂。下面介绍一个简化了的机床液压传动系统,用以概括地说明液压传动的工作原理。

    图0-1所示为简化了的机床工作台往复送进的液压系统图。液压缸10固定不动,活塞8连同活塞杆9带动工作台14可以做向左或向右的往复运动。图中所示为电磁换向阀7的左端电磁铁通电而右端的电磁铁断电状态,将阀芯推向右端。液压泵3由电动机带动旋转,通过其内部的密封腔容积变化,将油液从油箱1中,经滤油器2、油管15吸入,并经油管16、节流阀5、油管17、电磁换向阀7、油管20,压入液压缸10的左腔,迫使液压缸左腔容积不断增大,推动活塞及活塞杆连同工作台向右移动。液压缸右腔的回油,经油管21、电磁换向阀7、油管19排回油箱。当撞块12碰上行程开关11时,电磁换向阀7左端的电磁铁断电而右端的电磁铁通电,便将阀芯推向左端。这时,从油管17输来的压力油经电磁换向阀7,由油管21进入液压缸的右腔,使活塞及活塞杆连同工作台向左移动。液压缸左腔的回油,经油管20、电磁换向阀7、油管19排回油箱。电磁换向阀的左、右端电磁铁交替通电,活塞及活塞杆连同工作台便
循环往复左、右移动。当电磁换向阀7的左、右端电磁铁都断电时,阀芯在两端的弹簧作用下,处于中间位置。这时,液压缸的左腔、右腔、进油路及回油路之间均不相通,活塞及活塞杆连同工作台便停止不动。由此可见,电磁换向阀是控制油液流动方向的。

    调节节流阀5的开口大小,可控制进入液压缸的油液流量,改变活塞及活塞杆连同工作台移动的速度。

    在进油路上安装溢流阀6,且与液压泵旁路连接。液压泵的输出压力,可从压力表4中读出。当油液的压力升高到稍超过溢流阀的调定压力时,溢流阀开启,油液经油管18排回油箱,这时油液的压力不再升高,稳定在调定的压力值范围内。溢流阀在稳定系统压力和防止系统过载的同时,还起着把液压泵输出的多余油液排回油箱的作用。

    电磁换向阀7的阀芯两端弹簧腔泄漏油,通过油管22(泄漏口)排回油箱。

%%%%%%%%%%%%%%%%%%%%%%%%%%%插图%%%%%%%%%%%%%%%%%%%%%%%%%%%%%%


    在图0-1所示液压系统中,所采用的液压泵为定量泵,即在单位时间内所输出压力油的体积(称为流量)为定值。定量泵所输出的压力油,除供给系统工作所需外,多余的油液由溢流阀排回油箱,能量损耗就增大。为了节约能源,可以采用在单位时间内所输出的流量根据系统工作所需而调节的变量泵。如果机床液压系统的工作是旋转运动,则可以将液压缸改用液压马达。

    通过上述例子可以看到:

    (1)液压传动是以有压力的油液作为传递动力的介质,液压泵把电动机供给的机械能转换成油液的液压能,油液输入液压缸后,又通过液压缸把油液的液压能转变成驱动工作台运动的机械能。

    (2)在液压泵中,电动机旋转运动的机械能是依靠密封容积的变化转变为液压能,即输出具有一定压力与流量的液压油。在液压缸中,也是依靠其密封容积的变化,把输入的液压能转换为活塞直线往复运动的机械能。这种依靠密封容积变化来实现能量转换与传递的传动方式称为液压传动,它与主要依靠液体的动能来传递动力的“液力传动”(例如水轮机、离心泵、液力变矩器等)不同,后者在机床上用得极少。液压传动与液力传动,都是液体传动。

    (3)工作台运动时所能克服的阻力大小与油液的压力和活塞的有效工作面积有关,工作台运动的速度决定于在单位时间内通过节流阀流入液压缸中油液体积的多少。

    (4)在液压传动系统中,控制液压执行元件(液压缸或液压马达)的运动(速度、方向和驱动负载能力)是通过控制与调节油液的压力、流量及液流方向来实现的,即液流是处在液压控制的状态下进行工作的,因此液压传动与液压控制是不可分割的。然而通常所谓的液压控制系统是指具有液压动力机构的反馈控制系统。


\subsection{液压系统的组成}

    从分析上述系统可以看出,液压传动系统均由以下四个部分所组成:

    (1)动力元件(液压泵)。液压泵的作用是向液压系统提供压力油,是动力的来源。它是将原动机(电动机)输出的机械能转变为油液液压能的能量转换元件。

    (2)执行元件(液压缸或液压马达)。它的作用是在压力油的推动下,完成对外做功,驱动工作部件。它是将油液的液压能转变为机械能的能量转换元件。

    (3)控制元件。如溢流阀(压力阀)、节流阀(流量阀)及换向阀(方向阀)等,它们的作用是分别控制液压系统油液的压力、流量及液流方向,以满足执行元件对力、速度和运动方向的要求。

    (4)辅助元件。如油箱、油管、管接头、滤油器、蓄能器、压力表等,分别起储油、输油、连接过滤、储存压力能、测压等作用,是液压系统中不可缺少的重要组成部分。但从液压系统的工作原理来看,它们是起辅助作用的,故因此而得名。

    上述各类元件,将在以后章节中分别予以介绍。

\subsection{液压系统图的职能符号}

    图0-1(a)所示的液压系统,各元件的图形基本上表示了它们的结构原理,称结构式原理图。它直观性强,容易理解,发生故障时按此类图来检查和判断故障原因比较方便,但图形复杂,不便绘制。为了简化液压原理图的绘制以适应液压技术的迅速发展,我国国家标准(GB 786—76)规定了液压系统图的图形符号。这些符号只表示元件的职能、连接系统的通路,并不表示元件的具体结构和参数,是职能符号。图0-1(b)所示为该液压系统的职能符号式原理图。当无法用职能符号表示,或必须特别说明系统中某一重要元件的结构及动作原理时,也允许局部用结构式原理图表示。

    国家标准规定:图中各元件的符号均以静止状态(或零工位)表示;工作油路(包括主压油路和主回油路)以标准实线表示。泄漏油路以细实线表示,控制油路以虚线表示。

\section{液压传动的优、缺点及在机床上的应用}

    液压传动系统中的传动介质是油,油本身的物理特性(将在第一章中讲到),使液压传动与机械传动、电气传动、气压传动相比,具有以下优点:

    (1)能方便地实现无级调速,调速范围大。在液压传动中,可以在工作时进行无级调速,调速方便且调整范围大,可达100 :1\textasciitilde200 :1。

    (2)运动传递平稳、均匀。液压传动中的工作介质为液体,是无间隙传动且有吸振的能力,使液压传动工作平稳、均匀。不像机械传动装置,由于加工和装配误差总会存在传动间隙,从而会引起振动和冲击。

    (3)易于获得很大的力或力矩。液压传动的工作压力较高(可达$350\times{10^5}$Pa甚至更高),液压缸或液压马达的有效承压面积亦可取得较大,因此可获得很大的力或力矩。

    (4)单位功率的重量轻,体积小,结构紧凑,反应灵敏。在同等功率的情况下,液压泵或液压马达的重量为一般电机的10\%\textasciitilde20\%,外形尺寸为电机的15\%左右。液压马达的运动惯量不超过同等功率电机的10\%,启动中等功率的一般电动机需要1\textasciitilde2 s,而启动同功率的液压马达时间不超过0.1 s。液压传动反应灵敏,易于平稳地实现频繁的启、停、换向或变速。

    (5)易于实现自动化。液压传动的控制、调节比较简单,操纵比较方便、省力,易于实现自动化。当与电气或气压传动相配合使用时,更能实现远距离操纵和自动控制。

    (6)易于实现过载保护,工作可靠。在液压传动中,作为工作介质的油液压力很容易由压力控制元件来控制。只要设法控制油液压力在规定限度就可达到防止过载及避免事故的目的,使工作可靠。

    (7)自动润滑,元件寿命长。液压元件相对运动的表面因有液压油,能自行润滑,所以使用寿命较长。

    (8)液压元件易于实现通用化、标准化、系列化,便于设计、制造和推广使用。

    液压传动的主要缺点:

    (1)液压传动以液体作为工作介质,在相对运动的表面间无法避免泄漏,再加上液体具有微小的压缩性及油管产生弹性变形等原因,使液压传动不能实现严格的定比传动。泄漏使液压系统能量损失增加,效率降低;泄漏造成油液的浪费,污染周围环境。

    (2)温度对液压系统的工作性能影响较大。液体的黏度和温度有密切关系,当黏度因温度的变化而变化时,将直接影响液压系统的泄漏、液压损失和通过节流元件的流量等。故一般的液压系统不宜用于高温或低温的条件下。

    (3)传动效率较低。液压传动在能量转换及传递过程中存在着机械摩擦损失、压力损失和泄漏损失,传动效率往往较低。这一缺点,使液压传动在大功率系统中的使用受到限制,也不宜作远距离传动。

    (4)空气混入液压系统后引起工作不良,如发生振动、爬行、噪声等,因此,必须采取措施防止空气渗入。

    (5)为了防止泄漏以及满足某些性能上的要求,液压元件的制造精度要求高,使成本增加。
    
    (6)液压设备故障原因不易查找。液压传动的大部分故障都是由于油液不洁所造成的,因此要求工作液体清洁、无杂质。液压传动中的工作液体一般为各种矿物油,经过一段时间的使用后会变质,并可能混入铁屑、尘埃等杂物,油液在压力状况下通过液压泵及控制阀的缝隙,分








