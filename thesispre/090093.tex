
上面我们只是定性地分析一下产生径向不平衡力的原因,如进行定量分析,则需应用有关缝隙流量公式进行推导求出。

径向力不平衡问题是一个普遍存在的现象,智能设法减小,而不能完全消除。因为几何形状以及装配精度不可能达到理想状态。从上述分析可知,如阀芯出现锥状,则希望在装配时使其按顺锥形式安置,这样可减小卡紧现象。另外,应严格控制零件的制造精度,对其外圆表面,其粗糙度一般不低于$R_a0.2$阀孔粗糙度不低于$R_a0.4$,圆柱度、直线度等保持在$0.003\sim0.005 \ mm$范围内。配合间隙不宜过大,径向间隙一般在$5\sim15 \ u m$之间。

为了减小径向不平衡力,除了在加工工艺上严格要求以外,在滑阀阀芯结构上也可采取一定措施,如开环形均压槽。阀芯上开环形均压槽以后,其径向不平衡力将大大减小,如图$4-15$所示,没有开环形均压槽时,其径向不平衡力如虚线$A_1A_2$包围的面积所示,而开了环形均压槽后,其径向不平衡力如实线$B_1B_2$包围的面积所示。环形均压槽的尺寸:宽度为$0.3 \sim0.5 \ mm$,深度为$0.5\sim 1. 0\ mm$,槽间距离为$3\sim 5\ mm$。

3)滑阀上的液动力(包括稳态轴向液动力和瞬态液动力)。在第一章中对滑阀上的液动力作了分析与计算见式$(1-37a)$、式$(1-37b)$,它们对滑阀的工作性能,特别是动态性能具有很大的影响。对滑阀的操纵机械的设计也是必须认真对待。

图$4-15$滑 阀环形槽的功用图$4-16$ 换向阀换向卸荷回路

3.换向阀的应用

(1)利用换向阀换向和卸荷回路。当工作部件短时间暂停工作(如进行测量或装卸工件)时,为了节省功率,减少发热,减轻泵和电机的负荷,以延长其使用寿命,一般都让液压泵在空载状态下运转(液压泵在很低压力下工作),也就是让泵与电机进行卸荷,一般功率在$3\ k\textbf{\emph{W}}$以上的液压系统,大多设有能实现这种功能的卸荷回路。

采用$H$型(或$M$型、$K$型)滑阀机能,油路在换向阀左、右位工作时,可实现执行元件的运动变换。当换向阀处于中位时,液压泵输出油液通过换向阀中位通道直接流回油箱,泵的出口压力仅为油液流经管路与换向阀时所引起的压力损失,如图$4-16$所示。这种回路结构简单,所用元件少。但当泵从卸荷重新升压工作时,可能产生压力冲击,故不宜在高压大流量条件下使用。

图$4-17$所示是利用三位四通$O$型中位机能电磁换向阀实现油路换向。当三位阀处于中位时,二位二通电磁阀把液压泵的输出油全部接通油箱,实现液压泵的“无载”运转。这种回路要求二位二通阀的规格需和泵的容量相适应。同样,当泵从卸荷状态重新升压工作时,亦存在可能产生压力冲击的问题。
图$4-17$二位二通 阀卸荷回路图$4-18$行程阀式顺序动作回路

(2)图$4-18$所示是一种用行程阀(机动换向阀)实现顺序动作的回路。当电磁阙1通电时(图示位置),液压缸3的活塞先向右运动,并在其挡块压下行程阀2后、才使缸4的活塞右行。在阀1的电磁铁断电后、缸3的活塞先行左退并在其挡块松开行程阀2后,才使缸4的活塞也向左退回。这种回路工作可靠,但改变动作顺序比较困难。

\newpage \section* {\centerline{4-3压力控制阀}}

用以控制和调节液压系统油液压力,或以液压力作为控制信号的元件,统称为压力控越阀。按照压力控制间在液压系统中所起的具体作用又可分为溢流问、减压间顺序间和压力继电器等。

一、溢流阀

溢流阀使用在不同场合,具有不同的用途。它可用于定量采节流调速系统中作为溢流定压河:在容积调速系统中作为过载保护的安全阀:用作液压泵的低压部荷阅等。

l.结构和工作原理

(1)直动型溢流阀。直动型溢流阀是直接作用式,它的结构如图$4-19(a)$所示。直动型溢流阀由带阻尼活塞的阀芯(锥阀或球阀)$1$、阀体2、上盖3、弹簧4和调节手柄5等组成。


\centerline{图$4-19$\quad 直动型溢流阀}

$(a)$溢流阀结构图:$1$—阀芯,$2$—阀体,$3$—上盖,$4$—弹簧,$5一$调节手柄;$(b)$锥阀式结构局部放大图:$1$一偏流盘,$2$—锥阀,$3$—阻尼活塞

$P$口通压力油,$O$口接回油箱,压力油进入溢流阀后,阀芯底部进人压力油。由于阀芯顶部作用着弹簧力,因此阀芯的工作位置要由阀芯底部的油压力$p$与弹簧力两部分决定。当作用在阀芯上的液压力$pA_v$。小于弹簧力$F_s$,时,阀芯处于最低位置,$P$口与$O$口不通。当$pA_v$ 大于$F_s$时,阀芯上升,$P$口与$O$口接通,溢流阀溢流。当$\rm pA_v=F_s$,时,阀口处于某一开度,$P$ 口压力也就基本维护在这一压力数值$p=\frac{F_s}{A_v}$($A_v$为阀芯底部面积)。由于阀芯上下移动距离很小,因此,在这段距离内弹簧力F;也变化很小,可近似地视为不变,$p$也基本维持不变。这就是直动型溢流阀的工作原理。直动型溢流阀的压力调节可通过手柄$5$来进行,压力等级可调换弹簧$4$来实现,如压力级别为$2. 5 MPa,$6$ MPa,10 MPa ,20 MPa,31.5 MPa$和$40 MPa$等。

阻尼活塞的侧面铣一个小平面或加大配合间隙,以便压力油可以流到活塞底部。阻尼活塞有两个作用:在阀开启或闭合时起阻尼作用,以提高阀芯的工作稳定性;保证阀芯移动时的对中性,防止倾斜,以改善阀的静态特性。此外,在锥阀的端部设有偏流盘,偏流盘上开有一个环形槽,用以改变锥阀出油口的液流方向,产生-个与弹簧力相反的射流力。当通过溢流阀的流量增加时虽然 ,因为锥阀阀口增 大引起弹簧力增大,但由于与弹簧力方向相反的射流力同时增加,其结果抵消了弹簧力的增量,因此它改善了阀的启闭特性,提高了阀的压力和流量稳定性。偏流盘可以支撑较大的弹簧,为弹簧设计提供了方便。

直动型溢流阀通常用于小流量液压系统,益流稳压效果较好。当溢流量变化较大时,由于阀芯移动量变化大,使调压弹簧压缩量变化大,从而造成$F_s$变化较大,故压力波动较大,影响系统的工作性能。直动型溢流阀在系统中一般作安全阀使用。

(2)先导式溢流阀。直动型溢流阀用于大流量溢流时,压力波动较大。为了减小压力波动,使液压系统的压力更加稳定,则采用先导式溢流阀。图$4-20$ 所示为Y型先导式溢流阀结构,此种阀是一些液压系统中普遍使用的类型。


图$4-20\qquad Y_1$型先导式溢流阀

图$4-21$所示为一种推广型先导式溢流阀的结构。图$4-22$所示为先导式溢流阀的原理图。

先导式溢流阀由主阀芯1、主阀弹簧14、阀体15和先导阀$7$等组成。先导阀$7$相当一个直动型溢流阀。

压力油进人溢流阀直接作用在阀芯1上,同时经过阻尼孔2,3及控制管道4,5作用在主阀芯1上端面和先导阀$7$的先导锥阀6上。当系统的压力$p$低于弹簧8所调定的压力值时,锥阀6关闭,主阀芯1两端所受液压力相等,主阀芯1在弹簧14的作用下压向阀座,使$P$口与$0$口不相通。当系统压力p超过弹簧8的调定值时,先导锥阀6打开,压力油通过阻尼孔2、管道4、先导锥阀6、回油管道10流回油箱。此时由于液流通过阻尼孔的流动,造成主阀芯1两端的液压力的不平衡,这个压差超过弹簧14的作用力而使阀芯1移动,从而打开$P$和$O$的通道,实现溢流。

外控口K通过管道4和5,阻尼孔3与主阀芯1的弹簧腔相通,如在外控口K处接通控制油路,就可对溢流阀进行远程调压或卸荷。
