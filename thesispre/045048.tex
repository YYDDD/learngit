
对液压马达来讲,也是一-样,因为当液压马达的负载过大致使工作压力过大时,泄漏量增加,导致转速下降,效率降低,寿命减少,所以也有一个最大工作压力的限制值,即液压马达的公称压力。可见液压泵和液压马达的公称压力实际上是取决于它们本身结构的密封性能和规定的使用寿命。
\subsubsection{液压泵和液压马达的排量和流量}
液压泵的排量是指在没有泄漏的情况下,液压泵每转一转所排出的油液体积。在图2-2所示的液压泵中,凸轮轴每转一转,柱塞往复一次,它所排出的油液体积$q_{p}$(排量)等于柱塞截面积$A$和柱塞行程$l$的乘积,即
\begin{equation}
  q_{p}={A}{l}
\end{equation}

因此液压泵的排量仅仅取决于密封工作油腔每转变化的容积而与转速无关。

液压泵的理论流量$Q_{op}$是指在没有泄漏的情况下,单位时间内输出的油液体积,它等于排量和转速的乘积,即
\begin{equation}
  Q_{op}={q_{p}}{n_{p}}
\end{equation}

因此液压泵的理论流量只与排量和转速有关(即与密封容积变化的大小和变化的频率有关)而与压力无关。工作压力为零时,实际测得的流量可作为其理论流量。

与液压泵类似,液压马达的排量$q_{m}$是指在没有泄漏的情况下,液压马达转一转所输入的油液体积。液压马达的理论流量$Q_{om}$也是其排量和转速的乘积,即
\begin{equation}
  Q_{om}={q_{m}}{n_{m}}
\end{equation}

\subsubsection{液压泵和液压马达的功率和效率}
图2-1表示了液压泵和液压马达的能量转换图,液压泵是将原动机输入的机械能即转矩和转速(角速度)转换成液体的压力能即液体的压力和流量。液压马达则相反,它是将输入的液压能转换成机械能,若不考虑转换过程的能量损失,则输出功率等于输入功率,也就是它们的理论功率是
\begin{equation}
  P={p}{Q_{o}}={T_{o}}{\omega }
\end{equation}
\noindent 式中\
\begin{tabular}[t]{lll}
  $Q_{o}$ &——液压泵(液压马达)的理论流量;\\
  $T_{o}$ &——液压泵(液压马达)的理论转矩;\\
  $p$ &——液压泵(液压马达)的压力;\\
  $\omega$ &——液压泵(液压马达)的角速度
  \end{tabular}
实际上,液压泵和液压马达在能量转换过程中是有损失的,因此输出功率小于输入功率,两者之间的差值为功率损失。功率损失可以分为容积损失和机械损失两部分。

容积损失是因泄漏而造成流量上的损失,对液压泵来说,输出压力增大时泄漏加大,泵实际输出的流量$Q_{p}$减小。设泵的泄漏为${\Delta  }{Q_{1}}$,则
\begin{equation}
  Q_{p}={Q_{op}}-{\Delta  Q_{1p}}={{Q_{op}}-{C_{1p}}{p_{p}}}
\end{equation}
泵的容积损失可用容积效率$\eta_{V}$来表示,容积效率为液压泵的实际流量与理论流量之比,即
\begin{equation}
  \eta _{V_{p}}=\frac{Q_{p}}{Q_{op}}=\frac{{Q_{op}-\Delta Q_{1p} }}{Q_{op}}=1-\frac{{C_{1p}}{p_{p}}}{Q_{op}}
\end{equation}

机械损失是指因摩擦而造成的转矩上的损失。对液压泵来说,驱动泵的转矩总是大于其理论上所需要的转矩。设转矩损失为$\Delta  T_{p}$,则泵实际输入转矩为$T{p}={T_{op}}+\varDelta T_{p}$,机械损失可用机械效率$\eta _{jp}$,即液压泵的理论输入转矩与实际输入转矩之比来表示:
\begin{equation}
  \eta _{jp}=\frac{T_{op}}{T_{p}}=\frac{{T_{p}-{\Delta T_{p}}}}{T_{p}}=1-\frac{\Delta T_{p}}{T_{p}}
\end{equation}

由黏性摩擦和机械摩擦而产生的转矩损失的大小与油液黏性、转速以及工作压力有关。油液黏度愈大、转速愈高、工作压力愈高时,转矩损失就愈大。
液压泵的总效率是指其输出功率与输入功率之比。由前而几式可以得出
\begin{equation}
  \eta _{p}=\frac{p_{p}Q_{p}}{T_{p}\omega_{p}}=\eta_{V_{p}}\eta_{jp}
\end{equation}

即液压泵的总效率等于其容积效率和机械效率的乘积。

液压泵的输入功率$P_{p}$可表示为
\begin{equation}
  P_{p}=\frac{p_{p}Q_{p}}{\eta_{p}}
\end{equation}
若考虑常用单位,泵的输入功率的计算式为
\begin{equation}
  P_{p}=\frac{p_{p}Q_{p}}{600\eta_{p}}(kW)
\end{equation}
\noindent 式中\
\begin{tabular}[t]{lll}
  $P_{p}$ &——泵的输出压力($10^5$kW);\\
  $Q_{p}$ &——泵的实际输出流量(L/min);\\
  $\eta_{p}$ &——泵的总效率
  \end{tabular}

对于液压马达来说,输入功率为液压能,输出功率为机械能,因此其总效率为
\begin{equation}
  \eta_{m}=\frac{T_{m}\omega_{m}}{p_{m}Q_{m}}=\eta_{V_{m}}\eta_{jm}
\end{equation}
其中容积效率为液压马达的理论流量$Q_{om}$与实际输入流量$Q_{m}$之比,即
\begin{equation}
  \eta_{V_{m}}=\frac{Q_{om}}{Q_{m}}=\frac{Q_{m}-C_{1m}p_{m}}{Q_{m}}=1-\frac{C_{1m}p_{m}}{Q_{m}}
\end{equation}
机械效率为实际输出转矩$T_{m}$与理论转矩$T_{om}$之比,即
\begin{equation}
  \eta_{jm}=\frac{T_{m}}{T_{om}}=\frac{T_{om}-{\Delta T_{m}}}{T_{om}}=1-\frac{\Delta T_{m}}{T_{om}}
\end{equation}
对于液压马达,常需要根据输入的油液压力$p_{m}$和排量$q_{m}$来计算它的输出转矩$T_{m}$,由液压马达的公式可得
\begin{equation}
  T_{m}=\frac{1}{2\pi}p_{m}q_{m}\eta_{jm}
\end{equation}

\subsection{液压泵和液压马达的类型}
液压泵和马达的类型很多,常用的类型主要可分为柱塞式、叶片式和齿轮式三大类。而对每一类还可进一步细分,如柱塞式可分为轴向和径向柱塞式;叶片式可分为单作用与双作用式;齿轮式可分为外啮合式和内啮合式。根据泵或马达其排量$q$是否可以改变,又可分为定量泵、定量马达和变量泵、变量马达;调节排量的方式有手动和自动两种;而自动调节又分为限压式、恒功率式、恒压式和恒流量式等。根据转速高低和转矩大小,液压马达又可分为高速小扭矩和低速大扭矩马达等。
\begin{figure}
\centering
\ifOpenSource
\includegraphics[height=4cm]{cover.jpg}
\else
\includegraphics{fig0204.pdf}%例如 fig0506.pdf
\fi
\caption{液压泵和液压马达的图形符号}
\centering\scriptsize(a)定量泵;(b)变量泵;(c)定量马达;\\\footnotesize(d)变量马达;(e)双向变量泵;(f)双向变量马达
\label{fig:fig0204}%例如fig:fig0506
\end{figure}
液压泵和液压马达的图形符号见图2-4。




\section{齿轮泵}
\subsection{齿轮泵的构造和工作原理}
图2-5所示为普通常用的外啮合齿轮泵的工作原理。它是由装在壳体内的一对齿轮所组成的,齿轮的两个端面处用两个端盖(图中未示出)来密封。两个齿轮、壳体与端盖之间在齿轮啮合点的两侧形成两个密封的工作腔。当齿轮在电动机的带动下,按图示方向旋转时,在啮合点的右侧,啮合的齿轮逐渐脱开,使密封工作腔不断由小变大,形成局部真空,将油液从油箱经吸油口吸入,填充齿间。随着齿轮的旋转,油液被带到啮合点的左侧,由于齿轮在这里逐渐进入啮合,使密封的工作腔容积不断由大变小,油液便经压油口被挤到系统中去。

\begin{figure}
\centering
\ifOpenSource
\includegraphics[height=4cm]{cover.jpg}
\else
\includegraphics{fig0205.pdf}%例如 fig0506.pdf
\fi
\caption{齿轮泵的工作原理}
\label{fig:fig0205}%例如fig:fig0506
\end{figure}


\subsection{齿轮泵的流量}

从上述齿轮泵的工作原理可知,齿轮每转过一个齿,就会将一对齿间容积的油液挤出,所以齿轮泵的排量$q$应是其两个齿轮的齿间容积之总和。近似计算时,可假设齿间的容积等于轮齿的体积,且不计齿轮啮合时的径向间隙。当齿轮齿数为$z$,节圆直径为$D$、工作齿高为$h$、模数为$m$、齿宽为$b$时,泵的排量为
\begin{equation}
  q=\pi D h b=2\pi z m^2 b
\end{equation}

实际上齿间的容积比齿轮的体积大一点,齿数少时大得更多,为此可令数3.33来代替$\pi$值,则齿轮泵的排量应为
\begin{equation}
  q=6.66 z m^2 b
\end{equation}

当泵的转速为$n$,容积效率为$\eta_{V}$时,其实际输出流量应为
\begin{equation}
  q=6.66 z m^2 b n \eta_{V}
\end{equation}

式中的$Q$表示齿轮泵的实际平均流量。由于在齿轮不同的啮合点,密封工作腔的容积变化率不一样,因此瞬时输出的流量是变化的,这就是齿轮泵输出流量脉动的基本原因。液压泵输出流量的脉动程度,如图2-6所示,可用脉动率(或脉动系数)$\sigma$来表示,即
\begin{equation}
  \sigma=\frac{{Q_{max}-Q_{min}}}{Q}
\end{equation} 
\noindent 式中\
\begin{tabular}[t]{lll}
  $Q_{max}$ &——瞬时流量的最大值;\\
  $Q_{min}$ &——瞬时流量的最小值;\\
  $Q$ &——泵的实际平均流量
  \end{tabular}
流量脉动率$\sigma$是液压泵工作性能的重要参数之一,它直接影响系统工作的平稳性。齿轮泵的流量脉动率与齿数有关,齿数愈少,脉动率愈大。此外,外啮合齿轮泵比内啮合齿轮泵的脉动率要大。由于齿轮泵流量脉动率较大,一般为10$\%$~20$\%$故在精密机床上很少采用。

从式(2-17)可知,提高齿轮泵的转速,增大模数和齿数,可以增大流量。但转速的提高有一定限度,因为转速太高时,油液在离心力的作用下,不易填满齿间,会形成“空穴现象”,并会使容积效率降低。齿数增多时,将导致泵的体积加大。由于流量与模数的平方成正比,若想不增大泵的体积而要加大流量,则应尽量增大模数,减少齿数。所以一般齿轮泵的齿数较少(9~17齿)。齿数太少也不好,这会使流量脉动率增大。

\begin{figure}
\centering
\ifOpenSource
\includegraphics[height=4cm]{cover.jpg}
\else
\includegraphics{fig0206.pdf}%例如 fig0506.pdf
\fi
\caption{流量脉动曲线}
\label{fig:fig0206}%例如fig:fig0506
\end{figure}

\subsection{困油现象}



为保证齿轮传动平稳,供油连续,齿轮的重叠系数$\varepsilon$必须大于1,即一对轮齿即将脱开前,下一对轮齿已开始啮合,因此在某一短时间内同时有两对轮齿啮合如图2-7(a)所示,留在齿间中的油液被围困在两对轮齿间的封闭容腔内,既不与压油口连通,也不与吸油口连通。随着齿轮的旋转(由图2-7(a)转到图2-7(b)所示位置),该封闭容积由大变小。由于油液的可压缩性很小,因而压力急剧增高,油液只能从各缝隙里硬挤出去,使齿轮轴和轴承等受到很大的冲击载荷。当齿轮继续旋转(由图2-7(b))转到图2- 7(c)所示位置),该封闭容积将由小变大,造成局部真空,使油液中的空气分离出来,油液本身也会汽化,产生气泡,这就是困油现象。困油现象会使流量不均匀,形成压力脉动,产生很大的噪声,使泵的寿命降低。为了消除困油现象,可在齿轮两侧的端盖上铣两个凹下去的卸荷槽,如图2-7(d)所示。当封闭容腔缩小时,通过右边的卸荷槽与压油口连通,当封闭容腔增大时,通过左边的卸荷槽与吸油口连通,卸荷槽之间的距离$a$应保证在任何时候吸、压油口都不会串通。

\begin{figure}
\centering
\ifOpenSource
\includegraphics[height=4cm]{cover.jpg}
\else
\includegraphics{fig0207.pdf}%例如 fig0506.pdf
\fi
\caption{齿轮泵的困油现象}
\label{fig:fig0207}%例如fig:fig0506
\end{figure}


\subsection{径向液压作用力的不平衡}
由图2-5可知,齿轮啮合点的左侧是压油腔,其中压力为工作压力;右侧是吸油腔,其中压