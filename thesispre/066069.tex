
\section{思考题和习题}

\begin{figure}
\ifOpenSource
\includegraphics[width=5cm]{cover.jpg}
\else
\includegraphics{}
\fi
\caption{}
\label{}
\end{figure}
    

2-1什么是容积式液压泵?它是怎样进行工作的?这种泵的实际工作压力和输油量的大小各取决于什么?

2-2什么是液压泵和液压马达的公称压力?其大小由什么来决定?

2-3齿轮泵的困油现象、径向力不平衡是怎样引起的?对其工作有何影响?如何解决?

2-4为什么齿轮泵的齿数少而模数大?

2-5双作用叶片泵定子内表面的过渡曲线为何要做成等加速曲线?其最易磨损的地方在吸油区还是在压油区?

2-6为何国产双作用叶片泵的流量脉动很小?

2-7限压式变量叶片泵有何特点?适用于什么场合?用什么方法来调节它的流量压力特性?

2-8单作用叶片泵和双作用叶片泵的叶片倾角的方向为何相反?

2-9轴向柱塞液压马达输出轴上的转距是如何产生的?其输出转矩的大小与哪些因素有关?已知其排量为$q_m$,输入和输出的油压力分别为$p_m1$和$p_m2$,试求其理论平均转矩$T_m$

2-10试分析图2-19所示轴向柱塞式手动变量泵及图2-21所示轴向柱塞液压马达的工作原理及结构。

2-11如图2-26所示,已知液压泵的排量$q_p$=10mL/r,转速$n_p$=1000r/min;容积效率$\eta_V$随压力按线性规律变化,当压力为调定压力4Mpa时,$\eta_V$=0.6;液压缸A和B的有效面积皆为100c$m^2$;液压缸A和B需举升的物重分别为$W_A$=45000N,$W_B$=10000N,试求:

(1)液压缸A和B举物上升速度;

(2)上升和上升停止时的系统压力;

(3)上升和上升停止时液压泵的输出功率。

2-12已知轴向柱塞泵斜盘倾角$\delta =22\circ30'$,柱塞直径d=22mm,柱塞分布圆直径D=

\newpage
\noindent
68mm,柱塞数z=7,当输出压力$P_p$=10Mpa时,其容积效率$\eta_v$=0.98,机械效率$\eta_j$=0.9,转速$n_p$=960r/min。试计算:

\begin{figure} [htbp]
    \centering
    \ifOpenSource
    \includegraphics[height=5cm]{cover.jpg}
    \else
    \includegraphics{fig0226.jpg}
    \fi
    \caption{题2-11图} 
    \label{fig:fig0226}
\end{figure}


\newif\ifOpenSource
\OpenSourcetrue



(1)泵的实际输出流量(L/min);

(2)泵的输出功率(kW);

(3)泵的输入转矩(N·m)。

2-13已知液压马达的排量$q_m$=250mL/r,入口压力为$98*10^5$Pa,出口压力为$4.9*10^5$Pa,此时的总功率$\eta$=0.9;容积效率$\eta_V$=0.92,当输入的流量为22L/min时,试求:

(1)液压马达的输出转矩(N·m);

(2)液压马达的输出功率(kW);

(3)液压马达的转速(r/min)。

\newpage

\begin{center}
\section{第三章\  液压缸}


\subsection{3-1\  液压缸的基本类型和特点}
\end{center}

液压缸是液压传动系统中的执行元件,它和液压马达一样,都是将油液的压力能转换成机械能的能量转换装置。所不同的是,液压马达实现连续
的回转运动,而液压缸实现直线往复运动或摆动。由于液压缸结构简单,工作可靠,在机床及其他领域得到了广泛的应用。如用来驱动磨床、组合
机床的进给运动;刨床、拉床的主运动;送料、夹紧、定位、转位等辅助运动。不同的应用场合,对液压缸结构形式的要求也不同,因此液压缸的
类型很多。归纳起来,液压缸可以分为三大类:活塞式液压缸、柱塞式液压缸和摆动式液压缸。前二者实现直线往复运动,后者实现摆动运动。液压缸
除单个使用外,还可以几个组合起来或和其他机构组合起来,以完成特殊的功用。表3-1中列出了常用液压缸的类型和结构。

\newpage

\section{活塞式液压缸}
\subsection{双出杆液压缸}

这种液压缸其活塞两端都有活塞杆(见图3-1),它有两种不同的安装形式。图3-1(a)所示为缸体固定时的安装形式,缸体两端设有进出油口,活塞通过活塞杆带动工作台移动。
当活塞的有效行程为l时,整个工作台的运动范围为3l,所以运动部件占地面积大,一般适用于小型机床。当机床工作台行程要求长时,可采用图3-1(b)所示的活塞杆固定的形式。
这时,缸体与工作台相连,活塞杆通过支架固定在机床上,动力由缸体传出。在这种安装形式中,机床工作台的移动范围只等于液压缸有效行程的2倍(2l),因此占地面积小。进出
油口可以设置在固定不动的活塞杆的两端,使油液从空心的活塞杆中进出,也可以设置在缸体的两端,但这时必须使用软管连接。由于这种结构形式复杂,移动部分(缸体)的质量大
,惯性大,所以只用于中型和大型机床。

双出杆液压缸两端的活塞杆直径通常是相等的。因此它的左、右腔的有效面积亦相等。当分别向左、右腔
输入相同压力和相同流量的油液时,液压缸左、右两个方向的推力和速度相等。推力F和速度v的计算式如下:

\begin{equation}
F=A(P_1-P_2)=\frac{\pi(D^2-d^2)(p_1-P_2)\eta_j}{4}
\end{equation}


