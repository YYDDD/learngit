\begin{equation*}
    tan x = \frac{x}{\frac{D}{2}-\frac{D}{2}\cos \theta }
\end{equation*}
    
    $$ x = \frac{D}{2}(1-\cos \theta )\tan \delta $$

    柱塞轴向移动速度 $v$ 为
\begin{equation}
    v = \frac{dx}{dt} = \frac{dx}{d\delta} \frac{d\delta}{dt}=\frac{dx}{d\delta}\omega=\frac{D}{2}\omega\tan\delta\sin\delta
\end{equation}
    式中 $\omega $为泵传动轴的角速度(常数)。

    从式($2-28$)中可看出,柱塞的轴向移动速度随转交$\delta$而变,
因此每个柱塞的瞬时流量$q'_{z}$也随$\delta$而变,即
  $$q'_{z} =\frac{\pi }{4} d^2v=\frac{\pi }{4}d^2\frac{D}{2}\omega \tan \delta \sin \theta =\frac{\pi }{8}d2D\tan \delta \sin \theta$$
  
   泵的瞬时流量$Q'_{o}$为
\begin{equation}
    Q'_{o}=\sum_{i = 1}^{x_{0}} q'_{z} =\frac{\pi }{8}d^2D\varpi\omega \tan \delta 
    sum_{i = 1}^{x_{0}}\sin [\theta +(i-1)\frac{2\pi }{z}]
\end{equation}
   式中 \qquad$z_{0}$\ ——处于压油区的柱塞数;


\qquad \ \ $z$\ ——泵的柱塞总数。


z为偶数时
$$ z_{0}=\frac{z}{2}$$

z为奇数时
\begin{equation*}
    \begin{cases}
        z_{0}=\frac{z+1}{2}  \qquad    (\frac{\pi }{z}\geq \theta \geq 0) \\
        z_{0}=\frac{z-1}{2}  \qquad    (\frac{2\pi }{z}\geq \theta \geq \frac{\pi }{z})  
    \end{cases}
\end{equation*}

轴向柱塞泵的流量脉动系数$\sigma $与柱塞数$z$的关系如表2——1所示。由表中可以看出柱塞数z
愈大,流量脉动系数愈小,且柱塞数为奇数时的脉动系数比偶数时的脉动系数小得多,故一般采用奇数
柱塞,如z=7或9。

图2-19所示为轴向柱塞式手动变量泵的典型结构,这种泵由泵体和倾斜盘两部分组成。
泵体部分包括转子(缸体)6、配油盘7、柱塞5、传动轴8等零件。传动轴8利用轴左端的花键部
分带动缸体旋转,在缸体6的7个轴向排列的柱塞缸中,各装有柱塞5,柱塞5的球形头部铆合
在滑履4中,使滑履不会脱离柱塞球头,在球形配合面间可以相对转动。由传动轴中心弹簧通
过钢球和压盘3将滑履4紧压在倾斜盘2上,采用这种结构,可以使泵具有自吸能力。为了减
少滑履和倾斜盘之间的磨损,在柱塞的中心和滑履的中心开有直径为1mm的小孔,柱塞缸中
的压力油可经此小孔进入滑履和倾斜盘接触部分的中间油室中,使处于压油区各柱塞滑履对
倾斜盘的作用力大大减小,同时压力油进入有相对滑动的配合面,形成油膜,起着静压支承的作用。
因为倾斜盘不需要跟着转动,省去了支承倾斜盘的推力支承,使结构简单。在缸体6的
外表面,镶有钢套并由滚柱轴承支承,使倾斜盘给缸体的径向分力由该滚柱轴承来承受,从而
使传动轴和缸体不承受颠覆力矩,以保证缸体端面与配油盘均匀接触。

倾斜盘部分主要包括倾斜盘和变量机构。转动手柄1,通过丝杆移动螺母滑块,使倾斜盘
绕钢球中心摆动,改变倾斜盘斜角$\delta $的大小,实现流量的调节。

这种泵的优点是结构简单,体积小,重量轻,容积效率高,一般可达95 \%左右,公称压力为
$320×10^5\ Pa$,具有自吸能力。缺点在于滑履和倾斜盘之间的滑动表面易磨损。

\subsubsection{轴向液压马达的工作原理和结构}

轴向柱塞液压马达在机床液压系统中用得较多,它的结构和轴向柱塞泵基本相同。圈
2-20所示为轴向柱塞液压马达的工作原理,其中倾斜盘1和配油盘4是固定不动的,转子征
体2与液压马达传动轴5相连并一起转动。倾斜盘的中心线与转子缸体的轴线相交一个倾斜
角$\delta $,当压力油通过配油盘的进油窗口输入到缸体的柱塞孔时,处于高压区的各个柱塞,在压
力油的作用下,顶在倾斜盘的端面上。倾斜盘给每个柱塞的反作用力$F$是垂直于倾斜盘端面
的,该反作用力可分解为两个分力:一个为水平分力$Fx$,它和作用在柱塞上的液压推力相平
衡;另一个为垂直分力$Fy$,分别由下式求得:
$$F_{x}=\frac{\pi }{4}d^2p $$
$$F_{y}=F_{x}\tan \delta =\frac{\pi }{4}d^2p\tan \delta $$
式中  \qquad $d$——柱塞直径;

\qquad \ $p$——输入液压马达的油液压力;

\qquad\ $\delta$ —— 倾斜盘的倾斜角。

垂直分力$F_{y}$,使处于压油区的每个柱塞都对转子中心产生一个转矩,这些转矩的总和使
缸体带动液压马达输出轴作逆时针方向旋转。若使进、回油路交换,即改变输油方向,则液
马达的旋转方向亦随之而改变。

液压马达的转速$n_{m}$取决于输入液压马达的实际流量$Q_{m}$和液压马达的排量$q_{m}$,即
\begin{equation}
    n_{m}=\frac{Q_{m}}{q_{m}}\eta _{v_{m}}=\frac{Q_{m}}{\frac{\pi }{4}d^2zD\tan \theta } \eta _{v_{m}}  
\end{equation}
式中 \qquad  $\eta _{v_{m}}  $\ ——液压马达的容积效率;

\qquad \ \ \ $D$ ——柱塞在缸体上的分布圆直径;

\qquad \ \ \ \ $z$ —— 柱塞数。

改变倾斜角$\delta $的大小,就可调节液压马达的转速,倾斜角越小,液压马达的排量就越小,
当输入流量不变时,液压马达转速升高。倾斜角可调的液压马达就是轴向柱塞变量液压马达。

液压马达的实际平均输出转矩$T_{m}$可由式(2-14)求得,即
\begin{equation}
       T_{m}= \frac{1}{2\pi}q p_{m}\eta _{jm}=
     \frac{1}{8}d^2zDp_{m}\eta _{jm}\tan\delta  
\end{equation}
式中  \qquad $\eta _{jm}$\ ——液压马达的机械效率;

\qquad \ \  \ \  $p$\ ——液压马达输入的油液压力。

但实际上液压马达的输出转矩是脉动的,因为垂直分力$F_{y}$,所产生的使缸体旋转的转矩与
柱塞在高压区所处的位置有关。假设某一个柱塞在高压区所处的位置与缸体垂直中心线的夹
角为8,则该柱塞产生的转矩$T_{z}$为
$$T_{z}=Fy\frac{D}{2}\sin\theta =\frac{\pi }{2} d^2pD\tan \delta \sin \theta  $$

液压马达的理论瞬时输出总转矩应由所有处于高压区的柱塞产生的转矩所组成
\begin{equation}
    \begin{split}
        T'_{m}&=\frac{\pi}{8}d^2pD\tan\delta {\sin\theta +
        sin(\theta +\frac{\pi}{2})+\dots+\sin[\theta +(i-1)\frac{2\pi}{z}]}\\
        &=\frac{\pi}{8}d^2pD\tan\delta \sum_{i = 1}^{z_{0}}  [ \theta +( i-1)\frac{2\pi}{z}]
    \end{split}
\end{equation}
式中$z_{0}$为处在高压区的柱塞数。

由式(2-32)可知,瞬时输出的转矩$T'_{m}$随柱寒转角$\theta $而变化,其脉动情况与轴向柱塞泵的
流量脉动一样,当柱塞数较多且为单数时,输出转矩的脉动较小。同样,当输入流量不变时,输
出转速脉动较小,转动平稳。

图2-21所示为轴向柱塞液压马达的结构,它和轴向柱塞泵类似,也是由缸体7、柱塞9、配
油盘8、倾斜盘2、传动轴1等主要零件组成。为了保证缸体和配油盘相对运动表面之间的密
封性,应该使配油盘表面不受颠覆力矩,以减少磨损,为此将转子分成两段,左半段称鼓轮4,
右半段就是缸体7。鼓轮4上有可以轴向滑动的推杆10,推杆在柱塞的作用下,顶在倾斜盘
上,获得转矩,并通过键带动轴转动。缸体7是空套在传动轴上并由鼓轮上的传动销6拨动它
与轴一起转动。由于转子缸体本身不传递转矩,倾斜盘对推杆的反作用力所造成的颠覆力矩
不会作用在缸体和配油盘的配油表面上。此外,缸体7和柱塞9只受轴向力,因此使配油盘表
面以及柱塞和柱塞孔的磨损都较均匀。由于缸体与轴之间的配合面很窄,因此缸体具有自位
作用(浮动),缸体在3个弹簧5和柱塞孔底部液压力的作用下,能很好地与配油盘表面贴合,
既保证了密封性能又能自动补偿磨损。倾斜盘由推力轴承支撑,目的是为了减少推杆端部与
倾斜盘端面的磨损和提高液压马达的机械效率。因为该液压马达倾斜角是固定不变的,它的
排量不可调节,属定量液压马达,它的转速只能通过改变输入流量的大小来调节。

\subsection{液压泵的流量调节}

前面几节所介绍的泵大部分为定量泵,某些泵可作为变量泵,如单作用叶片泵和径向、辅
向柱塞泵能很容易地进行排量调节,只需用手动或自动的方式改变其偏心距或倾斜角,哪可在
泵转速不变的情况下调节流量,以适应液压系统的要求。采用变量泵调节液压系统的流量具
有节约能量的效果。近年来使用变量泵越来越广泛,品种发展也相当迅速,如有恒压变量泵、
恒流量变量泵、功率匹配变量泵、限压式变量泵等类型,在这里主要介绍一下限压式变量泵。