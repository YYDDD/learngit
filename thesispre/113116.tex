
	A和B口不同。由此可见,逻辑阀相当于一个液控单向阀。我们可以利用控制口C的的压力$P_{C}$的大小来控制锥阀的启闭以及开口的大小,把这种关系用逻辑代数去处理,可以实现逻辑阀的不同功能。特别是对于复杂的液压控制系统或是在于电气控制系统相结合的场合,运用逻辑设计方法,去简化各种控制问题,可以得到既满足动作要求,又使所用元件最少、最为合理的液压回路。
	
	图4-53为逻辑阀的应用,将四个逻辑锥阀按图示组合起来,则可以构成一个四通阀。通过控制锥阀1,2,3和4的启闭,可以得到很多种不同的工作状态。如:
	
	(1)锥阀全开,相当于四通阀的H型机能;
	
	(2)锥阀全关,相当于四通滑阀的O型机能;
	
	(3)锥阀1和3开启,2和4关闭时,B口通P口,A口通回油口O;
	
	(4)锥阀2和4开启,1和4关闭时,则A口通压力油口P,B口通回油口O;
	
	(5)锥阀2和3开启,1和4关闭,P口、A口、B口相通,相当于P型机能;
	
	(6)锥阀1和4开启,2和3关闭时,P口截止,A口、B口、O口相通,相当于Y型机能。
	
	由上可以看出,由四个锥阀单元组成的逻辑换向阀,通过先导阀控制可以解除M型以外的各种滑阀机能,它相当于多位四通阀。
	
	逻辑阀流动阻力小,流通能力大,动作速度快,密封性好,泄露少,制造容易,一阀多用,便于“三化”。对于高压、大流量的液压系统的控制具有很大潜在能力,是一种很有发展发展前途的液压元件。然而,对于小流量以及简单的控制系统,使用逻辑阀无疑是增加了液压元件数目,不尽合理。
	
	\begin{center}
		思考题和习题
	\end{center}
	
	4-1 何为换向阀的“通”和“位”?并举例说明。
	
	4-2 试说明三位四通阀的O型、M型、H型中位机能的特点和它们的使用场合。
	
	4-3	选用换向阀时要考虑那些问题?怎么考虑?
	
	4-4 滑阀阀芯的卡紧现象是怎么引起的?如何解决?
	
	4-5 电-液换向阀适用于什么场合?它的先导阀中位机能为O型行吗?为什么?
	
	4-6 直动式溢流阀为何不适用与高压大流量的溢流阀?
	
	4-7 采用先导式溢流阀为何不能减小系统的压力波动?图4-21所示的先导式溢流阀中的阻尼孔2和3各起什么作用?外控口K有什么作用?如果误把它当成漏油口而接油箱时,会出现什么问题?
	
	4-8 什么事溢流阀的启闭特性?它说明什么问题?溢流阀的动态特性指标有哪些?各说明什么问题?
	
	4-9 试举例说明溢流阀在系统中的不同作用:\ding{172}溢流恒压;\ding{173}安全限压,防止过载;\ding{174}远程调压;\ding{175}造成背压;\ding{176}使系统卸载。
	
	4-10 为什么减压阀的调压弹簧腔要接油箱?如果把这个油口堵死,将会怎样?
	
	4-11画出溢流阀、减压阀及顺序阀的职能符号图形,并比较它们在结构用途上的异同之处。
	
	4-12 有哪些阀在系统中可以当背压阀使用?性能有何差异?
	
	4-13 在图4-54中的(a),(b),(c)中,当完全关闭节流阀时,系统压力P各为多少?(各溢流阀的调定压力如图所示)
	
	4-14 夹紧油路如图4-55所示,若溢流阀调定压力Wie5MPa,减压阀的调定压力为2.5MPa,当活塞运动时(负载为零),A和B两点的压力各为多少?减压阀处于什么状态?当工件被夹紧时,A和B两点的压力you各为多少?减压阀又处于什么状态?
	
	4-15 影响节流阀流量稳定性的因素是什么?为何通常将节流口做成薄壁小口并且在小流量时尽量使用大的水力直径?
	
	4-16 试说明4-46及图4-48所示的调速阀及溢流节流阀去稳速作用的工作原理。
	
	4-17 如图4-56所示,将溢流节流阀安装在回油路上,能否起到稳定速度的作用?
	
	4-18 使用调速阀是,进、出油口能不能反接?为什么?
	
	4-19 如图4-57(a)所示,调速阀串联在进油路中,能否将其中的定差减压阀改为普通减压阀(定值减压阀)?若调速阀串联在回油路中(如图4-57(b)所示)时,用定值减压阀代替定差减压阀行不行?
	
	4-20 试分析图4-16、图4-17、图4-31所示三种卸荷方法的特点和应用场合。
	
	4-21 试分析4-5、4-39所示的两种平衡回路特点。
	
	4-22如图4-58所示,使缸1往复运动所需的负载压力为2MPa,使缸2往复运动所需的负载压力为1MPa,如不考虑管路压力损失,现利用一个单向顺序阀,要求实现两缸的运动顺序如图中箭头所示。请将油路图画出来,并确定顺序阀的调整压力应为多少。
	
	4-23 读懂图4-59所示的油路图,编写电磁铁动作顺序表,并说明其中液控单向阀的作用。
	
	4-24 图4-60所示是一种顺序动作回路,说明其顺序动作靠什么元件来实现。

