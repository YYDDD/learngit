1-8\quad 当阀门关闭时压力表的读数为$2.5\times 10^{5}\ Pa$,阀门打开时压力表的读数为$0.6\times 10^5\ Pa$
如果$d=12\ mm$,不计损失,求阀门打开时管中的流量$Q$(见图1-36)。

1-9\quad 将一平板置于油液的自由射流之内,并垂直于射流轴线,设该平板截去射流流量的
部分$Q_1$,并使射流的其余部分偏转一个角度$\theta $(见图1-37)。已知射流流速$v=30\ m/s$,总流
量$Q=30\ L/s$,$Q_1=10\ L/s$,若液体的重力和液体与平板间的摩擦可以忽略不计,油液的密度
$\rho=900\ kg/m^3$。试确定射流作用在平板上的力$F$及射流的偏转角$\theta $。(提示:不计损失且忽略高
度的影响可以证明$v=v_1=v_2$

1-10\quad 如图1-38所示,水沿垂直变径管向下流动,已知上管直径$D=0.2\ m$,流速$v=3\ m/s$,
为使上下两个压力表的读数相同,下管直径应为多大?水头损失不计。

1-11\quad 如图1-39所示,一柱体在压力$F=150\ N$作用下向下移动,将液压缸中的油通过
$\delta =0.05\ mm$的缝隙排到大气中去。假设活塞和缸筒处于同心状态,缝隙长$l=70\ mm$,柱塞直
径$d=20\ mm$,油液的动力黏度$\mu =50\times 10^{-3}\ Pa\cdot s$,试确定活塞下落$0.1\ m$所需的时间。


1-12\quad 如图1-40所示,运动黏度$\nu = 40\ mm^2/s$的油液通过$300\ m$长的光滑管道,管道
两端连接两个液面差保持不变的容器,液面差$h=30\ cm$,如果仅计管道中的沿程损失,求:

(1)当通过流量为$10\ L/s$时,液体作层流运动的管道直径;
    
(2)由(1)所得出的直径,求不发生紊流时两容器最高液面差$h_{max}$。

\chapter{液压泵和液压马达}
\section{概\ 述}
液压泵和液压马达在液压系统中都属于能量转换装置。如图2-1所示,液压泵是将电机输出的机械能
(电机轴上的转矩$T_p$和角速度$\omega_p$的乘积)转变为液压能(液压泵的输出压力$P_p$和输出流量$Q_p$的乘积),
为系统提供一定流量和压力的油液,是液压系统中的动力源。而液压马达是将系统的液压能
(液压马达的输入压力$P_m$和输入流量$Q_m$的乘积)转变为机械能(液压马达输出轴上的转矩$T_m$和角速度$\omega_m$的乘积),
使系统输出一定的转速和转矩,驱动机床工作部件运动,它是液压系统中的执行元件。液压缸和液压马达的作用一样,
也是执行元件,只是液压缸作直线运动。关于液压缸的详细内容将在下一章介绍。

\textbf{一、液压泵和液压马达的工作原理和特点}

尽管液压系统中采用的液压泵类型很多,但都是属于容积式液压泵,它的工作原理可以用图2-2所示的简单柱塞式液压泵来说明。

柱塞2在弹簧3的作用下紧压在凸轮1上,电机带动凸轮1旋转,使柱塞2在柱塞套中作往复运动。当柱塞向外伸出时,
密封油腔4的容积由小变大,形成真空,油箱中的油液在大气压力的作用下,顶开单向阀5(这时单向阀6关闭)进入油腔4,实现吸油。
当柱塞向里顶入时,密封油腔4的容积由大变小,其中的油液受到挤压而产生压力,当压力增大到能克服单向阀6中弹簧的作用力时,
油液便会顶开单向阀6(这时单向阀5封住吸油管)进入系统,实现压油。凸轮连续旋转,
柱塞就不断地进行吸油和压油。图示结构中只有一个柱塞向系统供油,所以油液输出是不连续的,只能作为润滑泵使用,
为实现连续供油,可以设置多个柱塞,使它们轮流向系统供油。

由上可知,容积式液压泵是依靠密封工作油腔的容积变化来进行工作的,因此,它必须具
有一个(或多个)密封的工作油腔。当液压泵运转时,该油腔的容积必须不断由小逐渐加大,形
成真空,油箱的油液才能被吸入。当油腔容积由大逐渐减小时,油被挤压在密封工作油腔中,压
力才能升高,压力的大小取决于油液从泵中输出时受到的阻力(如单向阀6的弹簧力)。这种泵
的输油能力(或输出流量)的大小取决于密封工作油腔的数目以及容积变化的大小和频率,故
称容积式泵。

泵在吸油时吸油腔必须与油箱相通,而与压油腔不通;在压油时压油腔与压力管道相通,
而与油箱不通,由吸油到压油或由压油到吸油的转换称为配流。图2-2中所示是分别由阀5和
阀6来实现的,阀5和6称为配流装置,配流装置是泵不可缺少的,只是不同结构类型的泵,具
有不同形式的配流装置,如叶片泵、轴向柱塞泵等的配流盘,径向柱塞泵的配流轴或配流阀等。

泵借助大气压力从比它的位置低的油箱中自行吸油的能力,叫泵的自吸能力,它用泵的中
心线到油箱液面间的吸油高度来表示。图2-2中弹簧3的作用在于使柱塞克服惯性力、摩擦力
等向外伸出,使泵具有自吸能力。如果没有这个弹簧,则柱塞不会自动伸出,就无法吸油,也就
失去了自吸能力。

从原理上来讲,液压泵与液压马达之间是可逆的,但它们在具体结构上仍有差异,图2-2
所示单柱塞泵不能作为液压马达使用。如果将压力油通入工作油腔4(输入液压能),则柱塞就
在液压力的作用下,顶向凸轮,产生转矩,而使凸轮旋转(输出机被能),输出转矩的大小取决于
输入油液的压力,凸轮轴的转速取决于输入的流量以及工作油腔容积变化的大小。

\textbf{二、液压泵和液压马达的基本性能}

1.{\kaishu 液压泵和液压马达的工作压力和公称压力}

液压泵的工作压力是指泵出口处的实际压力,由容积式泵的工作原理可知:液压泵每转一
转,总要将一定体积的油液输入系统,如果液压泵要驱动一
个如图2-3(a)所示的具有负载力$F$的液压缸时,油液在前
阻后推的情况下受到挤压,油液的压力就会逐渐升高,直到
克服各种阻力(管道阻力、摩擦力和外载力等)使液压缸运动
为止。阻力越大,则泵出口处油液的压力升得越高。如果使泵
的出口直接与油箱连通,且油管又粗又短(见图2-3(b)),这
时液压泵输油的阻力很小,则泵出口处的压力就建立不起
来。由此可见,液压泵的工作压力取决于泵的总负载。

液压马达的工作压力是指它的输入油液的实际压力,其
大小同样也是取决于液压马达的负载。

为了保证液压泵具有一定的效率和使用寿命,液压泵的工作压力有一个最大的限制值。因
为当工作压力随外加负载的增大而升高时,液压泵本身的泄漏也随着增加,所以实际输出的流
量会减少,导致效率降低。当压力超过某一定值时,实际输出流量不仅会低于公称的流量,同时
泵的使用寿命也会低于规定的值,这时的工作压力就是液压泵的公称工作压力,超过这个压
力就算是过载。从这个意义上来讲,并不是绝对不允许液压泵在一定程度上在大于其公称压
力下进行工作。此外,如果液压泵在低于其公称压力下工作,则泵的使用寿命将会增高。