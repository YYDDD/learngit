\subsection*{限压式变量泵的工作原理和结构} 
图2-22为限压式变量叶片泵的工作示意图,这种泵的流量可以根据其出口
压力的大小(油泵出口压力的大小取决于泵的负载)自动调节。如图所示, 
转子的中心$O$是固定不动的,定子的中心$O_1$可以左右移动,它在左边限压弹簧
的作用下被推向右端,使相对转子中心$O$有一个偏心量$e_x$。当转子以图示方
向旋转时,转子上半部为压油腔,下半部为吸油腔,定子在压力油的作用
下压在滑块上,滑块由一排滚针支承,以减小摩擦,增加定子的灵活性。
定子右侧装有压力反馈的柱塞小油缸,油缸与压油腔连通。设反馈柱塞油
缸的有效面积为$A_x$,泵的出口压力为$p$,则通过柱塞作用在定子上的反馈力
为$p$$A_x$。限压弹簧的预紧力$F_s$由弹簧左端的螺钉调定,当$p$$A_x$小于限压弹簧的
预紧力$F_s$时,弹簧把定子推向最右端,此时偏心距为最大值$e_{max}$($e_{max}$
的大小可
通过油缸右端的螺钉来调节),泵的流量最大。当$p$$A_x$>$F_s$时,反馈力将克服弹
簧的预紧力把定子向左推移,偏心$e_x$减小,流量也相应减小。压力愈高,
$e_x$愈小,输出流量亦愈小。当压力增大到使泵的偏心距减小到所产生的
流量只够用来补偿泄漏时,泵的输出流量为零。这时,不管负载再怎
样增大,泵的出口压力不会再升高,即泵的最大输出压力是受到限制的,
故称限压式变量泵。

变量叶片泵与单作用叶片泵相同,在压油腔叶片底部通压力油,在吸油腔
叶片底部通低压油,
使叶片的顶部和
底部受力基本上是平衡的,避免了在吸油腔定子内表面的严重磨损问题。
\subsection*{限压式变量叶片泵的静态特性}
限压式变量叶片泵的静态特性主要是指其流量和压力之间的关系,亦称流
量-压力特性。

由图2-22所示的限压式变量泵的工作原理可知:泵的理论流
量$Q_0$与泵的尺寸参数以及偏心距$e_x$的大小有关;泵的泄漏量$Q_t$与压力有关,
则泵的实际流量$Q$可用下式表示:
    \begin{equation*}
        Q=Q_0-Q_t=K_Qe_x-C_1p
    \end{equation*}   

式中
\begin{tabular}[t]{ll}
    $K_Q$&———单位偏心距所产生的理论流量,其值由泵的尺寸参数决定;\\
    $C_1$&———泵的泄露系数;\\
    $e_x$&———转子与定子之间的偏心距;
\end{tabular}\\

当柱塞油缸内的液压反馈力小于弹簧预紧力,即$p$$A_x$<$F$时,定子处在最右端
位置,这时$e_x$=$e_{max}$,故有
\begin{equation*}
    Q=K_Qe_{max}-C_1p
\end{equation*}

当柱塞油缸内的液压反馈力大于弹簧预紧力,即$p$$A_x$> $F_s$ 时,
弹簧产生附加压缩量x=$e_{max }$- $e_x$,使弹簧作用力增大
至$F_s$+$k_s$($e_{max}$-$e_x$)
,考虑支承滑块处有摩擦力,则定子在弹簧力方向上的受力平衡方程式为
\begin{equation*}
   pA_x\mp F_f=F_s+k_s(e_{max}-e_x)
\end{equation*}
式中
\begin{tabular}[t]{ll}
    $F_f$&———滑块支撑处的摩擦力(设定子内壁承受液压力的
    投影面积为$A_y$,摩擦系数为f,则有$F_f$=$p$$A_y$f);\\
    $k_s$&———限压弹簧的刚度。
\end{tabular}\\
在式(2-33)和式(2-35)中消$e_x$,整理后得

\begin{equation*}
   Q=\frac{K_Q}{k_s}(F_s+k_se_{max})-\frac{K_Q}{k_s}(A_x\mp A_yf+\frac{k_sC_1}{K_Q})p
\end{equation*}

由式(2 -34)和式(2 - 36)可画出限压式变量叶片泵的
$Q$-$p$曲线(见图2-23)。图中AB段曲线与与式(2-34)相对应。
在这区段内,由于$e_{mex}$为常数,相当于定量泵,故其理论
流量是一常数,压力只是通过泄漏量来影响实际输出流量,
BC段曲线与式(2-36)相对应,在这一区段内,泵的理论流
量随压力而改变。当压力增大时,偏心距$e_x$减小,理论流量
和实际流量迅速下降。B点所对应的压力为$p_c$,$p_c$值主要由
弹簧预紧力$F_s$决定。当$p$=$p_c$时,式(2-34)和式(2-36)中的流量相等,可得
\begin{equation*}
    p_c=\frac{F_s}{A_x\mp A_yf}
\end{equation*}

泵的最大输出压力$p_{max}$相当于其输出流量为零时的压力,令式(2-36)中
的$Q$=0,则得
\begin{equation*}
p_{max}=\frac{F_s+k_se_{max}}{A_x\mp A_yf+\frac{k_sC_1}{K_Q}}
\end{equation*}

调节弹簧的预紧力,可改变$p_c$和$p_{max}$的值,使BC段曲线左右平移。最大偏
心距$e_{max}$的值可通过反馈液压缸右端的限位螺钉来调节,并由此改变最
大输出流量的大小,这时,曲线AB上下平移,如图2-23所示。因为$p_{max}$值和
BC线段的斜率不变,所以$p_c$值要发生变化。如果更换弹簧改变$k_s$值,BC线
段的斜率相应也改变了。弹簧愈“软",即$k_s$值愈小,BC线段愈陡,$P_{max}$值
愈小,$p_{max}$和$p_c$的差距亦愈小。反之,弹簧愈“硬”,即$k_s$值愈大,BC线段
愈平缓,$p_{max}$值愈大,$p_{max}$和$p_c$的差距亦愈大。在应用时,可根据不同的
需要,通过可调环节来获得所要求的流量-压力特性。

限压式变量叶片泵的流量-压力特性正好满足既要实现快速行程又要实现
工作进给的工作部件对液压源的要求。快速行程时,负载压力低,流量
大,可以使泵的工作点落在AB线段上。工作进给时负载压力升高,流量减小,
工作点正好落在BC段。
\subsection*{限压式变量叶片泵的优缺点和应用}
限压式变量叶片泵与双作用定量叶片泵相比,结构复杂,尺寸大,相对运动的
机件多,轴上受单向径向液压力大,故泄漏大,容积效率和机械效率较低。
由于流量有脉动和困油现象的存在,因而压力脉动和噪声大,工作压力的
提高受到限制。国产限压式变量叶片泵的公称压力为63$\times $$10^5$Pa。但是这种
泵的流量可随负载的大小自动调节,故功率损失小,可节省能源,减少发
热。由于它在低压时流量大,高压时流量小,特别适合驱动快速推力小、慢
速推力大的工作机构,例如在组合机床上驱动动力滑台实现快速趋近$\rightarrow$工作
进给$\rightarrow$快速退回的半自动循环运动,以及在液压夹紧机构中实现夹紧保压
等。

为了提高工作压力和流量,目前已广泛采用了限压式柱塞变量泵。其工作
原理类同限压式变量叶片泵,也是利用反馈缸的液压力与弹簧预紧力相
互作用改变径向柱塞泵或轴向柱塞泵的定子偏心距和倾斜盘倾斜角度的方
法,来实现图2 - 23所示相似的流量-压力特性。
\begin{center}
    \section*{其他类型的泵}  
\end{center}
\subsection*{转子泵}
转子泵也称为内啮合摆线齿轮泵。在这种泵中,内外齿轮的齿形为一对共轭
的摆线(见图2-24(a)),小齿轮1称为内转子,内齿轮2称为外转子,它们的齿
数只相差1。图示转子泵中,外转子为7个齿,内转子为6个齿。内、外转子相
啮合形成若干密封工作腔。当内转子1由电机带动绕中心$O_1$旋转时,外转子2
被带着绕中心$O_2$同向旋转,用阴影表示的密封工作腔c(见图2 - 24(b))的
容积逐渐增大,通过侧面配油盘上的吸油窗口b将油液吸人。该密封工作腔
容积逐渐增大的过程可由图2- 24(b)$\sim $(h)清楚地表示出来。不难看出,在
内转子继续回转的后半周内,该工作腔的容积将逐渐减小,并通过配油盘
的压油窗口a将油压出。

转于泵的优点在于结构简单,尺寸小,重量轻:啮合重叠系数大,传动平稳;
齿轮同向旋转,滑动速度小,磨损小,寿命长;流量脉动、压力脉动以及噪
声小,油液在离心力的作用下易填入齿间,故允许高速旋转,容积效
率高。缺点是齿形复杂,加工困难,应用还不普遍,仅在低压系统中应用。
\subsection*{螺杆泵}
螺杆泵的工作原理和结构可参看图2-25. 一般它是由一根双头右旋主动螺杆4和
两根双头左旋从动螺杆5以及泵体6、泵盖1和7等零件所组成的。
泵体内的三根螺杆互相啮合,在垂直于轴线的剖面内,齿形为相互共轭的的摆线,螺杆
的啮合线把各螺杆的的螺旋槽分割成若干密封工作腔,当主动螺杆4带动两根螺杆5按图示
方向(从轴头伸出端看去为顺时针方向)旋转时,随着空间啮合曲线的移动,各密封工作腔
将沿着轴向从左向右移动。主动螺杆每转一转,各密封工作腔移动一个螺旋导程。在左端吸
油区,密封工作容积逐渐增大,完成吸油过程,随着螺杆的继续转动,充满油液的各螺旋
工作腔沿轴向移动到右端并进入压油区。这时,密封工作腔的容积逐渐减小,完成压油过程
。

螺杆泵的最大优点就是输油非常均匀,从理论上来讲是没有流量脉动的,只是由于泄漏等
原因使流量有微小的脉动,但比起其他类型的泵来讲仍小得多,因此,它宜被采用在某
些运动平稳性要求高的精密机床上。此外,螺杆泵的优点还有结构简单,紧凑;体积小,
重量轻;运动平稳,无困油现象,噪声小;由于螺杆转动惯量小,且由于液体是沿轴向移动,
螺杆的旋转不影响油液的吸入,所以允许采用高转速,容积效率高;对油的污染不敏感。
它的主要缺点是螺杆形状复杂,精度要求高,加工较困难,工作压力不易提高,多用于低压
系统。

总之,任何一种液压泵,都存在有自吸能力、配油装置泄漏、困油、径向力不平衡、
噪声、效率等问题,因此,在设计和使用液压泵时应注意到这些。表2-2列出了机床常
用液压泵的性能比较,可供选用液压泵时参考。