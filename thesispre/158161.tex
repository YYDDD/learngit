液压泵$\rightarrow$精滤油器$\rightarrow$先导阀(左边)$\rightarrow$单向阀I$_2$换向阀阀芯右端;

阀芯左端通向油箱的油路则先后出现三种接法:

(1)在图7-2所示的状态下,回油的流动路线为:

换向阀阀芯左端$\rightarrow$先导阀(左位)$\rightarrow$油箱。

回油路畅通无阻,阀芯移动速度很大,出现第一次快跳,右部制动锥很快地关小主回油路的通道,使工作台迅速制动。

(2)换向阀阀芯则快速移动一小段距离后,它的中部台肩移动到阀体中间沉割槽处,使液压缸两腔油路相通,工作台停止移动。此后换向阀在压力油作用下继续左移时,直通先导阀的油路被切断,回油流动路线改为:

换向阀阀芯左端$\rightarrow$节流阀J$_1$$\rightarrow$先导阀(左位)$\rightarrow$油箱。

这时阀芯按节流阀J$_1$调定的速度满速移动。由于阀体上沉割槽宽度大于阀芯中部台肩的宽度,液压缸两腔油路在阀芯慢速移动期间继续保持相通,使工作台的停止持续一段时间(可在0$\sim$5\ s内调整),这就是工作台在其反向前的端点停留。

(3)当阀芯慢速移动到其左部环形槽和先导阀相接的通道接通时,回油流动路线又改变成:

换向阀阀芯左端$\rightarrow$通道b$_1$$\rightarrow$换向阀左部环形槽$\rightarrow$先导阀(左位)$\rightarrow$油箱。

回油路又畅通无阻,阀芯出现第二次快跳,主油路被迅速切换,工作台迅速反向启动,最终完成了全部换向过程。

反向时,先导阀和换向阀自左向右移动的换向过程与上述相同,但这时a$_2$点接通油箱而a$_1$点接通高压油。

外圆磨床对往复运动的要求很高,不但应保证机床有尽可能高的生产率,还应保证换向过程平稳,换向精度高。为此机床上常采用行程控制制动式换向回路,图7-2所示就是采用了这种换向回路。还有一种回路比较简单,称之为时间控制制动式换向回路,如图7-3所示。

这个回路中的主油路只受换向阀控制。在节流阀J$_1$和J$_2$的开口大小调定之后,换向阀阀芯移动距离\emph{l}所需的时间(使活塞制动所经历的时间)就确定不变,因此,称这种制动方式为时间控制制动。时间制动式换向回路的主要优点是它的制动时间可以根据机床部件运动速度的快慢、惯性的大小、通过节流阀J$_1$和J$_2$开口量得到调节,以便控制换向冲击,提高工作效率;其主要缺点是换向过程中的冲出量受运动部件的速度和其他一些因素的影响,换向精度不高。所以这种换向回路主要用于工作部件运动速度较高但换向精度要求不高的场合,例如,平面磨床的液压系统。

2.砂轮架的快进、快退运动

这个运动由快动阀操纵,由快动缸来实现。在图7-2所示的状态下,快动阀右位接入系统,砂轮架快速前进到其最左端位置,快进的终点位置是靠活塞与缸盖的接触来保证的。为了防止砂轮架在快速运动终点处引起冲击和提高快进运动的重复定位精度,快动缸的两端设有缓冲装置,并设有抵住砂轮架的闸缸,用以消除丝杆和螺母间的间隙。快动阀左位接入系统时,砂轮架快速后退到其最后端位置。

3.砂轮架的周期进给运动

这个运动由紧急阀操纵,由砂轮架进给缸通过其活塞上的拨爪棘轮、齿轮、丝杆螺母等传动副来实现。砂轮架的周期进给运动可以在工件左端停留时进行,可以在工件右端停留时进行,也可以在工件两端停留时进行,也可以不进行,这些都由选择阀的位置决定。在图7-2所示的状态下,选择阀选定的是“双向进给”,进给阀在操纵油路的a$_1$和a$_2$点每次相互变换压力时,向左或向右移动一次(因为通道d与c$_1$和c$_2$各接通一次),砂轮架便做一次间隙进给。进给量大小由拨爪棘轮机构调整,进给快慢及平稳性则通过调节节流阀J$_3$和J$_4$来保证。

4.工作台液动手动的互锁

这个动作是由互锁缸来实现的。当开停阀处于图7-2所示位置时,互锁缸内通入压力油,推动活塞使齿轮z$_1$和z$_2$脱开,工作台运动时就不会带动手轮转动。当开停阀左位接入系统时,互锁缸接通油箱,活塞在弹簧作用下移动,使z$_1$和z$_2$啮合,工作台就可以通过摇动手轮来移动,以调整工件。

5.尾架顶尖的退出

这个动作是由一个脚踏式的尾架阀操纵,由尾架缸来实现。尾架顶尖只有砂轮架快速退出时才能后退以确保安全,因为这时系统中的压力油在快动阀左位接入时才能通向尾架阀处。

\subsubsection*{二、液压系统具有的特点}

(1)系统采用了活塞杆固定式双杆液压缸,保证左、右两向运动速度一致,并使机床的占地面积不大。

(2)系统采用了简单节流阀式调速回路,功率损失小,这对调速范围不需很大、负载较小且基本恒定的磨床来说是很相宜的。此外,出口节流的形式在液压缸回油腔中造成的背压力有助于工作稳定,有助于加速工作台的制动,也有助于防止系统中深入空气。

(3)系统采用了HYY21/3\ P——25T型快跳式操纵箱,结构紧凑,操纵方便,换向精度和换向平稳性都很高。此外,这种操纵箱还能使工作台高频抖动(即在很短的行程内实现快速往复运动),有利于提高切入磨削时的加工质量。

\section{液压机的液压系统}

液压机是利用液压传动技术进行压力加工的设备,可以用来完成各种锻压及加压成形加工。例如钢材的锻压,金属结构件的成型,塑料制品和橡胶制品的压制等。液压机是最早应用液压传动的机械之一,目前液压传动已成为压力加工机械的主要传动形式。在重型机械制造业、航空工程、塑料及有色金属加工工业等之中,液压机已成为重要设备。

\subsubsection*{一、工况特点及对液压系统的要求}

液压机的液压传动系统是以压力变换为主,系统压力高,流量大,功率大。因此,应特别注意提高原动机功率利用率和防止泄压时产生冲击振动,保证安全可靠。

液压机根据压制工艺要求主缸能完成快速下行$\rightarrow$减速压制$\rightarrow$保压延时$\rightarrow$泄压回程$\rightarrow$停止(任意位置)的基本工作循环(见图7-4),而且压力、速度和保压时间需能调节。顶出液压缸主要用来顶出工件,要求能实现顶出、退回、停止的动作。如薄板拉伸时,又要求有顶出液压缸上升、停止和压力回程等辅助动作。有时还需用压力缸将胚料压紧,以防止周边起皱。

液压机以主运动中主要执行机构(主缸)可能输出的最大压力(吨位)作为液压机主要规格,并已系列化。顶料缸的吨位常采用主缸吨位的20\%$\sim$50\%.液压机的顶出缸可采用主缸吨位的10\%左右。双动拉伸液压机的压力缸吨位,一般采用拉伸吨位的60\%左右。

由压力加工工艺需要来确定主缸的速度,一般在由泵直接供油的液压系统中,其工作行程速度不超过50\ mm/s,快进速度不超过300\ mm/s,快退速度与快进速度相等。

\subsubsection*{二、液压系统的工作原理}

现在介绍图7-5所示的YA32—200四柱式万能液压机的液压系统,用以概括地说明液压机的液压系统工作原理。

YA32-200四柱式万能液压机的工作循环图如图7-4所示。该液压机的液压系统由主油路、辅助油路和低压控制油路三部分组成。主油路和辅助油路能源为大流量的恒功率变量泵3,控制油路的能源是低压泵1.主缸工作压力由远程调节阀9来调整。运动速度由改变泵3的流量来调节。利用液控单向阀14(充液阀)来实现快慢速度转换。主缸的上下和保压以及顶出缸的顶出和顶退,都由相应的阀来控制。

表7-2为YA32—200四柱式万通液压机的电磁铁及阀的动作表。下面分别说明各部分液压系统的工作原理。

1.主缸的运动

(1)快速下行。在主缸快速下行的起始阶段,尚未触及工件时,主缸活塞在自重作用下迅速下行。这时泵3的流量还不足以补充主缸上腔空出的体积,因而上腔形成真空。处于液压机顶部的充液筒18在大气压作用下,打开液控单向阀14向主缸上腔加油,使之充满油液,以便主缸活塞下行到接触工件时,能立即进行加压。

进油: