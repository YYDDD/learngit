在工作循环过程中,液压泵的工作压力和流量变化较大时,液压泵的驱动功率应按各工作
阶段的功率进行计算,然后取平均值$P_{\text {av}}$,即
\begin{equation}
P_{\text{av}}=\sqrt{\frac{P^2_1t_1+P^2_2t_2+\cdots+P^2_n t_n}{t_1+t_2+\cdots+t_n}}
\end{equation}
\noindent 式中\
\begin{tabular}[t]{rl}
$t_1,t_2,\cdots,t_n$&——\ 在整个工作循环中各阶段对应的时间;\\
$P_1,P_2,\cdots ,P_n$&——\ 在整个工作循环中各阶段所需功率。\\
\end{tabular}
根据式(8- 15)算得的功率和液压泵要求的工作转速,可以从产品样本中选取标准电动
机,然后必须检查每一阶段电动机的超载量是否都在允许范围内。一般规定电动机在短时间
内可超载$25\%$,否则就按最大功率选取电动机。

由图8-5(c)可知,最大功率出现在快退阶段,其数值按式(8-12)计算\\
$$P_p=\frac{P_{p2}(Q_1+Q_2)}{\eta _p}=\frac{26.8\times 10^5\times(0.1+0.15)\times10^{-3} }{0.75}=893\text W$$
\noindent 式中\
\begin{tabular}[t]{rl}
$Q_1$&——\ 大泵流量,$Q_1=0.15\times 10^{-3}$m/s(9L/min);\\
$Q_2$&——\ 小泵流量,$Q_2=0.1\times 10^{-3}$m/s(6L/min);\\
$\eta_p$&——\ 液压泵总效率,取$\eta_p=0.75$。
\end{tabular}

根据快退阶段所需功率893W及双联叶片泵要求的转速,选用功率为$1.1\times10^3$W的标准型号电机。

\subsubsection* {元、辅件的选择}
(1)阀的选择依据。主要依据是根据该阀在系统工作的最大工作压力和通过该阀的实际
流量,其他还需考虑阀的动作方式、安装固定方式、压力损失数值、工作性能参数和工作寿命等
条件来选择标准阀类的规格。

(2)选择控制阀应注意以下几个问题:

1)应尽量选择标准定型产品,要求非标准元件尽量少,不得已时,才自行设计制造专用阀
或其他液压元件。

2)选择溢流阀时,按泵的最大流量选取,使泵的全部流量能回油箱,选择节流阀和调速阀
时,要考虑其最小稳定流量满足机床执行机构低速性能的要求。

3)一般选择控制阀的公称流量比管路系统实际通过的流量大些。必要时允许通过阀
的流量超过公称流量的20\%。
%183
4)应注意差动液压缸由于面积差形成不同回油量对控制阀的影响。

关于滤油器、蓄能器等辅助元件的选择详见有关手册的辅助元件部分。

根据液压泵的工作压力和通过阀的实际流量,选择各种液压元件和辅助元件的规格。本
例中只列出系统所用元件的名称和技术数据,型号从略(见表8-12)。

\subsubsection*{确定管道尺寸}

油箱尺寸一般可根据选定元件的连接口尺寸来确定。如需要计算,则先按通过管路的最
大流量和管内允许的流速选择油管内径,然后按工作压力确定油管的壁厚或外径。

当通过管路的油液流量Q一定时,油管内径$d$决定于管中油流的平均流速$v$,即

%184
\begin{equation}
d=\sqrt{\frac{4Q}{\pi v}}
\end{equation}
\noindent 式中\
\begin{tabular}[t]{rl}
$Q$&——\ 通过油管的最大流量;\\
$v$&——\ 管内允许流速,其值按表8- 13选取。\\
\end{tabular}

由于本系统液压缸差动连接时,油管内通油量较大,其实际流量$Q\approx 0.5\times 10^{-3} \text{m}^{3}$/s(30L/min),取允许流速$v=5$m/s,因此主压力油管$d$用式(8-16)计算,即
$$
d=\sqrt{\frac{4Q}{\pi v}}=1.13\sqrt{\frac{Q}{v}}=1.13\sqrt{\frac{0.5\times 10^{-3}}{5}}=11.3\times 10^{-3}\text{m(11.3mm)}
$$
圆整化取$d=12mm$。

油管壁厚一般不需计算,根据选用的管材和管内径查液压传动手册的有关表格得管的壁厚$\delta$ 。

选用$14\times12$mm,10号冷拔无缝钢管。

其他进油管、回油管和吸油管,按元件连接口尺寸决定油管尺寸,测压管选用$4\times 3$ mm紫
铜管或铝管。管接头选用卡套式管接头,其规格按油管通径选取。

\subsubsection*{确定油箱容积}
中压系统油箱的容积,一般取液压泵公称流量$Q_n$的$5\sim7$倍,故油箱容积
$$
V=7Q_n=7\times 15\times 10^{-3}=105\times 10^{-3}\text{m}^{3}\text{(105L)}
$$
\subsection{管路系统压力损失的验算}
由于有同类型液压系统的压力损失值可以参考,故一般不必验算压力损失值。下面以工
进时的管路压力损失为例计算如下。

已知:进油管、回油管长均为$l=1.5m$,油管内径$d=10\times 10^{-3}$m,通过流量$Q=0.077\times 10^{-3}$m/s,选用20号机械油,考虑最低工作温度为$15^{\circ}C$,$\upsilon=1.5 $cm$^2/$s。
%185
\subsubsection*{判断油流类型}

利用式(1- 19),经单位换算为
\begin{equation}
Re=\frac{vd}{\upsilon}\times 10^4=\frac{1.273\;2Q}{d\upsilon}\times 10^4
\end{equation}
\noindent 式中\
\begin{tabular}[t]{rl}
$v$&——\ 平均流速(m/s);\\
$d$&——\ 油管内径(m);\\
$\upsilon$&——\ 油的运动黏度(cm$^2/$s);\\
$Q$&——\ 通过流量(m$^3$/s)。\\
\end{tabular}

$$
Re=\frac{1.273\times 0.077 \times 10^{-3}}{12\times10^{-3}\times 1.5}\times 10^4\approx55<2000
$$
故为层流。

\subsubsection*{沿程压力损失$\sum \Delta p_1$}
\begin{equation}
\Delta p_1\approx 4.3\times10^{12} \frac{\upsilon l Q}{d^4}
\end{equation}
\noindent 式中\
\begin{tabular}[t]{rl}
$\Delta p_1$&——\ 油管的沿程压力损失$(Pa)$;\\
$\upsilon $&——\ 油的运动黏度(cm$^2$/s);\\
$Q$&——\ 通过流量(m$^3$ /s);\\
$l$&——\ 油管长度(m);\\
$d$&——\ 油管内径(mm)。\\
\end{tabular}
当系统中油流为紊流时,可利用式(1-45),但其阻力系数$\lambda$ 按紊流时的数值选取。利用式
(8-18)分别算出进、回油压力损失,然后相加即得到总的沿程压力损失。\\
在进油路上
$$
\Delta p_1= 4.3\times10^{12} \frac{\upsilon l Q}{d^4}=4.3\times 10^{12} \frac {1.5\times 1.5 \times 0.077 \times 10^{-3}}{12^4} \approx 0.4\times 10^5 \text{Pa}
$$

在回油路上,其流量$Q=0.0385\times 10^{-3}m^3/s$(差动液压缸$A_1\approx 2A_2$),压力损失
$$\Delta p_1= 4.3\times10^{12} \frac{\upsilon l Q}{d^4}=4.3\times 10^{12} \frac {1.5\times 1.5 \times 0.0385 \times 10^{-3}}{12^4} \approx 0.2\times 10^5 \text{Pa}$$

由于是差动液压缸,且$A_1\approx 2A_2$,故回油路的压力损失只有一半折合到进油腔。所以工进时总的沿程压力损失为
$$
\sum \Delta \text{P}_1=(0.4+0.5\times 0.2)\times 10^5=0.5\times 10^5\text{Pa}
$$
\subsubsection*{局部压力损失$\sum \Delta \text {P}_2$}

由于采用集成块式的液压装置,故只考虑阀类元件和集成块内油路的压力损失。油流流
经集成块时的压力损失可查标准集成块油路资料。在公称流量$Q_n$   下通过液压元件的压力损
失$\Delta p_n$可由产品样本中查到。但应注意,某些液压元件,如换向阀、顺序阀和滤油器等的实际
压力损失$\Delta p_2$与通过该元件的实际流量$Q$有关,即
\begin{equation}
\Delta p_{2} =\Delta p_n(\frac{Q}{Q_n})^2
\end{equation}

此外,对于流经节流阀、调速阀的实际流量与在系统中应保证的最小压力降基本无关。背
压阀的压力损失也与实际流量基本无关。
%%%%%%%%%%%%%%%%%%%%%%
\begin{figure}[!hbt]
\centering
\ifOpenSource

\else 
\includegraphics{fig0434.jpg}
\fi
\caption{测试用图}
\label{fig:fig0434}
\end{figure}
%%%%%%%%%%%%%%%%%%%%%素材中无图 该段仅供学习测试