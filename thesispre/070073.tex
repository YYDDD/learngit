

 
\begin{equation}
    v=\frac{4Q}{\pi(D^2-d^2)}\eta_v    
\end{equation}
\noindent 式中 
\begin{tabular}[t]{rll}
 A &——液压杠的有效面积;\\
 D,d&——活塞、活塞杆的直径;\\
 Q &——输入液压缸的流量;\\
 $p_1,p_2$ &——进油腔、回油腔压力;\\
 $\eta_v,\eta_j$ &——液压缸的容积效率,机械效率。
\end{tabular}

双出杆液压缸由于两端都有活塞杆,在工作时可以使活塞受拉力而不受压力,因此活塞杆可以做得比较细。
\begin{figure} [htbp]%保证位置
    \centering
    \ifOpenSource
    \includegraphics[height=5cm]{cover.jpg}
    \else
    \includegraphics{fig0301.jpg}
    \fi
    \caption{双出杆活塞式液压缸} 
    \label{fig:fig0301}%图片3-1
\end{figure}
\subsubsection {单出杆液压缸}

如图3-2所示,活塞只有一端带活塞杆,单出杆液压缸也有缸体固定和活塞杆固定两种形式,但它们的工作台移动范围都是最大行程的2倍。
\begin{figure} [htbp]
    \centering
    \ifOpenSource
    \includegraphics[height=5cm]{cover.jpg}
    \else
    \includegraphics{fig0302.jpg}
    \fi
    \caption{单出杆液压缸} %图片名称
    \label{fig:fig0302}%图片3-2
\end{figure}

这种液压缸由于左、右两腔的有效面积$A_1$和$A_2$(见图3-2(a),(b))不相等,因此,当进油腔和回油腔的压力分别为$p_1$和$p_2$,
输入左、右两腔的流量皆为Q时,左、右两个方向的推力和速度亦不相同。单出杆液压缸推力和速度不计机械与容积效率时的计算式如下:




\begin{align}
    F_1=p_1A_1-p_2A_2=\frac {\pi}{4}[D^2p_1-(D^2-d^2)p_2]\\
    F_2=p_1A_2-p_2A_1=\frac {\pi}{4}[(D^2-d^2)p_1-D^2p_2]        
\end{align}
\newpage
    
\begin{align}
&v_1=\frac{4Q}{\pi D^2}\\
v_2=&\frac{4Q}{\pi (D^2-d^2)}
\end{align}

\noindent 式中 
\begin{tabular}[t]{ll}
 $F_1,F_2$ &——压力油分别进入无杆腔、有杆腔时的活塞能力;\\
 $A_1,A_2$ &——无杆腔,有杆腔的有效面积;;\\
 $v_1,v_2$ &——压力油分别输入无杆腔、有杆腔时活塞的运动速度。\\
\end{tabular}\\

$v_2$与$v_1$之比称为速度比$\lambda _v$,即

\begin{align}
    \lambda _v=\frac{v_2}{v_1}=\frac{1}{1-(\frac{d}{D})^2}
\end{align}
上式说明:活塞杆直径越小,$\lambda _v$越接近1,活塞两个方向运动的速度差值也就越小。如果活塞杆较粗,活塞两个方向运动的速度差值较大,
这时可以用较小流量的液压泵获得快速退回运动。
$\displaystyle\tfrac{d}{D}$,$\lambda_v$和$\tfrac{A_2}{A_1}$之间的关系见表3-2。 \\

如果向单出杆液压缸的左、右两腔同时通压力油(见图3-2(c)),即所谓的差动连接,作差动连接的单出杆液压缸称为差动液压缸。开始时差动缸左、右两腔的油液压力相同,但是由于无杆腔的有效面积大于
有杆腔的有效面积,故活塞将向右运动,同时使有杆腔中排出的油液(流量为$Q^1$)也进入无杆腔,加大了流入无杆腔的流量(Q+$Q^1$),从而加快了活塞移动的速度,实际上当活塞运动时,由于差动缸两腔间
的管路中有压力损失,所以有杆腔中的油压力稍大于无杆腔的油压力,当该压力损失很小可忽略不计时,差动缸活塞推力$F_3$和运动速度$V_3$的计算式如下:
  

\begin{align}
    F_3=p_1(&A_1-A_2)=\frac{\pi}{4}[D^2-(D^2-d^2)]p_1=\frac{\pi}{4}d^2p_1
\end{align}
\begin{align*}
    &v_3=\displaystyle\frac{Q+Q^,}{\displaystyle\frac{\pi D^2}{4}}=\displaystyle\frac{Q+\displaystyle\frac{\pi}{4}(D^2-d^2)v_3}{\displaystyle\frac{\pi D^2}{4}}
\end{align*}
整理后得
\begin{align}
    v_3=\frac{4Q}{\pi d^2}
\end{align}

由式(3-8)和式(3-9)可知,差动连接时液压缸的推力比非差动连接时小,速度比非差动· 连接时大,正好利用这一点,可使在不加大油源流量的情况下得到机床工作台快速进、退和慢速进给的运动循环。例如,在组合机床中,液压驱动的动力滑台就是利用差动缸来完成快速趋
\newpage
\noindent
近(见图3-2(c))——慢速工作(见图3-2(a))——快速退回(见图3-2(b))的。\\
\indent
如果要使快进和快退的速度相等,即使$v_3$=$v_2$,则有

\begin{equation*}
\frac{4Q}{\pi(D^2-d^2)}=\frac{4Q}{\pi d^2}
\end{equation*}
\indent
这时液压缸面积$A_1$和活塞杆截面积$A_3$存在如下关系:
\begin{equation*}
    A_1=2A_3
\end{equation*}
或
\begin{equation}
    D=\sqrt{2}d    
\end{equation}
\section{柱塞式液压缸}
上述活塞式液压缸中,缸的内孔与活塞有配合要求,所以要有较高的精度,
当缸体较长时,加工就很困难,为了解决这个矛盾,可采用柱塞式液压缸,如图3-3所示。
\begin{figure} [htbp]
    \centering
    \ifOpenSource
    \includegraphics[height=5cm]{cover.jpg}
    \else
    \includegraphics{fig0303.jpg}
    \fi
    \caption{柱塞式液压缸} %图片名称
    \label{fig:fig0303}%图片3-3
\end{figure}

从图3-3看出,柱塞缸的内壁与柱塞并不接触,没有配合要求,故缸孔不需要精加工,柱塞仅与缸盖导向孔间有配合要求,这就大大简化了缸体加工和装配的工艺性。因此,柱塞缸特别适用于行程很长的场合。
为了减轻柱塞的重量,减少柱塞的弯曲变形,柱塞一般被做成空心的。行程特别长的柱塞缸,还可以在缸筒内设置辅助支撑,以增强刚性。图3-3(a)所示为单柱塞缸,柱塞和工作台连在一起,缸体固定不动,
当压力油进入缸内时,柱塞在液压力作用下带动工作台向右移动。柱塞的返回要靠外力(如弹簧力或立式部件的重力等)来实现,图3-3(b)所示为双柱塞缸,它是由两个单柱塞缸组合而成的,因而可以实现两个方向的液压驱动。

柱塞液压缸的推力F和运动速度v的计算式如下:
\begin{align}
    F&=\frac{\pi}{4}d^2p\\
    v&=\frac{4Q}{\pi d^2}
\end{align}
式中
\begin{tabular}[t]{rll}
    d&——柱塞直径;\\
    p&——缸内油液压力;\\
    Q&——输入液压缸的流量。\\
\end{tabular}

\section{摆动液压缸}
摆动液压缸主要用来驱动做间歇回转运动的工作机构,例如回转夹具、分度机构、送料、夹
\newpage
\noindent
紧等机床辅助装置,也有用在需要周期性进给的系统中。

图3-4(a)所示为单叶片摆动液压缸,叶片1固定在轴上,隔板2固定在缸体上,隔板2的槽中嵌有密封块4,密封块4在弹簧片3的作用下紧压在轴的表面上,
起密封作用。当压力油进入摆动缸时,在油压作用下,叶片带动轴回转,摆动角度小于$300^\circ$,单叶片摆动缸结构较简单,摆动角度大。但它有两个缺点:一是输出的转矩小;二是心轴受单向径向液压力大。
图3-4(b)所示为双叶片摆动缸,心轴上固定着两个叶片,因此在同样大小的结构尺寸下,所产生的转矩比单叶片摆动缸增大1倍,而且径向液压力得到平衡,但双叶片摆动缸的转角较小(小于$150^\circ$),且在相同流量下,转速也减小了。
\begin{figure} [htbp]
    \centering
    \ifOpenSource
    \includegraphics[height=5cm]{cover.jpg}
    \else
    \includegraphics{fig0304.jpg}
    \fi
    \caption{摆动液压缸} %章节编号
    \label{fig:fig0304}%文件名去掉扩展名
\end{figure}

叶片摆动缸的转矩T和角速度$\omega$ 的计算式如下:
\begin{align}
    T=zb&\int_{R_2}^{R_1}(p_1-p_2)rdr=\frac{1}{2}bz(R_2^2-R^2_1)(p_1-P_2)\\
    &\omega =\frac{2\pi Q}{\dfrac{\pi}{4}(D^2-d^2)bz}=\frac{8Q}{bz(D^2-d^2)}
\end{align}
式中
\begin{tabular}[t]{rll}
    \centering
    b&———叶片宽度;\\
    $R_1$,$R_2$ &———叶片底部、顶部的回转半径;\\
    r&———叶片径向长度;\\
    z&———叶片数;\\
    $p_1$,$p_2$ &———工作腔、回油腔的液压力;\\
    D,d&———缸体内径、转轴外径,D=2$R_2$,d=2$R_1$;\\
    Q&———进入摆动缸的流量。
\end{tabular}

应指出的是,在计算以上液压缸的推力和转矩时,未考虑在液压缸的密封装置上产生的摩擦力,
因此,在计算液压缸的有效推力和转矩时,应乘以液压缸的机械效率$\eta _j$,,一般$\eta _j$=0.9$\sim $0.95。

\begin{center}    
\subsection{液压缸的构造}
\end{center}

图3-5所示为单出杆活塞式液压缸的典型结构,它由缸体组件和活塞组件这两个基本部分组成。缸体组件包括缸体5与前、
后端盖1和8等。活塞组件包括活塞3、活塞杆4等零件,



