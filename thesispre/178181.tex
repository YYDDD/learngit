
\subsection{液压系统的组合}
在所选择基本回路的基础上,再综合考虑其他因素的影响和要求,便可组成完整的系统图。在图8-7中为了使液压缸(滑台)快进时实现差动连接,而在工作进给时使进油路与回油隔离,在系统中增设一个单向阀11及液控顺序阀8;在液压泵1和电磁换向阀3的出口处,分别增设单向阀9和12,以免当液压系统较长时间不工作时,在“油柱”的压力下油液流回油箱,形成局部真空,由于系统不可能绝对密封,使空气渗入系统,影响系统工作平稳性。单向阀9的另一个作用是防止液压系统在电机停转时反转。为了过载保护或行程终了利用压力控制来实现切换油路,在系统中还装有压力继电器13。为观察和调整系统压力,应在图8-7所示四处设置测压点,为减少压力表,设置一个多点压力表开关14。

初步拟定出液压系统图后,应检查其动作循环,并制定出系统工作循环表,见表8-9。
\subsection{选择液压元件的配置形式}
在确定液压系统图之后,应进一步确定液压元件的配置形式。目前主要是采用集成式配置,详细内容见有关设计手册。
\section{选择液压元件}
\subsection{选择液压泵和电机}
(1)确定液压泵的工作压力。液压泵的最大工作压力与执行元件的工作性质有关。若执行元件在工作行程终点运动停止时才需要最大压力,如液压机的压制、成形、校准,机床的定位夹紧等,液压泵的最大工作压力等于执行元件的最大工作压力。

对于执行元件运动过程中需要最大压力,如铣床和组合机床等。液压缸的工作压力为
\begin{equation}
p_{\rm p}=p_1+\sum\Delta p 
\end{equation}
式中$\quad p_1$——执行元件在稳定工况下的最高工作压力:

$\sum\Delta p$——进油路沿程的局部损失。初算时按经验数据选取,如管路简单的节流调速系统取$\sum\Delta p=(2\sim5)\times10^5{\rm Pa}$:管路复杂,进油路采用调速阀系统,取$\sum\Delta p=(5\sim15) \times 10^5{\rm Pa}$。亦可参考同类系统选取。

由图8-5和表8-7可知,液压缸在整个工作循环中的最大工作压力为$43.4\times10^5{\rm Pa}$。本系统采用调速阀进油节流调速,选取进油管路压力损失为$8\times10^5{\rm Pa}$,由于采用压力继电器,溢流阀的调整压力一般应比系统最高压力大$5\times10^5{\rm Pa}$,故泵的最高工作压力为
$$p_{\rm p1}=(43.3+8+5)\times10^5=56.4\times10^5{\rm Pa}$$

这是小流量泵的最高工作压力(稳态),即溢流阀的调整工作压力。

前面计算的液压泵压力$p_{\rm p}$是系统的稳态压力。系统工作时还存在有动态超调压力,其值总是超过稳态压力。所以选择液压泵规格时,其公称压力应比计算的最大压力高25$\%$ $\sim$ 60$\%$,液压泵的公称工作压力$p_n$为
$$ p_n=1.25p_{\rm p1}=1.25\times56.4\times10^5\approx70\times10^5{\rm Pa} $$

大流量泵只在快速时向液压缸输油,由图8-5(b)可知,液压缸快退时的工作压力比快进时大,这时压力油不通过调速阀,进油路较简单,但流经管道和阀的油流量较大,去进油路压力损失为$5\times10^5{\rm Pa}$,故快退时,泵的最高压力为
$$ p_{\rm p2}=(21.8+5)\times10^5=25.8\times10^5{\rm Pa} $$

这是大流量泵的最高工作压力,此值是液控顺序阀7和8(见图8-7)调整时的参考数据。

(2)液压泵的流量。单液压泵供给多个执行元件同时工作时,泵的流量要大于液压执行元件所需最大流量的总和,并考虑系统泄露和液压泵磨损后容积效率下降等因素,即
\begin{equation}
    Q_{\rm p}\geqslant K(\sum Q)_{\rm max}
\end{equation}
式中$\quad$K——考虑系统泄露的修正系数,一\\般取$1.1\sim1.3$,大流量取小值,小流量取大值;\\$(\sum Q)_{\rm max}$——多个执行元件同时工作时系\\统所需最大流量。对动作复杂的系统,将同时\\工作的执行元件的流量循环图组合在一起(见\\图8-8),从中求$(\sum Q)_{\rm max}$,图中$\Delta Q$为系统总\\泄漏量。

对于工作过程中采用节流调节的系统,确定液压泵的流量时,还需要加溢流阀稳定工作所需的最小溢流量$Q_{\rm min}$,即
\begin{equation}
    Q_{\rm p}\geqslant K(\sum Q)_{\rm max}+Q_{\rm min}
\end{equation}

采用差动连接液压缸时,液压泵流量为
\begin{equation}
    Q_{\rm p}\geqslant K(A_1 -A_2)v_{\rm max}
\end{equation}
式中 $A_1,A_2$——分别为液压缸无杆腔和有杆腔的有效工作面积;

    $v_{\rm max}$——活塞或液压缸的最大移动速度。
  
当系统采用蓄能器储存压力油时,液压泵的流量按系统在一个周期中的平均流量选择
\begin{equation}
    Q_{\rm p}\geqslant K\sum_{i = 1}^{n} \frac{V_i}{T}
\end{equation}
式中$\quad$T——主机工作周期;

$\quad V_i$——各执行元件在工作周期内总的耗油量;
       
$\quad$n——执行元件的个数。

泵的公称流量与系统设计的$Q_{\rm p}$相当。

由图8-5(a)可知,最大流量在快进时,其值为$0.2\times10^{-3}{\rm m^3/s(12L/min)}$。按式(8-8)计算液压泵的最大流量,取$K=1.15$,得
$$Q_p=1.15\times0.2\times10^{-3}=0.23\times10^{-3}{\rm m^{3}/s(13.8L/min)}$$

最小流量在工进时,其值为$0.077\times10^{-3}{\rm m^{3}/s(4.62L/min)}$,为保证工进时系统压力较稳定,应考虑溢流阀有一定的最小溢流量,取最小溢流量为$0.017\times10^{-3}{\rm m^{3}/s}$(约$1{\rm L/min}$),故小流量泵应取$0.094\times10^{-3}{\rm m^{3}/s}$(约$5.62{\rm L/min}$)。

根据以上计算数值,选用公称流量分别为$0.15\times10^{-3}{\rm m^3/s},0.1\times10^{-3}{\rm m^{3}/s}$,公称压力为$70\times10^5{\rm Pa}$的双联叶片泵。

(3)选择电动机。在工作循环中,当泵的压力和功率比较恒定时,驱动泵的电机功率$P_{\rm p}$为
\begin{equation}
    P_{\rm p}=\frac{p_{\rm p}Q_{\rm p}}{\eta _{\rm p}}
\end{equation}
式中$\quad p_{\rm p}$——液压泵的最高工作压力;

$\quad Q_{\rm p}$——液压泵的流量;

$\quad \eta _{\rm p}$——液压泵的总效率。

各种泵在公称压力下的总效率可参考表8-10,液压泵规格大\\时取大值,小时取小值。

应该指出,当液压泵的工作压力只有公称压力的$10\%\sim 15\%$\\时,泵的总效率将显著下降,有时只达0.5或更低。此外,当变量泵\\的流量为公称流量的1/4或1/3以下时,容积效率和总效率都要下\\降很多,因此,设计时必须注意。

限压式变量叶片泵的驱动效率,可按流量特性曲线拐点处的流\\量、压力值计算,如图8-9所示。一般拐点流量的压力在泵最大压力$80\%$处,即
\begin{equation}
    P_{\rm p}=\frac{p_{\rm p}Q_{\rm p}}{\eta_{\rm p}}=\frac{0.8p_{\rm max}Q_{\rm pn}}{\eta_{\rm p}}
\end{equation}
式中$Q_{\rm pn}$为泵的公称流量。

通常,限压式变量泵在工作时,当流量很小时,效率很低。可按下式粗略估算驱动功率:
\begin{equation}
    P_{\rm p}=p_{\rm p}Q_{\rm p}+\Delta P
\end{equation}
式中$\quad p_{\rm p,Q_{\rm p}}$——泵的实际工作压力和流量;

$\quad \Delta P$——一般机床常用的限压式变量泵在压力$p_{\rm p}$下的功率损耗,可按表8-11选取。
