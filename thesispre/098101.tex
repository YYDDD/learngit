
\mainmatter
图4-29所示为一种采用两个溢流阀的多级调压回路。图中3为远程调压阀,接溢流阀2的外控口即图4-20、图4-21中的K口。图示位置表明,当二位二通电磁阀4关闭时,泵的出口压力由溢流阀2调定为$p_{1}$。在二位二通电磁阀通电切换后,如远程调压阀3的调整压力$p_{2}$低于溢流阀2的调整压力$p_{1}$,则泵的出口压力由远程调压阀调定为$p_{2}$;如远程调压阀的调整压力$p_{2}$大于$p_{1}$,则远程调压阀就不起作用。

如果把二位二通电磁阀4放置在溢流阀2与远程调压阀3之间,则当压力切换时,可能产生较大的压力波动与冲击。

图4-30所示为两个溢流阀串联连接的二级调压回路,它可以供给两条油路以两种不同压力。泵出口压力$p_{1}$由两个调压阀2和3调定。油路b的工作压力$p_{2}$由溢流阀3调定,通常用于润滑、控制等需要较低压力和较小流量的支路。

图4-31为先导溢流阀式卸荷回路,图中二位二通电磁阀安装在先导式溢流阀的外控油路上,卸荷时(电磁阀通电),泵输出流量通过溢流阀的溢流口流回油箱,而通过电磁阀的流量很小,只是溢流阀控制腔的流量,故只须选用小规格的电磁阀。卸荷时,溢流阀处于全开状态。当停止卸荷系统重新工作时,不会产生压力冲击现象,故宜用于高压大流量系统中。但电磁阀连接溢流阀的外控口后,使溢流阀的控制容积增大,工作时易产生不稳定现象,故须在该两阀间的连接油路上必要时设置阻尼装置。

\section{减压阀}

减压阀在液压系统中起减压作用,使液压系统中某一部分得到一个降低了的稳定压力。

\subsection{结构和工作原理}

图4-32所示为一些液压系统广为使用的J型减压阀的结构。

图4-33为一种推广型先导式定值减压阀的结构,它的主要组成与溢流阀相同,外形亦相似。

进入减压阀的压力油的压力$p_{1}$经阀口降低为$p_{2}$,从减压阀出口流出。同时$p_{2}$还通过阻尼孔2、管道4进入先导阀7的阀座底部并与主阀弹簧腔相通。压力油$p_{2}$作用在主阀芯1两端并作用在锥阀6上,当出口压力$p_{2}$小于先导阀的调整压力时,锥阀6关闭,阻尼孔2无油流通,主阀芯1两端液压力相等,而主阀芯在弹簧14的作用下,阀口全部打开,使油液在压降较小的情况下流出,这时减压阀没有工作。当出口压力$p_{2}$大于先导阀的调整压力时,锥阀6打开,油液经阻尼孔2、管道4、先导阀弹簧腔9、管道10流回油箱。由于阻尼孔2的作用,主阀芯1弹簧腔的压力低于$p_{2}$,造成阀芯1两端的压力不平衡,使阀芯移动,进而使阀口减小,使压力油流过阀口时压降加大,出口压力$p_{2}$减至某调定值。出口处保持调定压力时,阀芯1处于某一平衡位置上,此时阀口保持一定的开口度,减压阀处于工作状态。如果由于某种原因使进口压力$p_{1}$发生变化当阀口还没有来得及变化时,$p_{2}$则相应发生变化,造成阀芯1两端的受力状况发生变化,破坏了原来的平衡状态,使阀芯到达另一平衡状态,以保持$p_{2}$的稳定。阻尼孔3起稳定阀芯1的作用。

图4-34所示为减压阀工作原理图,减压阀稳定工作时阀芯受力平衡方程式可列写如下:
\begin{equation}
p_{2}A_\text{g}=p^{'}_{2}A_\text{g}+F_\text{s}+G+F_\text{f}+F_\text{w};
\end{equation}

\noindent 式中\ 
\begin{tabular}[t]{ll}
$p_{2}$ &—— \ 减压阀出口压力;\\
$p^{'}_{2}$ &—— \ 流经阻尼孔2后的油液压力,由先导阀调定;\\
$A_\text{g}$ &—— \ 主阀芯端面积;\\
$F_\text{s}$ &—— \ 主阀芯上弹簧力;\\
$G$ &—— \ 主阀芯自重;\\
$F_\text{f}$ &—— \ 主阀芯与阀体之间的摩擦力;\\
$F_\text{w}$ &—— \ 稳态轴向液动力。
\end{tabular}  

如果忽略阀芯自重、摩擦力及液动力的影响,则上式可写成
\begin{equation}
\begin{split}
p_{2}A_\text{g}& =p^{'}_{2}A_\text{g}+F_\text{s}\\
p_{2}& =p^{'}_{2}+\frac{F_\text{s}}{A_\text{g}}
\end{split}
\end{equation}

$p^{'}_2$由先导阀调定,基本不变,而$F_\text{s}$因弹簧刚度较小,在位移过程$F_\text{s}$变化也很小,所以使减压阀出口压力$p_{2}$基本保持一个稳定的压力值。

减压阀与溢流阀的主要不同是:

(1)主阀芯结构不同;

(2)减压阀压力油$p_{1}$进人并经阀芯开口(使压力降低)变为$p_{2}$从出口流出,同时与阀芯弹簧平衡,进油口与出油口之间是常开的;

(3)先导阀弹簀腔的油液单独接油箱,与进出孔道不连通。

\subsection{减压阀在系统中的应用}

在液压系统中,一个油源供应多个支路工作时,由于各支路要求的压力值大小不同,这就需要减压阀去调节,利用减压阀可以组成不同压力级别的液压回路,如夹紧油路、控制油路和润滑油路等。

图4-35所示为减压阀应用在夹紧油路时的减压回路,液压泵1排出的油液,其最大工作压力由溢流阀2根据主系统的负载要求加以调节。当液压缸5这一支路需要比液压泵供油压力低的油液时,在支路上设置一减压阀3,就可得到比溢流阀2调定压力低的压力。但当溢流阀的调节压力低于减压阀的调节压力时,减压阀不起作用。

\section{顺序阀}

顺序阀是用压力作为控制信号以实现油路的通断。按调压方式的不同,可分为直控式顺序阀与液控式顺序阀两种。

顺序阀的结构与工作原理和溢流阀相似,现以图4-36为例说明其结构和工作原理。主阀和先导阀均为滑阀式,其外形与溢流阀相似。

压力油进人顺序阀作用在主阀一端,同时压力油一路经管道4进入先导阀7左端,作用在滑阀6的左端面上,一路经阻尼孔2进入主阀芯1上端,并进人先导阀的中间环形部分。当进油压力低于先导阀的调整压力时,主阀芯1关闭,顺序阀无油流出。一旦进油压力超过先导阀的调整压力时,进入先导阀左端的压力将滑阀6推向右边,此时先导阀7的中间环形部分与顺序阀出口沟通,压力油经阻尼孔2、主阀芯1上腔、先导阀7流向出口。由于有液阻,主阀芯1上腔压力低于进口压力,主阀芯移动,使顺序阀进出口沟通。从上分析可知,主阀芯1的移动是主阀芯

