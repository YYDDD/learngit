
$$ R_{e}=\frac{vd}{\nu } $$

在工程上常用一个临界雷诺数$R_{e_cr}$来判别流动状态是层流还是素流。
当$R_{e}<R_{e_cr}$ 层流;当$R_{e} > R_{e_cr}$时为紊流。
表1-3所示为常见液流管道的临界雷诺数$R_{e_cr}$.


对于非圆截面的管道来说,有
$$R_{e}=\frac{4vR}{\nu }$$

式中,R为通流截面的水力半径,它等于液流的有效面积A和它的湿周
(有效截面的周长)x之
比,即
$$R=\frac{A}{x}$$

例如,正方形每边长为b,则湿周为4b,面积为$b^2$,则水力半径
$$R=\frac{b^2}{4b}=\frac{b}{4}$$

通流截面相同的管道,其水力半径与管道形状有关。圆形管道水力半径最大,同心圆环截 面的水力半径最小。水力半径大小对管道通流能力影响很大,水力半径大,表明液流与管道壁 接触少,通流能力大;水力半径小,表明液流与管壁接触多,通流能力小,容易堵塞。

一般液压传动系统所用液体为矿物油,黏度较大,且管中流速不大,因而多属层流,只有当 液流流经阀口或弯头等处时才会形成紊流。

\subsection{连续性方程}

连续性方程是质量守恒定律在流体力学中的表达形式。假设液体是不可压缩的,
而且是作恒定流动,则液体的流动过程遵守质量守恒定律,
即在单位时间内流体流过通道任意截面的 液体质量相等。

如图1-11所示,液体在管内流动,任取两通流截面$A_1$和$A_2$,
在管内取一微小流束,面积 分别为$dA_1$和$dA_2$,流速为$u_1$和$u_2$,
因为是恒定流动,故流束形状不随时间变化,即液体不会穿 
过流束的侧面流入或流出;又因液体不可压缩,
所以$\rho_1=\rho_2=\rho$。根据质量守恒定律,在出时间 内流过两个微小
通流截面的液体质量相等,即
\[\rho {{\rm{u}}_1}d{A_1}dt = \rho {u_2}d{A_2}dt\]
化简为
\[{u_1}d{A_1} = {u_2}d{A_2}\]
对整个流管,则有
\[\int_{{{\rm{A}}_1}} {{u_1}d{A_1}}  = \int_{{A_2}} {{u_2}d{A_2}} \]
以通流截面A1和A2的平均速度v1和v2来表示,则有
$${A_1}{v_1} = {A_2}{v_2} =\text{常数}$$ 
即
$${Q_1} = {Q_2} = Q =\text{常数}$$ 
或
\[\frac{{{v_1}}}{{{v_2}}} = \frac{{{A_2}}}{{{A_1}}}\]
式(1-21)和式(1-22)称为流量连续方程。它表明在不可压缩的恒定流动的液流中,通 过各通流截面的流量相等,或通流截面面积与平均流速成反比。

例1-1某液压系统,两液压缸串联,缸1的活塞是主运动,
缸2的活塞对外克服负载(从 动运动),如图1-12所示。
已知小活塞的面积$A_1=14 cm^2 $。
大活塞的面积$A_2=40 cm^2 $连接 
两液压缸管路的流量$Q = 25 L/min$,
试求两液压缸运动速度及速比。
解	由式(1-21)和式(1-22)求得小活塞
运动速度
\[{v_1} = \frac{Q}{{{A_1}}} = \frac{{25 \times 1000}}{{14 \times 60}} \approx 30cm/s\]
流进大缸的流量仍为
\[{v_2} = \frac{Q}{{{A_2}}} = \frac{{25 \times 1000}}{{40 \times 60}} \approx 10cm/s\]
两活塞速比
\[i = \frac{{{v_1}}}{{{v_2}}} = \frac{{{A_2}}}{{{A_1}}} = \frac{{40}}{{14}} = 2.86\]
\subsection{18页}
四、伯努利方程——流动液体的能量守恒定律

伯努利方程式是能量守恒定律在流动液体中的表现形式。要说明流动液体的能量问题, 必须先研究液体的受力平衡方程,亦即它的运动微分方程。由于实际流体比较复杂,在讨论时 先从理想流体着手,然后再扩展到实际流体中去。

1.	理想流体的运动微分方程
在某一瞬时,,取微小流束中一微元体(见图1 -13),用dA和ds表示它的通流截面和长 度,在一维流动的情况下,分析这微元体的受力情况:质量力为重力,
其大小为pgdAds,方向垂 直向下,与微元体轴线夹角为们微元体所受压力(表面力)为
\[pdA - \left( {p + \frac{{\partial p}}{{\partial s}}ds} \right)dA = -\frac{{\partial p}}{{\partial s}}dsdA\]

\subsection{第19页}
这一微元体积的惯性力为
\[\begin{array}{l}
  ma = \rho dAds\frac{{du}}{{dt}} = \rho dAds\left( {\frac{{\partial u}}{{\partial s}}\frac{{ds}}{{dt}} + \frac{{\partial u}}{{\partial t}}} \right)\\
   = \rho dAds\left( {u\frac{{\partial u}}{{ds}} + \frac{{\partial u}}{{\partial t}}} \right)
  \end{array}\]
由牛顿第二定律知
\[ - \frac{{\partial p}}{{\partial s}}dsdA - \rho gdAds\cos \theta  = \rho dAds\left( {u\frac{{\partial u}}{{\partial s}} + \frac{{\partial u}}{{\partial t}}} \right)\]
化简上式得
\[ - g\cos \theta  - \frac{1}{\rho }\frac{{\partial p}}{{\partial s}} = u\frac{{\partial u}}{{\partial s}} + \frac{{\partial u}}{{\partial t}}\]
由于
\[\frac{{\partial z}}{{\partial s}} = \mathop {\lim }\limits_{ds \to 0} \frac{{dz}}{{ds}} = \cos \theta \]
将式(b)代入式(a),得
\[g\frac{{\partial z}}{{\partial s}} + \frac{1}{\rho }\frac{{\partial p}}{{\partial s}} + \frac{{\partial u}}{{\partial t}} + u\frac{{\partial u}}{{\partial s}} = 0\]
这就是理想流体一维流动的运动微分方程,也称欧拉方程。

2.	理想流体的伯努利方程

在恒定流动条件下,$ \frac{{\partial u}}{{\partial t}}=0 $;p,z,u
只是轴向距离s的函数。可将式(1-23)中偏导数改写
成全导数,从而得到理想液体一维恒定流动的欧拉方程
\[gdz + \frac{{dp}}{\rho } + udu = 0\]

由于微小流束的极限是流线,因此上述形式的欧拉方程是沿任意一根流线都是成立的。
式(1-24)表达了沿任意一根流线液体质点的压力、密度、速度和位移之间的微分关系。
$$gz + \int {\frac{1}{\rho }} dp + \frac{1}{2}{u^2} = \text{常数} $$ 

将式(1-24)沿流线积分得对于不可压缩的理想液体p=常数,再以g除各项则有
$$z + \frac{p}{{\rho g}} + \frac{{{u^2}}}{{2g}} =\text{常数} $$ 

这就是著名的伯努利方程。方程左端的各项分别代表单位重力液体的位能、
压力能和动 能或称比位能、比压能和比动能。
伯努利方程的物理意义是,理想的不可压缩液体在重力场中 
作恒定流动时,沿流线上各点的位能、压力能和动能之和是常数。


不难看出,伯努利方程的各项都具有长度量纲,
因此工程上常用液柱高度(称为水头)来 表示这三部分能量。
如图1-14所示,微小流束在1和2截面处的总水头均为H,而比位能、比
压能和比动能三者之间可以相互转换。
图中,ac和$a^{'}c^{'}$表示两截面的压力能和位能,
称为静水头,cb和$c{'}b{'}$表示两截面的动能,称为速度水头。

如果液体是在同一水平面内流动,或者流场中z坐标的变化与其他流动参数相比可以忽 略不计,则式(1-25)变成 

$$\frac{p}{{\rho g}} + \frac{{{u^2}}}{{2g}} =\text{常数} $$ 
该式表明,沿流线压力越低,速度越髙。

3.	实际液体的伯努利方程

由于实际液体在流动时存在有黏性,产生内庶 擦力,因而液体总的能星沿着流动方向逐渐减小。 
又由于液体在密闭的容器或管道中流动时,还会遇 到一些其他局部装置引起液体运动的扰动,同样也 要损失一部分能最。
这样,实际液体沿流絞上各点 的总机械能不再保持为常数。如任取两个点,则伯 努利方程应为
$$\frac{p_1}{\rho g}+z_1+\frac{u^2 _1}{2g}=\frac{p_2}{\rho g}+z_2+\frac{u^2_2}{2g}+h^{'} _w$$

式中$h^{'} _w$表示微小流束上从点1到点2单位重力液体的损失水头。

总流是由通过其通流截面全部微小流束所组成的。若求总流的伯努利方程,
只要将式(1-26)乘以微小流束上的液体重量$\rho gdQ$,
然后对总流通流載面$A_1$和$A_2$进行积分,即可求得,即

  
 \begin{equation}
  \begin{aligned}
 \int_{A_1}^{}z_{1}\rho gdQ+\int_{A_1}^{}\frac{p_1}{\rho g}\rho gdQ+  \int_{A_1}^{}\frac{u^2 _1}{2g}\rho gdQ
  = \\ \int_{A_2}^{}z_{2}\rho gdQ
  + \int_{A_2}^{}\frac{\rho _2}{\rho g}\rho gdQ
  +  \int_{A_2}^{}\frac{u^2 _2}{2g}\rho gdQ
  +\int_{A_1-A_2}^{}\rho gdQ 
  \end{aligned}
 \end{equation}
 

为了简化式(1-27)需引入两个概念:

(1)缓变流动。指流束内的流线夹角很小,几乎平行,通流概面总是垂直于流我。对缓变 流动而言,每一通流袱面都是与流动方向垂直的平面,这样,在每一通流截而上压力的分布即
可以按静压处理,即
$$z+\frac{p}{\rho g}=\text{常数}$$
于是公式(1-27)中等号两边前二项可写为
\[\int_{{{\rm{A}}_1}} {\left( {{z_1} + {\textstyle{{{p_1}} \over {\rho g}}}} \right)} \rho gdQ = \left( {{z_1} + {\textstyle{{{p_1}} \over {\rho g}}}} \right)\int_{{A_1}} {\rho gdQ}  = \left( {{z_1} + {\textstyle{{{p_1}} \over {\rho g}}}} \right)\rho g{Q_1}\]
\[\int_{{\rm{A}}2} {\left( {{z_2} + {\textstyle{{{p_2}} \over {\rho g}}}} \right)} \rho gdQ = \left( {{z_2} + {\textstyle{{{p_2}} \over {\rho g}}}} \right)\int_{{A_2}} {\rho gdQ}  = \left( {{z_2} + {\textstyle{{{p_2}} \over {\rho g}}}} \right)\rho g{Q_2}\]

(2)动能修正系数。由于实际速度在通流截面上是一个变觉,即给动能的计算带来了困 难,
而用平均速度v计算的动能代替用实际速度“计算的动能,必然有偏差,故需逬行修正而 
引入了动能修正系数$\alpha$ , $\alpha$ 表示用实际速度计算的动能与平均速度计算的动能的比借,
由式 (1-28)给出。
\[\alpha  = \frac{{\int_A {{\textstyle{{{u^2}} \over 2}}\rho dQ} }}{{{\textstyle{{{v^2}} \over 2}}\rho \int_A {dQ} }} = \frac{{\int_A {{u^2}dQ} }}{{{v^2}Q}} = \frac{{\int_A {{u^3}dA} }}{{{v^3}A}}\]

不难证明,动能修正系数是大于1的数,其数值与速度分布的均匀程度有关。
层流时约为 2;紊流时约为1。