\chapter{典型液压系统}

机床液压系统是根据机床的工作要求,选用合适的基本回路构成。本章通过对典型的机床液压系统的学习和分析,进一步加深对各个液压元件和回路综合应用的认识,并学会对机床液压系统的分析方法,为机床液压系统的调整、使用、维修或设计打下基础。各个典型系统图都用职能符号或结构式符号绘制,它表示了系统内所有液压元件及其连接或控制方式,其工作原理则通过机床的工作循环图和系统的动作循环表以及文字叙述或油液流动路线来说明。

\section{组合机床动力滑台液压系统}

组合机床是由通用部件和部分专用部件所组成的高效率专用机床,动力滑台是组合机床上实现进给运动的一种通用部件,配上动力头和主轴箱后便可以完成各种孔加工、端面加工等工序。液压动力滑台由液压缸驱动,在电气和机械装置的配合下可以完成各种自动工作循环。

图7-1和表7-1分别表示YT4543型动力滑台的液压系统图和系统的动作循环表。由图可见,这个系统在机械和电气的配合下,能够实现“快进$\rightarrow$ 工进$\rightarrow$ 停留$\rightarrow$ 快退$\rightarrow$ 停止”的自动工作循环,其工作情况如下:

\subsubsection {动力滑台快进}
按下启动按钮,电磁铁1\,DT通电,电液换向阀左位接人系统,顺序阀因系统压力不高仍处于关闭状态。这时液压缸作差动连接,限压式变量泵输出最大流量。系统中油液流动情况为: 

进油路:变量泵→单向阀$I_{1}$ $\rightarrow$ 换向阀(左位)$\rightarrow$ 一行程阀(右位)$\rightarrow$ 液压缸左腔;

回油路:液压缸右腔$\rightarrow$ 换向阀(左位)$\rightarrow$ 单向阀$I_{2}$ $\rightarrow$ 行程阀(右位)$\rightarrow$ 液压缸左腔。

\subsubsection {第一次工作进给}

当滑台快速前进到预定位置时,挡块压下行程阀。这时系统压力升高,顺序阀打开;变量泵自动减小其输出流量,以便与一工进调速阀的开口相适应。系统中油液流动情况为:

进油路:变量泵 $\rightarrow$ 单向阀$I_{1}$ $\rightarrow$ 换向阀(左位)$\rightarrow$ 一工进调速阀 $\rightarrow$ 电磁阀(右位)$\rightarrow$ 液压缸左腔;

回油路:液压缸右腔 $\rightarrow$ 换向阀(左位)$\rightarrow$ 顺序阀 $\rightarrow$ 背压阀 $\rightarrow$ 油箱。

\subsubsection {第二次工作进给}

当第一次工作进给结束时,挡块压下行程开关,电磁铁3\,DT通电。顺序阀仍打开,变量泵输出流量与二工进调速阀的开口相适应。系统中油液流动情况为:

进油路:变量泵 $\rightarrow$ 单向阀$I_{1}$ $\rightarrow$ 换向阀(左位)$\rightarrow$ 一工进调速阀 $\rightarrow$ 二工进调速阀 $\rightarrow$ 液压缸左腔;

回油路:液压缸右腔 $\rightarrow$ 换向阀(左位)$\rightarrow$ 顺序阀 $\rightarrow$ 背压阀 $\rightarrow$ 油箱。

\subsubsection {死挡块停留及动力滑台快退}

在动力滑台第二次工作进给碰到死挡块后停止前进,液压系统的压力进一步升高,压力继电器发出动力滑台快速退回的信号,电磁铁1\,DT断电,2\,DT通电,这时系统压力下降,变量泵流量又自动增大。系统中油液的流动情况为:
 
进油路:变量泵 $\rightarrow$ 单向阀$I_{1}$ $\rightarrow$ 换向阀(右位)$\rightarrow$ 液压缸 $\rightarrow$ 液压缸右腔。

回油路:液压缸左腔 $\rightarrow$ 单向阀$I_{3}$ $\rightarrow$ 换向阀(右位)$\rightarrow$油箱。

\subsubsection {动力滑台原位停止}

当动力滑台快速退回到原位时,挡块压下行程开关,使电磁铁1\,DT,2\,DT,3\,DT断电,这时换向阀处于中位,液压缸两腔封闭,滑台停止运动。系统中油液的流动情况为:

卸荷油路:变量泵 $\rightarrow$ 单向阀$I_{1}$ $\rightarrow$ 换向阀(中位) $\rightarrow$ 油箱。

由上述可知,YT4543型动力滑台的液压系统主要由下列一些回路组成:

\begin{enumerate}[(1),leftmargin=0pt,itemindent=3.5\ccwd]
 \item 由限压式变量叶片泵、调速阀、背压阀组成的容积节流调速回路;
 \item 差动连接式快速运动回路;
 \item 液换向阀式换向回路;
 \item 行程阀和电磁阀式速度换接回路;
 \item 三位换向阀式卸荷回路。
\end{enumerate}

系统具有以下一些特点:

\begin{enumerate}[(1),leftmargin=0pt,itemindent=3.5\ccwd]
 \item 系统采用了“限压式变量叶片泵-调速阀-背压阀”式调速回路,能保证稳定的低速运动(进给速度最小可达6.6\ mm/min)、较好的速度刚性和较大的调速范围(R$\approx$100)。
 \item 系统采用了限压式变量泵和差动连接式液压缸来实现快进,能量利用比较合理。滑台停止运动时,换向阀使液压泵在低压下卸荷,减少能量损耗。
 \item 系统采用了行程阀和顺序阀实现快进与工进换接,不仅简化了油路,而且使动作可靠,换接精度亦比电气控制式高。至于两个工进之间的换接则由于两者速度都较低,采用电磁阀完全能保证换接精度。
\end{enumerate}

从上面介绍的组合机床动力滑台液压系统工作原理来看,动力滑台的行程范围及有关加工尺寸等主要靠行程挡块来保证和调节,加工过程中滑台在指定位置上的停留时间可用定时器(或延时元件)来实现。目前普遍采用的一种方法是用可编程控制器(PC,\ Programmable\  Controller)来实现上述功能。

可编程控制器是一种以微型计算机为基础的工业控制器,其控制特点是以开关量为主,带有定时、计数等指令能,可完成顺序动作的逻辑控制操作。可编程控制器的可靠性高,逻辑关系易于修改,体积小,因此,它比现有的继电器控制线路有更大的优越性,它在类似的组合机床机电控制设备上已经得到了广泛的应用。读者若有兴趣,可查阅有关可编程控制器原理及应用方面的教材。

\section  {M1432A型万能外圆磨床的液压系统}

M1432A型万能外圆磨床主要用于磨削内外圆柱、圆锥以及阶梯形表面等。它是一种较典型的换向频繁而平稳和换向精度要求高的系统。工作台的往复运动和抖动、手动和机动的互锁、砂轮架的间歇进给和快速运动、尾架的松开等都是液压来实现的。图7\,-\,2所示为M1432A型万能外圆磨床的液压系统图。

\subsection {液压系统的工作原理}

\subsubsection {工作台的往复运动}

在图7\,-\,2所示状态下,开停阀、先导阀和换向阀都处于右端位置,工作台向右运动,主油路中的油液流动情况为:

进油路:液压泵 $\rightarrow$ 换向阀(右位)$\rightarrow$ 工作台液压缸右腔;

回油路:工作台液压缸左腔 $\rightarrow$ 换向阀(右位)$\rightarrow$ 先导阀(右位)$\rightarrow$ 开停阀(右位)$\rightarrow$ 节流阀 $\rightarrow$ 油箱。

当工作台向右移动到预定位置时,工作台上的左挡块拨动先导阀,并使它最终处于左端位置。这时操纵油路上$a_{2}$点接通高压油,$a_{1}$点接通油箱,使换向阀亦处于其左端位置,于是主油路中油液流动情况就变为:

进油路:液压泵 $\rightarrow$ 换向阀(左位)$\rightarrow$ 工作台液压缸左腔;

回油路:工作台液压缸右腔 $\rightarrow$ 换向阀(左位)$\rightarrow$ 先导阀(左位)$\rightarrow$ 开停阀(右位)$\rightarrow$ 节流阀 $\rightarrow$ 油箱。

工作台向左运动,并在其右挡块碰上拨杆后发生与上述情况相反的变换,使工作台又改变方向向右运动,如此不停地反复进行下去,直到开停阀拨向左位时才使运动停下来。

工作台换向过程:工作台换向时,先导阀先受到挡块的操纵而移动,接着又受到抖动缸的操纵而产生快跳。这样就使工作台的换向经历了迅速制动、停留和迅速反向启动三个阶段。具体情况如下:

当先导阀(见图7\,-\,2)被拨杆推着向左移动时,先导阀中段的右制动锥逐渐将通向节流阀的通道关小,使工作台逐渐减速,实现预制动。当工作台挡块推动先导阀直到先导阀阀芯右部环形槽使$a_{2}$点接通高压油,左部环形槽使$a_{1}$点接通油箱时,控制油路被切换。这时左、右抖动缸便推动先导阀向左快跳,因为这里的油液流动情况是:

进油路:液压泵$\rightarrow$ 精滤油器$\rightarrow$ 先导阀(左位)$\rightarrow$ 左抖动缸;

回油路:右抖动缸$\rightarrow$ 先导阀(左位)$\rightarrow$ 油箱。

液动换向阀亦开始向左移动,因为阀芯右端接通高压油。