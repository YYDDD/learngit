

闸门通大气。设管长为$l$,截面积为$A$,在阀门正常开启情况下,管中流速为$v_0$,压力为$p_0$(不计沿程失),当阀门突然关闭时,首先是紧靠门的一层厚度为$\varDelta l$的液体停上运动,它的动能在极短约时间内转化为压力增量△p,同时液体被压缩,压力位升高,如此继续下去,管中液体一层接一层地逐步停止运动,同时压力升高,在停止流动液体形成的高压区和在流动液体的原有低压区的分界(称为增压波面),以速度$a$向水胞方向传递,称为压力被传播。$a$是液压冲击波的传需速度,其值等于液体中声速。

在阀门关闭后的$t_1=l/a$时刻,第一次液压冲击波从阀门传到了管道入口,此时管中的液体全部停止流动。而且液体处于压缩状态,使管内压力大于水池中的压力,处于一种不平衡状态,于是管中紧邻人口处的第一层厚为$\varDelta l$的液体将会以速度$v_0$向水池冲击,与此同时,该层液体结束了受压状态,液体的压力增量随即消失,恢复到正常压强,这样,管中液体次结束受压状态,液体高压区和低压区的分界面即为减压波面,在阀门关闭后的$t_2=l/a$时刻管内全部液体的压力和体积都恢复了原状。但由于惯性作用,紧靠阅门的液体仍然企图以速度$v_0$流向水池,这就使得紧阀门的第一层液体开始受到拉松,因使压力突然降低$\varDelta p$大小。

同样,紧接各层液体依次放松,这就形成一减压波面,并以速度$v_0$向水池方向传去,经$t_3=3l/a$时刻后,减压波面传到管道人口处,管内全部液体都处于低压而且是静止的状态,这时水池中压力大于管中压力,在此压差作用下,液体又由水池向管中冲去,这又使管道入口处的第一层液体首先恢复原来正常情况下的压力和速度,接着依次一层一层地以速度$a$a向阀门方向恢复原状。直到$t_4=4l/a$时刻,管内全部液体的压力和速度都恢复到正常状态,即液体仍以速度$v_0$流向阀门。

这时若阀门仍然关闭着,则将重复上述四个过程,若无能量消耗,则上述情况将水远持续。

实际上由于液体的黏性和管壁变形都将消耗液体的能量,液压冲击产生的能量将逐渐消失,于是压力将逐渐减弱而直至消失。

2.运动部件制动时产生的液压冲击

如图1-28所示,活塞以正常运动速度$v_0$带动负载$\sum{m}$向左运动,当换向阀突然关闭时,油液被封死在的缸两腔及管道中,由于惯性作用,活塞不能立即停止运动,将继续向左运动使左腔内演液受到压缩,压力急剧上升达到某一峰值。产生液压冲击。封团在右腔的油液因容积扩大并没有油液补充进来将使压力突然降低。当运动部件的动能全部转化为油液的弹性能时,活塞将停止向左运动,此时油液的弹性能将释放出来,使括塞改变其运动方向而向右运动,这样来国运动将持续地振荡一段时间,直到泄漏与摩擦失耗尽了全部能量为止。

同样利用能量守恒定律,可以求出冲击压力峰值

$$
\frac{1}{2}\sum{mv_{0}^{2}=\frac{1}{2}K_h\varDelta l^2}
$$

\noindent 式中
\begin{tabular}[t]{ll}
       $\varDelta l$ & ——关闭阀门后活塞移动的距离;\\
       $K_h$ & ——液压弹簧刚性系数,由式(1-5)可知$K_h=A^2K/V$。

\end{tabular}   

%\noindent 
\noindent  又
$$
\varDelta pA=K_h\varDelta l^2
$$

将上边各式整理得

$$
\varDelta p=v_0\sqrt{\frac{\sum{mK}}{V}}
$$

由式(1-52)可以看出,运动部件质量越大,起始运动速度越大,产生的冲击压力越大。在推导公式(1-52)时,是假设速度减至零,并未考虑其他损失,因此公式是近似的。对以上两种情况分析得出,液压冲击现象对管道和液压机械都是十分有害的,因此应设法将其消除或减弱之,常用的办法有:

(1)缓慢关闭阀门。若使阀门关闭时间$t_c>2l/a$,则当返回的减压波回到阀门时,阀门还在关闭过程中,这后来产生的压力升高值将与返回的减压波相抵消掉一部分,因此,液压冲击压力峰值将减小。

(2)缩短管子长度$l$,即使$t_c>2l/a$减小,也同样可达到前项所说的效果。

(3)限制管中液体的流速$v_0$。

(4)在靠近液压冲击源处安装安全阀、蓄能器等装置。

液压冲击现象并非有百害而无一利,事实上人们早已利用液压冲击的能量制成了一种水、锤泵,用来扬水。

\textbf{例1-4} \ \ 
有一直径$d=205mm$,管壁厚度$\delta =10.5mm$的管道,管中水流速度$v=2m/s$,此时阀门处的压力$p_0=1.5MPa$,已知水的容积弹性模量$K=2.1×10^3MPa$,管壁材料的弹性模量$E=10^5MPa$,若阀门突然关闭,求管壁内产生的应力。

\textbf{解} \ \ \   
由式(1-52)得解由式(1-52)得

$$
\varDelta p=\rho v_0\sqrt{\frac{\frac{K'}{\rho}}{1+\frac{dK'}{\delta E}}}
$$

\noindent 取$\delta =1000kg/m^2$,则有

$$
\varDelta p=1000\times 2\times \sqrt{\frac{2.1\times 10^9/1000}{1+\frac{0.205\times 2.1\times 10^9}{0.0105\times 10^{11}}}}=2.44MPa
$$

\noindent 故发生液压冲击时的总压力应为

$$
p=p_0+\varDelta p=1.5+2.44=3.94MPa
$$

\noindent 此时,管壁中的应力为

$$
\sigma =\frac{pd}{2\delta}=\frac{3.94\times 20.5}{2\times 1.05}=38.46MPa
$$

\noindent 而正常时管壁中的应力为

$$
\sigma =\frac{p_0d}{2\delta}=\frac{1.5\times 20.5}{2\times 1.05}=14.6MPa
$$


\textbf{ 二、气穴(或空穴)}

在流动的液体中,如果某一点处的绝对压力低于液体的空气分离压,液体中溶解的空气就分离出来,产生大量气泡,这就是气穴。另外,当绝对压力低于液体的饱和蒸气压时,液体中会产生大量的蒸气泡,这也是气穴。气穴现象使液压装置产生噪声和振动,使金属表面受到腐蚀。为了说明这种现象的机理,有必要介绍一下液压油的空气分离压和饱和蒸气压。
1.空气分离压和饱和蒸气压

液压油总是含有一定量的空气的。液压油中所含空气体积的百分数称为它的含气量。空气可溶解在液压油中,也可以以气泡的形式混合在液压油中。空气的溶解量和液压油的绝对压力成正比。常用的矿物型液压油,常温时在一个大气压下约含有$5\%~10\%$的溶解空气,溶解空气对液压油的体积弹性模量没有影响。

在一定温度下,当液压油压力低于某值时,溶解在油中的过饱和的空气将会突然地迅速从油中分离出来,产生大量气泡,这个压力称为液压油在该温度下的空气分离压。含有气泡的液压油的体积弹性模量将降低。

当液压油在某温度下的压力低于一定数值时,油液本身将迅速汽化,产生大量蒸气气泡,这时的压力称为液压油在该温度下的饱和蒸气压。一般来说,饱和蒸气压相当小,比空气分离压小得多。几种液压油的饱和蒸气压值如表1-5所示。



由上述可知,要使液压油不产生大量气泡,它的压力最低不得低于液压油所在温度下的空气分离压。

2.节流口处的气穴现象

在液压系统中的节流口,在突然关闭的阀门附近,在吸油不畅的油泵吸油口等处,均可能产生气穴。现以图1-29所示节流口的喉部为例进行分析。根据伯努利方程知,该处流速大、压力低,如压力低于该液压油工作温度下的空气分离压,溶解在油中的空气将迅速地分离出来变成气泡。这些气泡随着p液流流到高压区时,会因承受不了高压而破灭,产生局部的液压冲击,发出噪声并引起振动。当附在金属表面上的气泡破灭时,它所产生的局部高温和高压会使金属剥落,使表面粗糙或出现海绵状的小洞穴。节流口下游部位常发生这种腐蚀的痕迹,这种现象称为气蚀。

其他像液压泵吸油管太细,安装位置太高等都会使吸油口绝对压力过低,即真空度太大,而产生气穴现象,使液压泵输出流量和压力急剧波动,系统无法稳定地工作;严重时使泵的机件腐蚀,出现气蚀现象。

\begin{center}
 \textbf{思考题和习题}
\end{center}

1-1  如图1-30所示,直径为$d$,重量为$F_G$的柱塞浸在液体中,并在外力$F$的作用下处于静止状态。若液体的密度为$\rho $,柱塞浸入深度为$h$,试确定液体在测压管内的上升高度$x$。

1-2  有一容器充满了密度为$\rho $的油(见图1-31),其压力$p$由水银压力计的读数$h$来确定。若测压计与容器以柔软胶管连接,现将测压管向下移动距离$a$,这时虽然容器中压力不变化,但测压管中的读数则由$h$变为$h+\varDelta h$。试求$\varDelta h$与$a$的关系式。


1-3  转轴直径$d=0.36m$,轴承长度$1=1m$,轴与轴承间的健隙$\delta =0.2mm$,其中充满动力黏度从$\mu =0.72Pa·s$的油,若轴的转速为$n=200r/min$,求克服油的黏性阻力所需的功率。

1-4  如图1-32所示,液压泵从油箱吸油,吸油管直径$d=6cm$,流量$Q=150L/min$,液压泵入口处的真空度为$0.2 \times 10^5Pa$,油液的运动黏度$v=20\times10^{-6}m^2/s$,油液的密度$\rho =900kg/m^3$,不计任何损失,求最大吸油高度。

1-5  将流量$Q=16L/min$的液压泵安装在油面以下,已知油的运动黏度
$ v = 0.11 cm^2 / s $,油液的密度$\rho=880kg/m^3$,弯头处的局部阻力系数$\zeta =0.2$,其他尺寸如图1-33所示。求液压泵入口处的绝对压力。

1-6  如图1-34所示,管道输送$\rho=900kg/m^3$的液体,已知$h=15m$,1处的压力为$4.5×105Pa$,2处的压力为$4×105Pa$,判断管中油流的方向。

1-7  如图1-35所示,活塞上作用有外力$F=3000N$,活塞直径$D=50mm$,若使油从液压缸底部的锐缘孔口流出,设孔口直径$d=10mm$,孔口速度系数$C_v=0.97$,流量系数$C_d=0.63$,油液的密度$\rho =870kg/m^3$,不计摩擦,试求作用在液压缸缸底壁面上的力。


