常重要。这类液压系统是高压,又有保压要求,密封问题更为突出,应予以特别注意。


\section*{思考题与习题}

7-1\quad YT4543型动力滑台的液压系统:

(1)液压缸快进时如何实现差动连接?

(2)如何实现液压缸的快慢速运动换接和进给速度的调节?

7-2\quad M1432A型万能外圆磨床的液压系统:

(1)时间控制换向回路及行程控制换向回路的工作原理是怎样的?各适用于何种情况?

(2)换向阀实现第一次快跳、慢移和第二次快跳时,三种不同的回油通道是怎样的?各起什么作用?

(3)抖动缸起何作用?

(4)尾顶针与砂轮架为何要互锁?油路如何实现?

(5)闸缸起什么作用?

7-3\quad 列出图7-8所示油路中电磁铁动作状态表(电磁铁通电用“+”表示)。

\begin{figure}
    \centering
    \ifOpenSource
    \includegraphics[width=5cm]{logo.png}
    \else
    \includegraphics{cover,jpg}
    \fi
    \caption{logo}
    \label{fig:fig0701}
 \end{figure}


7-4\quad YA32-200四柱式万能液压机的液压系统:

(1)如何实现主缸的快速下行、减速加压、保压延时、泄压回程及回程停止?

(2)如何实现顶出缸的顶出、顶退、停止及压边?

(3)如何解决静止时的下滑问题?

(4)如何解决主缸和顶出缸的动作协调问题?

(5)电液换向阀由外控供油,为什么不宜直接引用主油路的高压油?

\newpage

\chapter{机床液压系统的设计与计算}

\section{概述}

在前述几章中,对液压传动的基本原理,液压元件的结构、工作原理和基本回路(包括典型机床的液压系统)等进行了分析,本章的任务是应用这些基本知识来讨论液压传动系统设计、计算的步骤和方法。液压系统的设计必须重视调查研究,注意借鉴别人的经验。一般说来,液压系统设计应着重解决的主要问题是满足工作部件对力和运动两方面的要求。在满足工作性能和工作可靠性的前提下,应力求系统简单、经济且维修方便。

在机床上决定采用液压传动方案之后,液压系统的设计任务才会被提出来。具体的设计步骤大致如下:

 1.明确设计依据,进行工况分析

(1)设计依据。设计开始时,首先根据任务进行调查研究,明确下列主要问题:

 1)机床总体布局和加工的工艺要求,明确机床哪些运动采用液压传动,用哪种液压执行元件及空间尺寸的限制。

 2)机床的工作循环(复杂的机床要给出动作周期表),液压执行元件的运动方式,运动速度,调整范围,工作行程等。

 3)液压执行元件的负载性质和变化范围,以及精度、平稳性要求。

 4)机床各部件(电气、机械、液压)的动作顺序、转换和互锁要求等。

 5)其他要求,如工作环境、占地面积和经济性等。

 (2)工况分析。经调查研究之后,就可以对液压执行元件进行工况分析,即动力分析(负载循环图)和运动分析(速度循环图)。有些简单的液压系统,可以不绘制上述两种图,但必须找出最大负载点、最大速度和最大功率点。通过负载循环图和速度循环图可以清楚地看出液压执行元件的负载、速度和功率随时间变化的规律,它们是确定系统方案,选择泵、阀和电机功率的依据,同时便于设计中检查、改进和完善液压系统。

 2.初步确定液压系统参数

压力与流量是液压系统最主要的两个参数。当液压回路尚未确定时,其系统压力损失和泄漏都无法估算。这里所讲的确定系统主要参数,实际上是确定液压执行元件的主要参数。

 3.拟定液压系统图

拟定液压系统图是整个设计中的重要步骤,它将以简图的形式全面、具体地体现设计任务中提出的动作要求和性能。这一步骤涉及的面广,需要综合运用前面各章,特别是第四、六章中的知识,亦即要拟定一个比较完善的液压系统,必须对各种基本回路、典型液压系统有全面深刻的了解。

 4.计算、选择或设计液压元件

 对泵和阀类元件主要是通过计算来确定它们的两个主要参数,即压力和流量。这两个参数是选择泵、电机、阀及辅助元件的依据。选择元件时应尽量选用标准元件,在有特殊要求时才设计专用元件。

 5.液压系统的性能验算和绘制工作图、编写技术文件

 性能验算包括系统压力损失验算和液压系统的发热与温升验算。

 正式工作图一般包括正式的液压系统工作原理图、系统管路装配图和各种非标准液压元件的装配图和零件图。

 正式的液压系统原理图,就是对初步拟定的系统图经过反复修改完善,选定了液压元件之后,所绘制的液压系统图。图中应列出明细、规格和调整值;对复杂系统应按各执行元件的动作程序绘制工作循环图和电气控制程序状态表。一般按停车状态画液压系统原理图。然后绘制系统的管路装配图(或管路布置示意图),在管路装配图上应表示出各液压部件和元件在机床或工作地的位置和固定方式,油管的规格和分布位置,各种管接头的形式和规格等。

 对于自行设计的非标准液压件如液压缸、油源站等,必须画出部件装配图和专用零件图。

 当绘制装配图时,应考虑安装、使用、调整和维修方便,管道应尽量短,其中弯头和接头尽量少。

 编写的技术文件,一般应包括设计任务书、计算书和使用维修说明书;零、部件目录表,标准件,通用件和外购件总表等。

 应该指出,在实际设计过程中,根据所设计机床的用途和掌握的资料情况,上述步骤有的可以省略,有的可以合并。同时,各设计步骤是相互联系、相互影响的。设计中往往是互相穿插,交叉进行,有时还要经过多次反复才能完成。


\section{液压系统设计与计算举例}

某厂自制一台卧式钻、镗组合机床的动力滑台,其工况要求:

(1)工作性能和动作循环。动力滑台加工铸铁的箱形零件的孔系,要求孔的加工精度为二级,表面粗糙度为$\rm{R_a1.6}$(精镗)或$\rm{R_a6.3}$(粗镗)。工作循环为快进、工进、快退、原位停止。

(2)动力和运动参数。轴向最大切削力12\ 000N,动力滑台自重20\ 000N,工作进给速度要求在$0.33×10^{-3}\sim 20×10^{-3}$\ m/s范围内无级调节,快进和快退的速度均为$v_1=0.1$\ m/s导轨形式为平导轨,静、动摩擦系数:$f_{\rm{s}}=0.2$,$f_{\rm{d}}=0.1$。往返运动的加速、减速时间为0.2\ s,快进行程$L_1$\ 为0.1\ m,工进行程$L_2$\ 为0.1\ m。

(3)自动化程度。采用液压与电气配合,实现工作自动循环。

根据上述工况要求和动力滑台的结构安排,应采用液压缸为执行元件,由液压缸筒与滑台固结完成工作循环,活塞杆固定在床身上。由于要求快进与快退的速度相等,为减少液压泵的供油量,决定采用差动型液压缸,取液压缸前、后腔的有效工作面积为2:1,活塞杆较粗,结构