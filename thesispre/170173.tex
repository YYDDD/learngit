
\noindent 上可允许油管通过,进、出油管穿过活塞杆,直接使用硬管与液压装置或液压泵连接。这样就避免了由于较长软管的弹性变形引起动力滑台在转换中产生“前冲”、“后坐”现象。使液压缸无杆腔为高压工作腔,这样能得到较大的输出动力,并可得到较低的稳定工作速度,以便满足精加工的要求。

下面按设计步骤进行计算。

\section{计算外负载}
动力滑台受力情况如图8-1所示。当机床上的液压缸做直线往复运动时,液压缸必须克服的外载$F$为
\begin{equation}
    F=F_t +F_f+F_{\text{m}}+F_g+F_{\text{b}}
\end{equation}
式中
$F_t$——工作负载;

\ $F_f$——摩擦负载;

\ $F_{\text{m}}$——惯性负载;

\ $F_g$——重力负载;

\ $F_{\text{b}}$——背压阻力。

\subsection{工作负载}
工作负载与机床的工作性质有关,它可能是定值,也可能是变值。一般工作负载是时间的函数,即$F_t=f(t)$,需根据具体情况分析决定。如机床进给系统,其工作负载就是沿进给方向的切削分力,若负载方向与进给方向相反,如钻、镗、扩、攻丝时沿进给方向的切削力(亦称切削阻力)称正值负载。负载方向与进给方向相同,如顺铣的切削阻力称负值负载。切削阻力值的大小由实验测出或按切削力公式估算。

本例切削阻力为已知,即$$F_t=12\ 000\ \text{N}$$

\subsection{摩擦力}
液压缸驱动工作部件工作时要克服机床导轨处的摩擦阻力,它与导轨形状、安放位置及工作台的运动状态有关。

图8-2所示为机床上常见的两种导轨形式,其摩擦阻力的估算公式如下:

平导轨
\begin{equation}
    \label{8-2}
    F_f=f(F_g+ F_{\text{n}})
\end{equation}

V型导轨
\begin{equation}
    \label{8-3}
    F_f=f\frac{F_g+ F_{\text{n}}}{\sin{\frac{\alpha}{2}}}
\end{equation}
式中
\ $F_{\text{n}}$——切削力垂直于导轨上的正压力;

\ \ $\alpha$\ ——V形导轨的夹角;

\ \ $f$\ ——导轨摩擦系数,启动时按静摩擦系数$F_{\text{s}}$计算,其余按动摩擦系数$f_{\text{d}}$计算,参考表8-1。

工作部件倾斜$\beta$角放置时如图8-3所示,将$(F_g+F_{\text{n}})$变为$(F_g\cos{\beta}+F_{\text{n}})$后代入式\ref{8-2}
和式\ref{8-3}中。

详细计算各种导轨上的摩擦阻力见机床设计有关部分,计算中如颠覆力矩数值较小可以
忽略不计。

本例导轨摩擦阻力由动力滑台和颠覆力矩产生,若忽略颠覆力矩的影响,则

静摩擦阻力
$$F_{f\text{s}}=f_{\text{s}}F_g=0.2\times 20\ 000=4\ 000\ \text{N}$$

动摩擦力
$$F_{f\text{d}}=f_{\text{d}}F_g=0.1\times 20\ 000=2\ 000\ \text{N}$$

\subsection{惯性负载}
工作部件在启动和制动过程中产生惯性力,可按牛顿第二定律求出,即
\begin{equation}
    F_m=ma=\frac{F_g}{g}\frac{\Delta v}{\Delta t}
\end{equation}
式中
\ \ $g$\ ——重力加速度;

$\Delta v$——加(减)速时速度的变化量;

$\Delta t$\ ——启动或制动时间,一般机床的主运动取$0.2\thicksim 0.5$ s,进给运动取$0.1\thicksim 0.5$ s,磨床取$0.01\thicksim 0.05$ s,工作部件较轻或运动速度较低时取小值。

本例惯性阻力包括以下两部分:

(1)动力滑台快速时惯性阻力$F_m$。动力滑台启动加速、反向启动加速和快退减速制动的加速度相等,$\Delta v=0.1\ \text{m/s}$,$\Delta t=0.2$ s,故惯性阻力为
$$F_m=\frac{F_g}{g}\frac{\Delta v}{\Delta t}=\frac{20\ 000}{9.8}\times\frac{0.1}{0.2}\thickapprox 1\ 020\ \text{N}$$

(2)动力滑台工进时惯性阻力$F'_m$。动力滑台由工进转换到制动是减速,取$\Delta v=20\times 10^{-3}\ $m/s,$\Delta t=0.2\ $s,故惯性阻力为$$F'_m=\frac{F_g}{\Delta t}=\frac{20\ 000}{9.8}\times \frac{20\times 10^{-3}}{0.2}\thickapprox 204\ \text{N}$$

\subsection{重力负载}
当工作部件垂直运动或倾斜放置时,它的自重也是一种负载,向上移动时为正负载,向下运动时为负负载。当工作部件水平放置时,$F_g=0$。

本例由于动力滑台为卧式放置,所以负载不考虑重力。

以上为液压缸所克服的外负载,实际上,液压缸工作时还必须克服其内部密封装置产生的摩擦阻力$F_{\text{s}}$,它包括活塞及活塞杆处的摩擦力,其值与密封装置的类型、液压缸制造质量和油液工作压力有关,计算比较繁琐,详细计算查液压传动手册中有关部分,一般将它计入液压缸的机械效率中。

此外,液压缸还必须克服回油路上的阻力,称为背压阻力$F_{\text{b}}$,其值为
\begin{equation}
    F_{\text{b}}=p_{\text{b}}A
\end{equation}
式中
$A$——回油腔有效工作面积;

$p_{\text{b}}$——液压缸背压,在系统方案、结构尚未确定之前,一般按经验数据估算一个数值,如进油节流调速时取$p_{\text{b}}=(2\sim 5)\times 10^5\ \text{Pa}$;回油路上有背压阀或调速阀时取$p_{\text{b}}=(5\sim 15)\times 10^5\ \text{Pa}$;对于闭式回路$p_{\text{b}}=(8\sim 15)\times 10^5\ \text{Pa}$。

根据以上分析,计算各工况负载列表8-2。本机床动力滑台所受负载亦为液压缸所受负载。

\section{绘制负载图和速度图}
根据已给的快进、快退、工进的行程和速度,配合表8-2中相应负载的数值,可绘制液压缸的$F-l$与$v-l$图,或近似计算快进、工进、快退的时间如下:

\subsection{快进}
$$t_1=\frac{l_1}{v_1}=\frac{100\times 10^{-3}}{0.1}=1\ \text{s}$$

\subsection{工进}
工进所需最长时间$t_{2\max}$为
$$t_{2\max}=\frac{l_2}{v_{2\min}}=\frac{100\times10^{-3}}{0.33\times 10^{-3}}=303\ \text{s}$$

工进所需最短时间$t_{2\min}$为
$$t_{2\min}=\frac{l_2}{v_{2\max}}=\frac{100\times10^{-3}}{20\times 10^{-3}}=5\ \text{s}$$

\subsection{快退}
$$t_3=\frac{l_3}{v_3}=\frac{200\times10^{-3}}{0.1}=2\ \text{s}$$

配合表8-2中相应负载的数值,可绘制$F-t$和$v-t$图,如图8-4所示。

此图清楚地表明了液压缸在运动循环内负载的变化规律。图中最大负载值是初选液压缸工作压力和确定液压缸结构尺寸的依据。

\section{确定液压系统参数}
\subsection{初选液压缸的工作压力}
液压缸工作压力的选择是否合理,直接影响到整个系统设计的合理性,确定时不能只考虑满足负载要求,应全面考虑液压装置的性能要求和经济性。如果液压缸的工作压力选定较高,则泵、缸、阀和管道尺寸可选得小些,这样结构较为紧凑、轻巧,加速时惯性负载也小,易于实现高速运动的要求。但工作压力太高,对系统的密封性能要求也相应提高了,制造较困难,同时缩短了液压装置的使用寿命。此外,高压会使构件弹性变形的影响增大,运动部件容易产生振动。

对于各类机床的液压系统,由于各自特点和使用场合不同,其液压缸的工作压力亦不相同,一般常用类比法,参考表8-3或表8-4来选择。