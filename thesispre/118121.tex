
\chapter{辅助装置}

液压系统中的辅助装置,是指除液压泵、液压缸(包括液压马达)和各种控制阀之外的其他各类组成元件,如油箱、滤油器、蓄能器、压力表、密封件和管件等等。它们虽称之为辅助装置,但却是液压系统不可缺少的组成部分,而且它们的性能对液压系统的工作性能好坏有直接影响。因此,对它们的设计和选用不能掉以轻心。本章只对蓄能器、滤油器、管件和油箱的设计、选用做一介绍。

\section{蓄能器}

\subsection{蓄能器的用途}

蓄能器是储存和释放液体压力能的装置,它在液压系统中的主要用途有以下几个方面:

\subsubsection{短期大量供油}

对于短时间内需要大量压力油的液压系统,采用蓄能器辅助供油可减小液压泵容量,从而减少了电机功率的消耗,降低了液压系统的温升。

\subsubsection{维持系统压力}

在液压系统的保压回路中采用蓄能器。图5-1所示是蓄能器用于夹紧油路的情况。当压力达到压力继电器调定压力时,压力继电器发出信号,使二位二通电磁阀换向,液压泵卸荷,由蓄能器把原先储存起来的压力油供应出来,补偿系统泄漏,以维持系统压力。这样做也可以减少电机功率消耗,降低系统温升。

\subsubsection{吸收冲击压力或液压泵的脉动压力}

对于由液压缸的突然停止或换向,换向阀的突然关闭或换向以及液压泵的突然启、停所引起的液压冲击,可采用蓄能器来加以吸收,避免系统压力过高造成元件损坏。对于一些要求液压源供油压力恒定的液压系统,需要在液压泵出口处安装蓄能器,以吸收液压泵的脉动压力。图5-2给出了在上述情况下使用蓄能器的情况。用来吸收冲击压力的蓄能器尽可能安装在靠近产生冲击的地方。

除以上三项用途外,蓄能器还可作紧急动力源用,以及作热膨胀补偿器用。也可用来改善压力补偿式变量泵的频率特性。

\subsection{蓄能器的种类}

蓄能器的类型有重锤式、弹簧式和充气式等几种,但在机床上常采用的是充气式蓄能器。下面就只介绍这类蓄能器的结构和性能。

\subsubsection{活塞式蓄能器(见图5-3)}

在活塞2的上腔1中充有高压气体,下腔3与液压系统管路相通,进入压力油。活塞随着蓄能器中油压的增减在缸筒内移动。这种蓄能器结构简单;油气隔离,油液不易氧化又能防止气体进入,工作可靠;安装容易;维护方便;寿命长。但活塞有惯性和摩擦阻力,故反应不灵敏,容量小,主要用来蓄压。

\subsubsection{气囊式蓄能器(见图5-4)}

气囊3用特殊橡胶制成,固定在壳体2的上半部。气体(常用氮气)从气门1充入,气囊外面加压力油。在蓄能器下部有一受弹簧力作用的提升阀,它的作用是防止油液全部排出时气囊膨胀出壳体之外。这种蓄能器的优点是气囊的惯性小,因而反应快,容易维护,重量轻,尺寸小,安装容易。缺点是气囊制造困难。气囊有折合型和波纹型两种,前者容量较大,适用于蓄能器,后者则适用于吸收冲击。

\subsubsection{隔膜式蓄能器(见图5-5)}

用耐油橡胶隔膜把油和气分开,工作原理与上述两种相同。其优点是容器为球形,重量与体积之比值最小;缺点是容量很小。适用于吸收冲击,广泛用在航空机械中。

\subsubsection{气瓶式蓄能器}

这是一种油和气在壳体内直接接触的蓄能器。其优点是容量大,惯性小,反应灵敏,轮廓尺寸小,没有摩擦损失;缺点是气体易混入油中,影响系统工作的平稳性,气体消耗量大,需经常补充,附属设备多(空气压缩机、高低位液面计等)。仅适用于中、低压大流量回路。

\subsection{蓄能器的计算}

选用蓄能器时,应知道它该有多大的容量,而计算蓄能器容量的方法又视其使用情况有所不同。下面以气囊式蓄能器为例,来说明其容量的计算方法。

\subsubsection{储存能量时的容量计算方法}

蓄能器容量$V_{\Lambda}$和充气压力$p_{\Lambda}$是根据它在工作中将要输送出去的油液体积$V_{\mbox{\scriptsize W}}$,系统最高工作压力$p_{1}$和所要维持的最低工作压力$p_{2}$来决定的。由气体定律可知
\begin{equation}
p_{\Lambda}V^{n}_{\Lambda}=p_{1}V^{n}_{1}=p_{2}V^{n}_{2}=\mbox{常数}
\end{equation}
式中
\begin{tabular}[t]{ll}
$V_{1}$&——最高压力下气体的体积;\\
$V_{2}$&——最低压力下气体的体积;\\
n&——指数。
\end{tabular}

当蓄能器用来保持系统压力、补偿泄漏时,它释放能量的速度是缓慢的,可以认为气体在等温下工作,取n=1;当蓄能器用来大量供应油液时,它释放能量的速度是迅速的,可认为气体在绝热条件下工作,取n=1.4。

令$V_{\mbox{\scriptsize W}}=V_{2}-V_{1}$,因此,由式(5-1)得
$$V_{A}=(\frac{p_{2}}{p_{\Lambda}})^{\frac{1}{n}}V_{2}=(\frac{p_{2}}{p_{\Lambda}})^{\frac{1}{2}}(V_{\mbox{\scriptsize W}}+V_{1})=(\frac{p_{2}}{p_{\Lambda}})^{\frac{1}{n}}[V_{\mbox{\scriptsize W}}+(\frac{p_{\Lambda}}{p_{1}})^{\frac{1}{n}}V_{\Lambda}]
$$
整理后,得
$$
V_{\Lambda}=\frac{V_{\mbox{\scriptsize W}}(\frac{p_{2}}{p_{\Lambda}})^{\frac{1}{n}}}{1-(\frac{p_{2}}{p_{1}})^{\frac{1}{n}}}
$$
故有
\begin{equation}
V_{\mbox{\scriptsize W}}=V_{\Lambda}p_{\Lambda}^{\ \frac{1}{n}}[(\frac{1}{p_{2}})^{\frac{1}{n}}-(\frac{1}{p_{1}})^{\frac{1}{n}}]
\end{equation}
$p_{\Lambda}$值在理论上可与$p_{2}$值相等,但由于系统中有泄漏,为了保证系统压力为$p_{2}$时蓄能器还有可能补偿泄漏,应使$p_{\Lambda}>p_{2}$,一般取$p_{2}=(0.8\sim0.85)p_{\Lambda}$。

\subsubsection{吸收液压冲击时蓄能器容量的计算}

从理论上虽可导出适用于完全液压冲击的容量计算公式,但在实际应用中常采用下述经验计算公式
\begin{equation}
V_{\Lambda}=\frac{0.004Qp_{2}(0.0164L-t)}{p_{2}-p_{1}}
\end{equation}
式中
\begin{tabular}[t]{ll}
$V_{\Lambda}$&——蓄能器容量(L);\\
L&——产生冲击波的管道长度(m);\\
Q&——阀口关闭前管内流量(L/min);\\
t&——阀口由开到关闭的持续时间(s);\\
$p_{1}$&——阀口关闭前的工作压力($10^{5}\mbox{Pa}$);\\
$p_{2}$&——系统允许的最大冲击压力,一般可取$p_{2}=1.5p_{1}(10^{5}\mbox{Pa})$。
\end{tabular}

\subsubsection{吸收液压泵脉动压力时蓄能器容量计算}

一般采用以下经验公式进行计算
\begin{equation}
V_{\Lambda}=\frac{q^{i}}{0.6K}
\end{equation}
式中
\begin{tabular}[t]{ll}
q&——液压泵每转排量(L/r);\\
i&——排量变化率$\frac{\Delta{q}}{q}$,$\Delta{q}$是超过平均排量的过剩排出量(L);\\
K&——液压泵的压力脉动率,$K=\frac{\Delta{p}}{p_{\mbox{\scriptsize p}}}$,是压力脉动单侧振幅。
\end{tabular}

使用时,取蓄能器充气压力$p_{\Lambda}=0.6p_{\mbox{\scriptsize p}}$。

\section{滤油器}

\subsection{对滤油器的要求}

对液压系统中保持油的清洁十分重要,因为油中的杂质颗粒会引起相对运动零件划伤、磨损以至卡死,或堵塞节流阀和管道小孔导致液压系统不能正常工作,因此需要对油液进行过滤。一般对过滤器的基本要求是:

(1)具有较好的过滤能力,即能阻挡一定尺寸以上的机械杂质;

(2)通油性能好,即油液全部通过时不致引起过大的压力损失;

(3)过滤材料要有足够的机械强度,在压力油作用下不致破坏;

(4)过滤材料耐腐蚀,在一定温度下工作有足够的耐久性;

(5)容易清洗和便于更换滤芯;

(6)价格便宜。

滤油器的过滤精度按过滤颗粒的大小可分为四级:粗滤油器(滤去杂质直径大于0.1mm)、普通滤油器(滤去杂质直径为0.1$\sim$ 0.01mm)、精滤油器(滤去杂质直径为0.01$\sim$ 0.005mm)、特精滤油器(滤去杂质直径为0.005$\sim$ 0.001mm)。
