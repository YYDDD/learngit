为计算方便,将工进时油流通过各种阀的流量和压力损失列于表8-14。

\noindent 利用式(8-19),计算各阀局部压力损失之和$\sum \triangle p_\text{ v}$如下:
\[
\begin{split}
\sum \triangle p_\text{v}  & = 2\times10^5\times(\frac{4.62}{25})^2+2\times10^5\times(\frac{4.62}{25})^2+5\times10^5\times(\frac{4.62}{25})^2+\\ & \frac{1}{2}\times1.5\times10^5\times(\frac{2.31}{25})^2+\frac{1}{2}\times6\times10^5\times(\frac{2.31}{10})^2 =\\
&0.47\times10^5\rm \ Pa
\end{split}
\]

取油流通过集成块时的压力损失为
\[
\Delta p_\text{I}=0.3\times10^5\rm \ Pa
\]

故工进时总的局部压力损失为
\[
\sum \Delta p_2 =(0.47+0.3)\times10^5 = 0.77\times10^5\rm \ Pa
\]
所以 $\sum \Delta p =(0.5+0.77)\times10^5 = 1.27\times10^5\rm \ Pa$

这个数值加上液压缸的的工作压力(由外负载决定的压力)和压力继电器                                                                                                                                                                                                                                                                                                                                                                                                                                                                                                                                                                          要求系统调高的压力(取其值为$5\times10^5\rm \ Pa$),可作为溢流阀调整压力的参考数据。其压力调整值p为
\[
p=\sum\Delta p+p_1+5\times10^5\rm \ Pa
\]

\noindent 式中$p_1$为液压缸工进时克服外负载所需压力,
\[
p_1=\rm F_0/A_1=15556/(38.5\times10^{-4})=40.4\times10^5\ \rm Pa
\]
所以$p=(40.4+1.27+5)\times10^5=46.67\times10^5\ \rm Pa$

这个值比估算的溢流阀调整压力值$56.4\times10^5\ \rm Pa$小。因此,主油路上的元件和油管直径均可不变。

应该指出,本系统液压缸快退时,由于流量大和液压缸前后腔压力折算的影响,此时管路系统总的压力损失比工进时要大。若工进时负载较小,则其溢流阀的调整压力就有可能要按快退时所需压力调定。

\subsection{液压系统的发热与温升预算}

从图8-4知,本机床的工作时间主要是工进工况。为简化计算,主要考虑工进时的发热,故按工进工况验算系统温升。

\subsubsection{液压泵的输入功率}

工进时小流量泵的压力$p_\text{p1}=56.4\times10^5\rm \ Pa$,流量$Q_\text{p1}=0.1\times10^{-3} \rm \ m^3/s$,小流量泵功率为
\[
\rm P_1=\frac{p_\text{p1}Q_\text{p1}}{\eta_\text{p}}=\frac{56.4\times0.1\times10^2}{0.75}=752\  \rm W
\]
式中$\eta_\text{p}$为液压泵的总效率。

工进时大流量泵卸荷,顺序阀的压力损失$\Delta p=1.5\times10^5\rm \ Pa$,即大流量泵的工作压力$p_\text{p2}=1.5\times10^5\rm \ Pa$,流量$Q_\text{p2}=0.15\times10^{-3}\rm \ m^3/s$,大流量功率
\[
\rm P_2=\frac{p_\text{p2} Q_\text{p2}}{\eta_\text{p}}=\frac{1.5\times0.15\times10^2}{0.75}=30\rm \ W
\]
故双联泵的合计输入功率
\[
\rm P_i=\rm P_1+\rm P_2=752+30=782\rm \ W
\]
\subsubsection{有效功率}

工进时,液压缸的负载$F=14000\rm \ N $(见表8-2),取工进速度$\upsilon=1.67\times10^{-3} \rm \ m/s$(0.1\ m/min),输出功率$P_0$为
\[
\rm P_0=f\upsilon=14000\times0.00167=23.4\rm \ W
\]
\subsubsection{系统发热功率$\rm P_\text{h}$}

系统的发热功率$\rm P_\text{h}$为
\[
\rm P_\text{h}=\rm P_\text{i}-\rm P_\text{o} \approx 759\rm \ W
\]
\subsubsection{散热面积}

油箱容积$V=105\rm \ L=105\times10^{-3}\rm \  m^3$

油箱近似散热面积A为
\[
\rm A=0.065\sqrt[3]{V^2}=0.065\sqrt[3]{105^2
}=1.447\rm \ m^2
\]
\subsubsection{油液温升$\Delta T$}

假定采用风冷,取油箱的散热系数$C_\text{T}=23\rm \ W/(m^2\cdot ^\text{o} \rm C)$,利用式(8-20)可得油液温升为
\begin{equation}
\Delta T=\frac{P_\text{h}}{\sum C_\text{T} A}=\frac{759}{23\times1.447}\approx22.8 \  ^\text{o} \rm C
\end{equation}
设夏天的室温为30\ $ ^\text{o} \rm C$,则油温为$30+22.8=52.8\  ^\text{o} \rm C$,没有超过最高允许油温($50\thicksim70\  ^\text{o} \rm C$)。
\begin{center}
\textbf{思考题和习题}
\end{center}
\qquad 8-1\qquad 试按图8-10所示压力机的液压系统,对其系统主要工作参数进行计算。已知:

(1)工作循环为快速下降$\rightarrow $压制工件$\rightarrow $快速退回$\rightarrow $原位停止(或再快速下降);

(2)液压缸无杆腔面积$A_\text{1}=1\ 000\ \rm cm^2$,有杆腔有效面积$A_\text{2}=50\ \rm cm^2$,移动部件自重$F_\text{g}=5\ 000\ \rm N$;

(3)快速下降时的外负载$F_\text{1}=10\ 000\ \rm N$,速度$\upsilon_\text{1}=6\ \rm m/min$;

(4)压制工件时的外负载$F_\text{2}=50\ 000\ \rm N$,速度$\upsilon_\text{2}=0.2\ \rm m/min$;

(5)快速回程时的外负载$F_\text{3}=10\ 000\ \rm N$,速度$\upsilon_\text{3}=12\ \rm m/min$;

管路压力损失、泄露损失、液压缸的密封摩擦力以及惯性力等均忽略不计。

试求:

(1)液压泵的最大压力及流量。

(2)阀3,4,6各起什么作用?它们的调整压力各为多少?

8-2\qquad 某组合机床的动力滑台,其液压系统如图8-11所示,其工作循环为快进$\rightarrow $工进$\rightarrow $快退$\rightarrow $原位停止。

已知:液压缸直径D=63\ mm,活塞杆直径d=45\ mm,工作负载$F=16\ 000\ \rm N$,液压缸的效率$\eta_\text{m}=0.95$,不计惯性力和导轨摩擦力,快速运动$\upsilon_\text{1}=7\ \rm m/min$,工作进给速度$\upsilon_\text{2}=53\ \rm mm/min$,系统总的压力损失折合到进油路上$\sum \Delta p=5\times10^5\ \rm Pa$。

试求:

(1)该系统实现工作循环时电磁铁、行程阀、压力继电器的动作顺序表。

(2)计算并选择系统所需元件,并在图上标明各元件型号。

8-3\qquad 试按下列技术条件设计一台拉床的液压系统并对系统进行计算。

(1)最大切削力$F=10\ 000$N;

(2)工作进给速度$\upsilon_\text{1}$为0.5\ $\thicksim$\ 4m/min;

(3)快速回退速度$\upsilon_\text{2}$为10\ $\thicksim$\ 20m/min;

(4)运动循环为:工进$\rightarrow $停(或不停)$\rightarrow $快退$\rightarrow $停(或再工进);

(5)工作行程 $s$不小于1.2\ m;

(6)加工时要运动平稳。

以下条件供设计时参考:

(1)可按容积调速系统设计;

(2)用一台ZBSV40轴向柱塞式手调变量泵,可用电机直接连接,取转速$n=1460\rm \ r/min$,液压泵总效率$\eta=0.9$;

(3)设本拉床进油管长$L_\text{1}=4\ \rm m$,回油管长$L_\text{2}=3\ \rm m$,活塞杆直径$d=55\ \rm mm$,滑鞍重$3\ 000\ \rm N$;

(4)活塞杆密封为V形密封,活塞密封用活塞环。

\chapter{液压伺服系统}

液压伺服系统是采用液压控制元件和液压执行元件,根据液压传动的原理建立起来的。它能使执行元件以一定的精度自动地眼随微弱的输入信号而动作。作为一种自动控制系统,液压伺服系统具有快速性好、伺服精度高、体积小等优点,因而在航空、机床、船舶、能源等各行业中获得广泛应用。下面就它的工作原理、类型、动态和静态特性分析以及使用给予介绍。

\section{液压伺服系统的工作原理与类型}
根据不同的应用场合,液压伺服系统可以有各种各样的组成形式。按照输入信号的形式不同,大致可以分为两大类:机械液压式伺服系统和电气液压式伺服系统(简称机液伺服系统和电液伺服系统)。

\subsection{机械液压式伺服系统}

这种系统是通过机械传动方式,将机械运动形式的信号输入到系统中去操纵有关的液压控制元件动作,来控制液压执行元件使其跟随输入信号而动作。这类伺服系统几乎都是专门用来进行位置控制的,即控制液压执行元件运动的位置。下面通过几个具体的实例来说明其工作原理。

图9-1所示为一种车床上的液压仿形刀架的工作原理图。整个仿形刀架安装在车床纵拖板上,它是由液压缸2、随动阀3、刀架1以及恒压油源所组成。刀架随纵拖板沿车床床身导轨做等速纵向运动$s_\text{0}$时,仿形销5在弹簧7的作用下,压在固定不动的样板6的工作面上,并在其上滑动。在样板的作用下,仿形销绕固定在液压缸上的支点4做上下摆动$x_\text{r}(t)$。通过杠杆将这个运动传给随动阀阀芯3,使它在与液压缸固定在一起的随动阀阀套中前、后移动$x_\upsilon$,图中可见,随动阀阀芯利用中间的两个凸肩棱边与阀套相对应的两个凹槽校边组成两个节流口 $\delta_\text{1}$和$\delta_\text{2}$。当阀芯在阀套内移动时,将改变这两个节流口的通流截面尺寸。刀架1除了随纵拖板一起获得纵向送进运动$s_\text{0}$以外,还由伺服液压缸 2驱动获得前、后方向的仿形送进运动$x_\text{c}(t)$。液压缸有效面积小的一腔直接与恒压油源相连,其压力不变,即$p_\text{1}=p_\text{p}$。压力油经节流口$\delta_\text{1}$流到液压缸有效面积大的另一腔,由于节流口$\delta_\text{1}$的作用,其压力由$p_\text{p}$下降为$p_\text{2}$。压力油并经节流口$\delta_\text{2}$流回油箱,其压力由 $p_\text{2}$下降为大气压。显然,液压缸大腔的压力 $p_\text{2}$取决于节流口$\delta_\text{1}$和$\delta_\text{2}$的大小。当仿形销5沿样板上平行于纵向方向的直线段滑动时,仿形销将无摆动,即$x_\text{r}(t)=0$,随动阀阀芯3也无前后移动,此时阀芯3处于中间平衡位置。在这个位置上节
