由于液压技术的发展,当前国内外许多人认为,就目前材质情况和生产水平,取液压系统压力为$350\times 10^5{\rm Pa}$左右为最经济,并有资料论证低压系统的价格比高压系统的价格高$0.5\sim 2$倍。为此,国内液压行业正在研制高压系列的泵、阀,以供不同压力的液压系统使用。

关于组合机床液压系统的工作压力,一般为$(30\sim 50)\times 10^5{\rm Pa}$(参照表8-4)。本例初选液压缸工作压力$p_{1}=44\times 10^5{\rm Pa}$。为防止钻通孔时动力滑台发生前冲,液压缸回油腔应有背压,背压$p_{2}=6\times 10^5{\rm Pa}$。假定快进、快退回油压力损失$\Delta p_{2}=5\times 10^5{\rm Pa}$。

\subsection{计算液压缸尺寸}
(1)按最大负载初定液压缸的结构尺寸。计算液压缸的有效面积时,还要考虑往返行程的速比$\lambda_{\rm v}$的要求,活塞杆受拉或受压的情况以及背压力$p_{\rm b}$的数值(在系统方案尚未拟定,回油路结构尚未确定之前,背压力是无法估算的。这里只能参考背压力$p_{\rm b}$的经验数据暂选一个)。利用第三章的有关公式求出液压缸左右有效工作面积$A_{1}$及$A_{2}$、直径$D$和活塞杆直径$d$等的值。

(2)按液压缸最低运动速度验算其有效工作面积。有效工作面积决定于负载和速度两个因素。用负载和初选压力计算出来的有效工作面积,还须按下式进行检验:
\begin{equation}
A\geqslant \frac{Q_{\rm min}}{v_{\rm min}}
\end{equation}

\noindent 式中\quad
\begin{tabular}[t]{ll}
$v_{\rm min}$ &——液压缸的最低工进速度;\\
$A$ &——液压缸的有效工作面积;\\
$Q_{\rm min}$ &——液压缸最小的稳定流量。
\end{tabular}

在节流调速系统中,$Q_{\rm min}$决定于调速阀或节流阀的最小稳定流量,其值可在产品样本性能表上查到。在容积调速系统中,液压缸的最小稳定流量决定于变量泵的最小稳定流量。

如果有效工作面积$A$不能满足式(8-6),则应适当加大液压缸直径。将确定的液压缸直径和活塞杆直径圆整化为规定的标准值(见表8-5和表8-6),以便采用标准的密封件和标准的工艺装备。

本例由于取液压缸前、后腔有效面积之比为$2:1$,因此得液压缸无杆腔有效工作面积$A_{1}$为
\begin{equation*}
A_{1}=\frac{F_0}{(p_1-\frac{1}{2}p_2)}=\frac{15556}{(44-\frac{6}{2})\times 10^5}\approx 37.9 \times 10^{-4}m^2
\end{equation*}

取$$A_1=38\times 10^{-4}m^2$$

故液压缸内径$D$为
\begin{equation*}
D=\sqrt {\frac{4A_1}{\pi}}=\sqrt{\frac{4\times 38\times 10^{-4}}{\pi}}\approx 6.96\times 10^{-2}m
\end{equation*}
按表8-5取标准值 $$D=7\times 10^{-2}m$$

按式(3-10)计算活塞杆直径
\begin{equation*}
d=0.7D\approx 5\times 10^{-2}m(\text{标准直径})
\end{equation*}

液压缸尺寸取标准值之后的有效工作面积:

无杆腔面积 
\begin{equation*}
A_1=\frac{\pi D^2}{4}=\frac{3.14\times (7\times 10^{-2})^2}{4}\approx 38.5\times 10^{-4}m^2
\end{equation*}
有杆腔面积
\begin{equation*}
A_2=\frac {\pi}{4}(D^2-d^2)=\frac{3.14}{4}(7^2-5^2)\times 10^{-4}\approx 18.8\times 10^{-4}m^2
\end{equation*}
活塞缸面积 
\begin{equation*}
A_3=A_1-A_2=19.7\times 10^{-4}m^2
\end{equation*}
\subsection{计算液压缸在工作循环中各阶段所需的压力、流量和功率}
根据表8-7计算,表中 $F_0$为液压缸的驱动力,由表8-2查得。
\subsection{绘制液压缸的工况图}
根据表8-7,即可绘制液压缸的流量图、压力图和功率图,如图8-5所示。
工况图的作用是:
(1)通过工况图找出最大压力、最大流量点和最大功率点,分析各工作阶段中压力、流量变化的规律,作为选择液压泵和控制阀的依据。

(2)验算各工作阶段所确定参数的合理性。例如,当功率图上各阶段的功率相差太大时,可在工艺情况允许的条件下,调整有关阶段的速度,以减小系统需用的功率。当系统有多个液压缸工作时,应把各液压缸的功率图按循环要求叠加后进行分析,若最大功率点相互重合,功率分布很不均衡,则同样应在工艺条件允许情况下,适当调整参数,避开或削减功率“高峰”,增加功率利用的合理性,以提高系统的效率。

(3)通过对工况图的分析,可以合理地选择系统主要回路、油源形式和油路循环形式等,如果在一个循环内流量变化很大,则不适宜采用单定量泵,也不宜采用蓄能器,而适宜采用“大小泵”的双泵供油回路或限压式变量泵的供油回路。

以上分析、计算和调整,有利于拟定出较为合理、完善的液压系统方案。
\section{拟定液压系统原理图}
\subsection{调整方式的选择}
钻、镗组合机床工作时,要求低速运动平稳性好,速度负载特性好。由图8-5可知,液压缸快速和工进时功率都较小,负载变化也较小,因此采用调速阀的进油节流调速回路。为防止工作负载突然消失(钻通孔)引起前冲现象,在回油路上加背压阀。
\subsection{快速回路和速度换接方式的选择}
本例已选用差动型液压缸$(A_1=2A_2)$实现“快、慢、快”的回路,即采用快进和快退速度相等的差动回路作为快速回路。由于快进转为工进时有平稳性要求,故决定采用行程阀来实现,而工进转快退则利用压力继电器来实现。

综上所述,本系统的主要液压回路为进油节流调速回路与差动回路。为实现这两种回路的要求,可以有多种不同形式的进油节流调速回路与差动回路的组合。下面对图8-6所示的(a),(b),(c)和(d)四种回路进行分析比较。

图8-6中,(a)回路是利用两个二位三通电磁换向阀代替(b)回路和(c)回路中的一个三位五通电磁换向阀。二位换向阀通道简单,压力损失小,而且(a)回路比(b)回路和(c)回路少用一个液压顺序阀。(b)回路与(c)回路两种换向阀的中间机能,一为V形,一为O形。前者中位时液压缸两腔可以卸荷,换向冲击较小。后者换向阀在中间位置时液压缸前后两腔封闭,应用于立式机床较合适。本系统换向精度要求不高,为减小换向冲击,可选(b)回路。(d)回路也有应用,因O形三位四通电磁换向阀在液压系统中应用较为普遍,一般工厂常有备件,故也有用(d)回路方案,其性能与(c)回路方案基本相同。

综合上述,(a)回路是利用两个二位三通电磁换向阀代替(b)回路中的三位五通电磁换向阀和液控顺序阀,从回路性能上看两者是完全相似的,而且价格也接近,因此,两种方案均可采用。但目前在设计“进油节流一次进给液压系统”时,习惯采用(b)回路作为调速回路,为利用标准图纸采用(b)回路较为方便。
\subsection{油源的选择}
由图8-5清楚地看出,其系统特点是快速时低压大流量时间短,工进时高压小流量时间长。显然选用单定量液压泵效率低,系统发热量大,故应采用双联叶片泵或限压式变量泵,两者比较见表8-8。本机床要求系统压力平稳,工作可靠,为此采用双联叶片泵。
