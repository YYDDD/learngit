

(ii)机动式:如图4-10所示,挡块移动,压下阀芯,使油路接通。

(iii)电磁式:如图4-11所示,线圈通电,衔铁被吸动,推动顶杆使滑阀阀芯移动接通油路。断电后,阀芯在弹簧作用下复位,使油路换向。

(iv)液动式:如图 4-12 所示,控制油从 K 口通入,推动阀芯移动,使油路接通。断开控制油路,阀芯在弹簧作用下复位,油路中断。

以上四种类型都是用二位二通换向阀为例说明这四种操纵、定位方式的原理,这四种方式同样应用在二位、三位的三通、四通和五通换向阀上。

(V)电-液式:如图4-13所示,电-液式是个组合换向阀,利用电磁阀作先导阀去控制液动阀改变主油路的方向。

电磁铁线圈1和3都不通电时,电磁阀阀芯2处在中位,液动阀阀芯6两端都接油箱,也处在中位。电磁铁线圈1通电时,阀芯2移向右位,压力油经单向阀7接通主阀芯6的左端,其右端的油则经节流阀4 和电磁阀而与油箱相通,于是主阀芯在压力油作用下向右移动,移动速度的快慢由节流阀4的开口大小决定。同理,当电磁铁线圈3通电,阀芯2移向左位时,主阀芯6也移向左位,其移动速度的快慢由节流阀8的开口大小决定。

阀芯定位除了手动式定位采用如钢珠、弹簧外,其余均采取不解除操纵动力方式。有些阀芯定位还采用双重定位,即除了不解除操纵动力以外,还附加钢珠、弹簧定位。此种定位是确保在换向前不因操纵动力的解除而变位。

操纵方式的选择视具体情况而定。手动式用于小流量、低压以及便于随时变换的场合。机动式常用于行程控制,要求换向性能好,布置方便的场合。电磁式常用于远距离或自动控制系统中。液动式则多用于阀芯行程长、高压、大流量的液压系统。而电-液式则用于要求换向平稳无冲击,高压、大流量的液压系统中。

由于电磁式使用方便,在机床液压系统中使用较普遍。电磁式换向阀有直流和交流供电两种,二者都有国产产品。交流采用市电(220 V,50 Hz),启动力大,换向时间短,约0.01$\sim$0.07 s内完成一次换向,但它换向冲击和振动大,衔铁吸不上时易烧坏线圈,可靠性差,体积大,市电对人身也不安全。直流须采用专门的整流装置,但它工作可靠,不易烧坏线圈,体积小,寿命长,换向冲击小,对人身安全。

(2)性能分析。

1)中位机能。三位换向阀,当阀芯处于中间位置时,阀的通道内部可根据使用的需要有各式各样的连通,常用的连通类型如表4-1所示,这种中间位置通道内部连通类型称为三位换向阀的中位机能。

 O型中位机能的特点是:油口全部被封住,油液不流动,执行元件可在任意位置被锁住,不能应用手动机构。由于液压缸内充满着油,从静止到启动较平稳,但换向时冲击较大。

 H型中位机能的特点是油口全部连通,液压泵卸荷,液压缸处于浮动状态,可用手动机构。由于回油口通油箱,当停车时,执行元件中的油流回油箱,再次启动时,易产生冲击。由于油口全通,换向时比O型平稳,但冲出量较大,换向精度较低。当用于单出杆液压缸时,中位机能不能使液压缸在任意位置停止。

M型中位机能的特点是压力油口P与回油口O连通,其余封闭,液压泵卸荷,不能使用手动机构,液压缸可在任意位置停止,启动平稳,换向时有冲击现象。

其他类型的中位机能的特点,读者可自行分析。

2)液压卡紧现象。滑阀式换向阀的阀芯从理论上讲,只要克服阀芯与阀体的摩擦力以及恢复弹簧的弹力就可移动。然而在实际上,由于阀芯几何形状的偏差以及阀芯与阀体的不同心,在中、高压控制油路中,阀芯停止一段时间后或换向时,阀芯在操纵动力作用下不移动,或操纵动力解除后,恢复弹簧不能使阀芯复位,这种现象叫做液压卡紧现象。

阀芯的卡紧现象是由于阀芯所受径向力不平衡所造成的。它会使操纵费力,液压动作失灵,故必须尽可能地排除产生卡紧的因素。

图4-14 所示为阀芯径向力不平衡的几种情况。

图4-14(a)所示阀芯是理想的圆柱形,当它与阀体产生一个平行轴线的偏心 e时,由于阀芯沿轴线间隙均匀,根据沿间隙压力分布规律可知,阀芯上、下沿轴线的压力是对应相等的,不会因阀芯的偏心而产生径向力的不平衡。

图4-14(b)所示是阀芯加工具有锥度,且大头在高压油一边(倒锥),当阀芯与阀体产生一平行于轴线的偏心e时,由于上部间隙小,沿轴线方向压力下降梯度大,而下部间隙大,沿轴线方向压力下降梯度小,在阀芯对应处产生径向力的不平衡。由图中可看出,这种径向不平衡力,将使阀芯的较小间隙的一侧进一步缩小而趋于卡死。

图4-14(c)所示为阀芯加工具有锥度,且小头在高压油一边(顺锥),当阀芯与阀体轴线不重合产生一平行于轴线偏心 e时,由于大头在低压油一边,上边间隙小,下边间隙大,沿轴线方向的阻力上边比下边的要大,因而沿轴线的压力下降梯度,上边就比下边的要小,如图所示。在此情况下,径向不平衡力使偏心减小,不会产生卡紧现象。
