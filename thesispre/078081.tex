活塞杆密封处设置防尘圈。防尘圈应放在朝向活塞杆外伸的那一端,如图3-11(d)所示。
\begin{figure}[htbp]
\begin{minipage}[t]{0.45\linewidth}
\centering
\includegraphics[height=4cm,width=4cm]{logo.pdf}
\caption{活塞杆和端盖处的密封装置}
\end{minipage}
\begin{minipage}[t]{0.45\linewidth}
\centering
\includegraphics[height=4cm,width=4cm]{logo.pdf}
\caption{回转轴密封圈}
\end{minipage}
\end{figure}

在液压泵、液压马达和摆动缸的转轴上,通常采用回转轴密封圈,其形状如图3-12所示。它由耐油橡胶压制而成,内部有一个断面为直角的金属骨架1支撑着。内唇由一根螺旋弹簧2收紧在轴上,防止油液沿轴向泄漏到壳体外面去。它的工作压力一般不超过$1\times10^5$Pa,最大允许速度为4~8m/s,且应在有润滑的情况下进行工作。
\subsection{缓冲装置}
    当液压缸所驱动的工作部件质量较大,移动速度较快时,由于具有的动量大,致使在行程终了时,活塞与端盖发生撞击,造成液压冲击和噪声,甚至严重影响工作精度和发生破坏性事故,因此在大型、高速或要求较高的液压缸中往往须设置有缓冲装置。尽管液压缸中的缓冲装置结构形式很多,但它的工作原理都是相同的。当活塞接近端盖时,增大液压缸回油阻力,使缓冲油腔内产生足够的缓冲压力,使活塞减速,从而防止活塞撞击端盖。
\begin{figure}[htbp]
\centering
\ifOpenSource
\includegraphics{cover.png}
\else
\includegraphics{logo.pdf}
\fi
\caption{液压缸的缓冲装置}
\label{logo}
\end{figure}
    液压缸上常用的缓冲装置如图3-13所示。图3-13(a)为间隙缓冲装置,当活塞移近端盖时,活塞上的凸台进入端盖的凹腔,将封闭在回油腔中的油液从凸台和凹腔之间的环状间隙$\delta$中挤压出去,吸收了能量形成缓冲压力,从而使活塞减慢了移动速度。这种缓冲装置结构简单,但缓冲压力不可调节,且实现减速行程较长,适用于移动部件惯性不大,移动速度不高的场合。图3-13(b)所示为可调节流缓冲装置,它不但有凸台和凹腔等结构,而且在端盖中还装有针形节流阀1和单向阀2。当活塞移近端盖时,凸台进入凹腔。由于凸台和凹腔之间有O形密封圈挡油,所以回油腔中的油液只能经针形节流阀流出。由于回油阻力增大,因而使活塞受到制动作用。这种缓冲装置可以根据负载情况调整节流阀开口的大小,改变吸收能量的大小,因此适用范围较广。图3-5中缓冲装置就属此类。图3-13(c)所示为可变节流缓冲装置,它在活塞上开有横断面为三角形的轴向斜槽1。当活塞移近液压缸端盖时,活塞与端盖间的油液须经轴向三角槽流出,而使活塞受到制动作用。从图中可看出,它在实现缓冲过程中能自动改变其节流口大小(随着活塞移动速度的降低而相应关小节流口),因而使缓冲作用均匀,冲击压力小,制动位置精度高。
\subsection{排气装置}
    当液压系统长时间停止工作,系统中的油液由于本身重量的作用和其他原因而流出时,易使空气吸入系统,如果液压缸中有空气或油中混入空气,都会使液压缸运动不平稳,因此一般在机床工作前应使系统中的空气排出,为此可在液压缸的最高部位(那里往往是空气聚积的地方)设置排气装置。排气装置通常有两种:一种是在液压缸的最高部位处开排气孔(见图3-14(a)),并用管道连接排气阀进行排气,当系统工作时该阀应关闭;另一种是在液压缸的最高部位处装排气塞(见图3- 14(b),(c))。
\begin{figure}[htbp]
\centering
\ifOpenSource
\includegraphics{cover.png}
\else
\includegraphics{logo.pdf}
\fi
\caption{排气装置}
\label{logo}
\end{figure}

\section{液压缸结构设计中应注意的问题}
    液压缸的设计是整个液压系统设计的重要内容之一,由于液压缸是液压传动的执行元件,它和机床工作机构有直接的联系,对于不同的机床及其工作机构,液压缸具有不同的用途和工作要求。因此在设计液压缸之前,应作好充分的调查研究,收集必要的原始资料和设计依据,包括机床用途、性能和工作条件;工作机构的类型、结构特点、负载情况,行程大小和动作要求;液压缸所选定的工作压力和流量;同类型机床液压缸的技术资料和使用情况以及有关国家标准和技术规范等。

    不同的液压缸有不同的设计内容和要求,一般在设计液压缸的结构时应注意下列几个问题:

(1)在保证满足设计要求的前提下,尽量使液压缸的结构简单紧凑,尺寸小,尽量采用标准形式和标准件,使设计、制造容易,装配、调整、维护方便。

(2)应尽量使活塞杆在受拉力的情况下工作,以免产生纵向弯曲。为此,在双出杆活塞式液压缸中,活塞杆与支架连接处的螺栓固紧螺母应安装在支架外侧。对单出杆活塞式液压缸来讲,应尽量使活塞杆在受拉状态下承受最大负载。

(3)当确定液压缸在机床上的固定形式时,必须考虑缸体受热后的伸长问题。为此,缸体只应在一端用定位销固定,而让另一端能自由伸缩。双出杆液压缸的活塞杆与支架之间不能采用刚性连接。

(4)当液压缸很长时,应防止活塞杆由于自重产生过大的下垂而使局部磨损加剧。

(5)应尽量避免用软管连接。

(6)液压缸结构设计完后,应对液压缸的强度稳定性进行验算。有关验算校核的方法详见材料力学的有关公式。
\section*{思考和习题}
\begin{enumerate}[\hspace{2em}3-1]
 \item 常用液压缸有哪些类型?结构上各有何特点?各用于什么场合?
 \item 缸体组件、活塞组件的连接方式有哪几种?各用于什么场合?
 \item 活塞与缸体,活塞杆与端盖之间的密封方式有哪几种?各用于什么场合?
 \item 液压缸的缓冲方式有哪几种?各有何特点?
 \item 设计液压缸的结构时应注意哪些问题?
 \item 两个单出杆液压缸,其结构尺寸如图3-15所示,(a)为活塞杆固定,左侧进油压力为$p_1$,回油压力为$p_2$;(b)为液压缸固定,差动连接,进油压力为$p_1$。试问:
\begin{figure}[htbp]
\centering
\ifOpenSource
\includegraphics{cover.png}
\else
\includegraphics{logo.pdf}
\fi
\caption{题3-6图}
\label{logo}
\end{figure}
  \begin{enumerate}[\hspace{2em}(1)]
   \item 输入油量Q相同,两者运动速度是否一样?
   \item 两者运动方向怎样?
   \item 两缸能承受的最大负载$F_a$和$F_b$各为多少?
  \end{enumerate}
 \item 设计差动连接液压缸,要求快进速度($v_{\text {快进}}$)为快退速度($v_{\text {快退}}$)的2倍,则缸筒内径D是活塞杆直径d的几倍?
 \item 今需设计组合机床动力头驱动液压缸,其快速趋近、工作进给和快速退回的油路分别在图3-16(a),(b),(c)中示出,现采用限压式变量泵供油,其最大流量$Q_M$=30L/min。要求$v_{\text {快进}}$=8m/min;$v_{\text {工作}}$=1m/min。试求液压缸内径D、活塞杆直径d以及工作进给时变量泵的流量Q。
 \item 如图3-17所示,两个相同的液压缸串联起来,它们的无杆腔和有杆腔的有效面积分别为$A_1=100cm^2$,$A_2=80cm^2$,两缸的负载F相等,输入的压力$p=9\times10^5$Pa,流量Q=12L/min,试求:
  \begin{enumerate}[\hspace{2em}(1)]
   \item 可承受的负载F(N);
   \item 两缸活塞运动速度$v_1$和$v_2$(m/min)。
  \end{enumerate}
\end{enumerate}
\begin{figure}[htbp]
\centering
\ifOpenSource
\includegraphics{cover.png}
\else
\includegraphics{logo.pdf}
\fi
\caption{题3-8图}
\label{logo}
\end{figure}
\begin{figure}[htbp]
\centering
\ifOpenSource
\includegraphics{cover.png}
\else
\includegraphics{logo.pdf}
\fi
\caption{题3-9图}
\label{logo}
\end{figure}


