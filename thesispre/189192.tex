

    流口$\delta_1$和$\delta_2$所决定的液压缸大腔的压力$p_2$与液压缸小缸中的恒压力$p_1$对液压缸的总作用力是相对平衡的,于是液压缸静止不动,刀架将无前后方向的仿形送进运动,即$x_{\rm c}$($t$)=0,只有纵向送进运动$s_0$,所以车削出圆柱表面来。当仿形销在样板作用下绕支点向上摆动了$x_{\rm r}$($t$)距离时,通过杠杆使随动阀阀芯相对阀套向上移动,其结果是使节流口$\delta_1$开大,$\delta_2$关小。由液压原理可知,液压缸大缸的压力$p_2$加大,打破原来的平衡状态,则液压缸在油压的作用下连同随动阀阀套和杠杆支点4一起向上运动,使刀架获得向上的仿形送进运动$x_{\rm c}$($t$)。液压缸向上运动的结果将使$\delta_1$重新关小,而$\delta_2$重新开大。一旦液压缸运动到使$\delta_1$和$\delta_2$恢复原来尺寸,使$p_2$回到原来平衡状态的数值,液压缸两缸总作用力重新回到平衡时,运动也终止。不难看出,此时液压缸向上移动的距离$x_{\rm c}$($t$)将与仿形销的位移距离$x_{\rm r}$($t$)一样。如果仿形销在样板的作用下再继续向上移动,则将再次打破平衡,重复上述过程,使刀架再跟随向上移动,而刀架运动的结果又促使系统再回到平衡。只要仿形销连续移动,则将连续不断地重复平衡—不平衡—平衡的变化过程,于是刀架就获得与仿形销移动一致的连续向上的仿形送进运动$x_{\rm c}$($t$),因而加工出与样板一致的工件。

    当仿形销在样板作用下向下运动时,整个过程与上述相似,将使$\delta_1$减小,$\delta_2$加大,因此使$p_2$下降,打破了液压缸上、下作用力的平衡,于是液压缸连同刀架将向下运动,直至$\delta_1$和$\delta_2$重新恢复原来尺寸,系统回到平衡状态为止。由上述工作原理可以看出:

(1)液压伺服系统是一个具有负反馈的闭环自动控制系统,其框图如图9-2所示。正是系统的输入信号$x_{\rm r}$($t$)(仿形销的机械运动)与输出信号$x_{\rm c}$($t$)(刀架的仿形送进运动)的不一致,即出现了位置误差$e$($t$)=$x_{\rm r}$($t$)—$x_{\rm c}$($t$),将引起随动阀阀芯相对阀套产生偏移$x_{\rm v}$($t$),改变了节流口$\delta_1$和$\delta_2$的尺寸,改变了进入液压缸的压力油的压力和流量,从而产生液压缸的运动$x_{\rm c}$($t$),且运动到输入与输出信号之间的误差$e$($t$)消除为止。

(2)液压伺服系统是一个功率放大装置,推动仿形销的力很小,一般不超过5 $\thicksim$ 10N,而液压缸上产生的力很大,达几千牛顿乃至几万牛顿。系统中作为功率放大的关键环节是随动阀,它根据输入的微弱机械运动信号的大小,输出相应的具有很大功率的压力油(液压信号:$p_1$,$Q_1$)去驱动液压缸。所以随动阀又称为液压放大器。

(3)上述液压伺服系统主要是依靠液压放大器上的两个节流口$\delta_1$和$\delta_2$的通流截面积的改变,即液阻的改变来控制液压缸的运动。所以,这样的系统实质上是一个自动的节流调速系统,因此具有节流调速的基本特点:系统结构简单,工作可靠,但效率很低,不易用于大功率的地方。

    必须说明,在车床仿形刀架上加工图9-3所示零件时,车刀沿工件表面的切向送进运动$s$实际上是由纵向送进$s_0$与仿形送进运动$x_c$合成的。在加工中,$s_0$一般是一个等速运动,因此为了获得垂直于车床主轴的合成送进运动,以便加工轴类的端面,刀架必须相对车床主轴方向成$45^{\circ}$ $\thicksim$ $60^{\circ}$。

    在上述例子中,系统的输入信号(仿形销的移动)$x_r$与输出信号(液压缸和刀架的移动)$x_{\rm c}$都是直线运动。图9-4所示为输入与输出信号均为旋转运动的机液伺服系统 —— 液压转矩放大器的原理图。小功率的伺服电机1产生很小的转矩即可通过齿轮副2带动随动阀阀芯3转动。阀芯3的右端有反馈丝杠4与螺母5相配合,而螺母5则固定在液压马达6的输出轴上。在随动阀上,阀芯与阀套的棱边组成的四个节流口$\delta_1$,$\delta_2$,$\delta_3$及$\delta_4$以及它们间的油路连接,利用这四个节流口分别控制液压马达两腔的压力。当处于图示中间平衡位置时,四个节流口完全一致,因而油马达两腔的压力相同,液压马达静止不动。当伺服电机1转动一角度$\theta_{\rm r}$($t$) 时,使随动阀阀芯3转过$\beta $($t$)角度($\beta $=$i\theta $,$i$为齿轮副2的传动比),在丝杠4和螺母5的作用下,将使阀芯从原来中位移动一个距离,改变了四个节流口的状态,使液压马达两腔压力平衡被打破,在油压的作用下,液压马达轴旋转,输出旋转运动$\theta_{\rm c}$($t$),且输出很大的转矩来驱动负载。液压马达轴的旋转又联动螺母5一起转动,螺母旋转而丝杠不旋转则会使丝杠移动,因此通过丝杠4使随动阀阀芯反方向移动。当液压马达轴的转角$\theta_{\rm c}$($t$)也达到$\beta$($t$)角时,随动阀阀芯反向移动到原来的中位上,使四个节流口重新恢复一致,于是液压马达两腔压力恢复平衡,转动停止。当伺服电机连续转动时,液压马达也将跟随连续转动,并且转过的角位移与伺服电机的角位移成比例,即$\theta_{\rm c}$($t$)=$i$$\theta_{\rm r}$($t$)。我国生产的DMY型电液脉冲马达就属于此类转矩放大器,它的伺服电机为步进电机,它由伺服步进电机、随动阀和液压马达组成一单独部件供某些数控机床送进系统使用。

\subsection{二、电气液压式伺服系统}

    在伺服系统中,用电气信号控制就有传递快,线路连接方便,适用于远距离控制,易于测量、比较和处理等优点。用液压能作为动力就有输出力(或力矩)大,惯性小,响应快等优点。因此两者结合而成的电液伺服系统是一种控制灵活、精度高、快速性好、输出功率大的控制系统。

    图9-5所示是一个采用电液伺服阀控制的液压缸驱动的电液伺服系统的组成图。系统的液压执行元件——液压缸,根据输入系统的电气信号而动作,从而驱动负载输出相应的物理量,即系统的输出信号(如位移、速度、力等),这个输出信号经电气测量反馈装置测得并回输到系统输入端与输入信号相比较,如不一致,将产生反映二者误差大小的电压信号,即误差信号,该信号经过伺服放大器放大成具有一定功率输出的电流信号后被送入电液伺服阀。在电液伺服阀内部,首先把输入的电流信号通过电气—机械转换装置按比例地变换成类似上述各例中的随动阀阀芯的机械位移,从而改变了相应的节流口状态,输出具有一定压力和流量的压力油(即输出具有足够大的液压功率的液压信号)去驱动液压执行元件及负载,执行元件运动到输入信号与反馈信号完全一致,误差信号消失为止。这样的电液伺服系统广泛用于位置控制、速度控制和施力控制等。根据不同的输出信号和使用要求,反馈测量装置可以是电位器、旋转变压器、测速发电机和力传感器等。由上述可知,电液伺服阀在系统中是完成电气-液压信号转换和最后的功率放大的关键环节,关于它的结构和工作原理将在第9-4节中讲述。

\subsection{节流控制与容积控制液压伺服系统}

    前面所举的液压伺服系统的基本控制方式都是利用液压放大器(随动阀)中的几个节流口的通流截面积的改变来控制输给执行元件的压力油的压力和流量,从而达到控制执行机构运动的目的,所以它们都属于节流控制的液压伺服系统。这样的系统和节流调速一样效率很低。在大功率、大流量的系统中将造成很大的能量损耗,因此要求采用效率高的类似容积调速那样的容积控制的液压伺服系统。图9-6所示为这种系统的方框图。输入的控制信号首先通过前述的小功率节流控制的液压伺服系统来控制主液压泵的变量机构,如控制轴向柱塞泵的斜盘倾斜角度、变量叶片泵的定子偏心距等,从而改变主液压泵输出的液压功率(输出的压力油的流量和压力)来控制系统执行元件的运动(系统的输出信号)。这个输出信号再经反馈装置回输到输入端与输入信号比较,实现闭环控制。

    综上所述,液压伺服系统的分类情况可概括为


\section{液压放大器}


    液压伺服系统的核心是液压放大器,如上节各例中的随动阀。它根据输入的微弱机械位移信号来控制压力油的流量和分配,从而控制执行元件的运动。因此,液压放大器是一种具有功率放大作用的、起到机械与液压信号转换作用的液压控制元件。根据控制方式的不同,液压放大器分为滑阀式、喷嘴挡板式和射流管阀式三种。

\subsection{一、滑阀式液压放大器}

	图9-1和图9-4所示各例中的液压放大器都属于此类。它们都是利用圆柱滑阀阀芯上的凸肩棱边与阀套上对应的凹槽棱边组成控制节流口,当阀芯相对阀套移动时,改变这些节流口的通流截面积来控制输出的压力油的流量与压力。根据组成节流口的数目不同,滑阀式液压放大器又分为四边控制、双边控制和单边控制滑阀式液压放大器。

\subsubsection{1. 四边控制滑阀式液压放大器}

	图9-4所示实例中的随动阀就是此类放大器。他是利用滑阀阀芯与阀套组成四个节流口来进行控制,其工作原理不在赘述。为了便于进一步理解这四个节流口的控制作用,我们用图9-7(c)所示四臂电桥等效电路来描绘。每一个节流口相当于电桥的一个臂上的电阻,而液压缸作为负载被连在电桥中间,油流量相当于电路中的电流,油压则相当于电压。当阀芯移动时,使各节流口开大或关小,即液阻(电阻)减小或加大,从等效电路中不难理解,这将改变加在负载上的压力差和流过负载的流量的大小和方向,从而达到控制负载运动的目的。

	根据节流口在中间平衡位置时不同的初始开口量,又有正开口、零开口和负开口三种滑阀,如图9-8所示。当阀芯移动时,不同的初始开口量将有不同的流量输出,图9-8(d)所示为
