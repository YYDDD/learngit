




甚至会因转速太高(飞车)而造成事故,故不宜采用这种换向方式。

(4)液压泵和液压马达随负载的增加而容积效率降低,使泄露增加,故这种回路也存在随负载增加而速度下降的现象。

\subsubsection{变量泵和变量液压马达组成的调速回路}

如图6-18所示,双向变量泵1不仅可以改变输出流量,而且可以改变输油方向,以实现变量马达的调速与换向。由于双向供油,故在辅助泵3的油路中增加了单向阀6和8,在安全阀4的油路中增加了单向阀7和9。

这种调速回路是上述两种调速回路的组合,由于泵和马达的排量均为可变,故扩大了调速范围,并扩大了液压马达转矩与功率输出特性的选择余地。其输出特性曲线如图6-19所示。

\begin{figure}
\centering
\ifOpenSource
\includegraphics{logo.pdf}
\else
\includegraphics{fig0618.pdf}%例如 fig0506.pdf
\fi
\caption{变量泵-变量马达式容积调速回路}
\label{fig:fig0618}%例如fig:fig0506
\end{figure}

\begin{figure}
\centering
\ifOpenSource
\includegraphics{logo.png}
\else
\includegraphics{fig0619.pdf}%例如 fig0506.pdf
\fi
\caption{变量泵-变量马达式容积调速回路的输出特性}
\label{fig:fig0619}%例如fig:fig0506
\end{figure}

一般工作部件都在低速时要求有较大的转矩,因此,这种系统在低速范围内调速时,先将液压马达的排量调为最大(使马达能获得最大的输出转矩),然后改变泵的输油量,当变量泵的排量由小变大,直至达到最大输油量时,液压马达转速亦随之升高,输出功率随之线性增加;若要进一步加大液压马达转速,则可将变量马达的排量由大变小,此时输出转矩随之降低,而泵则处于最大功率输出状态不变,故液压马达亦处于恒功率输出状态。

这种回路和上两种调速回路相同,亦有随负载增大而泄露增加、转速下降的特性。

\subsection{容积节流调速回路}
这种调速回路采用变量泵和节流阀(或调速阀)相配合进行调速,是容积式与节流式调速的联合,故称联合调速。液压泵的供油量与执行元件所需流量相适应,回路中没有溢流损失,故效率比节流调速方式高;变量泵的泄露由于压力反馈作用而得到补偿,进入执行元件的流量由调速阀控制,故速度稳定性比容积式调速好。因此在调速范围大、中等功率的机床液压系统中常采用之。

机床上常用的容积节流调速方法有限压式变量泵和调速阀的联合调速;差压式变量泵和节流阀的联合调速。

\subsubsection{限压式变量泵和调速阀式容积节流调速回路}

如图6-20(a)所示,系统由限压式变量泵供油,压力油经调速阀进入液压缸工作腔,回油经背压阀流回油箱。调节调速阀的开口大小,即可改变进入液压缸的流量Q$_1$,从而调节活塞的运动速度。设泵的流量为Q$_p$,从图可见,稳态工作时,Q$_i$=Q$_1$。可是在关小调速阀的一瞬间,Q$_1$减小,而液压泵的每转排量还未来得及改变,流量Q$_p$没有变,于是出现了Q$_p$>Q$_1$,因回路中没有溢流阀,多余油液使泵和调速阀间的油路压力升高,也即使泵的出口压力升高,从而使限压式变量泵输出流量自动减小,直至Q$_p$=Q$_1$为止。反之,开大调速阀的一瞬间,将出现Q$_p$<Q$_1$,就会使限压式变量泵出口压力降低,输出流量自动增加。调速阀在这里不仅保证进入液压缸的流量稳定,而且可使泵的供油量自动地和液压缸所需流量相适应。液压泵的流量总是和负载流量相匹配,故这种回路又称流量匹配回路。调速阀亦可装在回油路上。

\begin{figure}
  \centering
  \ifOpenSource
  \includegraphics{logo.png}
  \else
  \includegraphics{fig0620.pdf}%例如 fig0506.pdf
  \fi
  \caption{限压式变量泵-调速阀式联合调速回路(a)回路图;(b)特性曲线}
  \label{fig:fig0620}%例如fig:fig0506
  \end{figure}

图6-20(b)所示是这种回路的特性曲线。曲线1是限压式变量泵的压力-流量特性曲线。曲线2是某一开度下调速阀的压差-流量特性曲线。两条曲线的交点b是回路的工作点(此时泵油的供油压力为p$_p$,流量为Q$_1$),改变调速阀的开口度,使曲线2上下移动,回路的工作状态便相应改变。为了保证调速阀的正常工作(调速阀中的减压阀具有压力补偿机能,当负载变化时,通过调速阀的流量不变)所需的最小压力降$\Delta$p$_t$$\text{min}$(一般为5$\times$10$^5$Pa左右),限压式变量泵的供油压力应调节为

\begin{equation}
  p_p\geq p_1+\Delta p_t \text{min}
\end{equation}

\noindent 系统最大工作压力应为

\begin{equation}
  p_1 \text{max} \leq p_p-\Delta p_t \text{min}
\end{equation}

\noindent 同时,应使p$_0$大于快速移动时所需压力,此时,便可保证当负载变化时,执行元件工作速度不随负载而变。如采用“死档铁停留”发信号时,为保证压力继电器可靠地工作,则泵的供油压力还应调得更高些(使泵按曲线3工作)。当然,泵的供油压力也不能调得过高,以免功耗过多,发热增加。

由于限压式变量泵一经调定后其压力-流量曲线是不变的,因此当负载F变化引起p$_1$发生变化时,调速阀的自动调节作用,使调速阀内节流阀上的压差$\Delta$p保持不变,流过此节流阀的流量Q$_1$也不变,从而使泵的输出压力p$_p$和流量Q$_p$也就不变,回路就能保持在原工作状态下工作,速度稳定性好,速度-负载特性较硬。

若不考虑泵、缸和管路的损失,回路效率为

\begin{equation}
  \eta= \frac{({p_1-{p_2 \frac{A_2}{A_1}}})Q_1}{p_pQ_1}= \frac{p_1-{p_2( \frac{A_2}{A_1})}}{p_p}
\end{equation}

\noindent 若无背压,p$_2$$\approx$0,则

\begin{equation}
  \eta= \frac{p_1}{p_p}=1-\frac{\Delta p_t}{p_p}
\end{equation}

这种回路在重载条件下工作时,效率较高,轻载下工作时效率较低,故不宜用于负载变化大,且大部分时间在小负载下工作的场合。

\subsubsection{差压式变量泵和节流阀式的容积节流调速回路}

如图6-21(a)所示,系统由差压式(或称稳流量式)变量泵3(如变量叶片泵)供油,液压泵输出流量Q$_p$全部通过节流阀4进入液压缸5,即Q$_p$=Q$_1$,没有溢流损失。泵的变量机构由定子3两侧的控制杠1和2与弹簧组成,控制杠的左腔引入泵的出口压力,亦即是节流阀前的压力p$_p$而右腔则经阻尼孔7引入节流阀出口的工作压力p$_1$。变量泵的定子相对转子的移动,改变二者之间的偏心量,达到调节泵流量Q$_p$的目的。偏心量的改变是靠控制缸的液压力之差与弹簧力的平衡来实现的,即

\begin{equation}
  p_pA_1+p_p(A-A_1)=p_1A+F_s
\end{equation}

\noindent 可得

\begin{equation}
 p_p-p_1=\Delta p_j= \frac{F_s}{A}
\end{equation}

\noindent 式中\ 
\begin{tabular} [t]{ll}
F$_s$ &——\hspace{1mm}弹簧压紧力;\\
$\Delta$p$_j$ &——\hspace{1mm}节流阀前后压差,也是液压泵控制缸左、右腔的压力差。
\end{tabular}

泵的输出流量与压差的关系如图6-21(b)中的abcd曲线所示,类似限压式变量叶片泵(见图2-23所示)。改变压差即可改变输出流量。图6-21(a)中的阻尼孔7用以增加变量泵定子移动的阻尼,避免发生振荡。8为安全阀限制工作压力p$_1$的最大值。

调节节流阀通流截面积A$_j$即可改变$\Delta$p$_j$从而调节泵的输出流量Q$_p$。从节流阀的流量公式(4-11)有:

\begin{equation}
 Q_p=Q_1=C_jA_j \Delta p_j^\varphi
\end{equation}

\noindent 式中\ 
\begin{tabular} [t]{ll}
A$_j$ &——\hspace{1mm}节流阀的开口面积。
\end{tabular}

由上式可以绘出不同开口A$_j$(A$_j$$_1$,A$_j$$_2$,A$_j$$_3$ …)的节流阀流量-压差曲线族,如图6-21(b)所示。既然节流阀的压差和流量就是变量泵的控制缸压差与泵输出流量,则节流阀的流量-压差曲线族与变量泵的流量-压差曲线的交点就是系统的工作点。从图6-21(b)可知调节节流阀的开口A$_j$即可调节流量Q$_p$(Q$_1$),达到调节液压缸运动速度的目的。节流阀开口调定后,负载p$_1$变化时,p$_p$也跟随变化,从而使其压差并没有改变(从式(6-47)亦可看出),因而流量及速度也就稳定不变。

\begin{figure}
  \centering
  \ifOpenSource
  \includegraphics{logo.png}
  \else
  \includegraphics{fig0621.pdf}%例如 fig0506.pdf
  \fi
  \caption{差压式变量泵-节流阀式联合调速回路(a)回路图;(b)特性曲线}
  \label{fig:fig0621}%例如fig:fig0506
  \end{figure}


在p$_p$跟随p$_1$而增加的瞬间,泵的泄露也有所增加,使输出流量Q$_p$也瞬间有所下降,由式(6-48)可知节流阀的压差$\Delta$p$_j$也因此而减小,通过泵的控制缸而使定子左移,偏心加大,从而使泵的输出流量Q$_p$有所回升,弥补泄露的增大,直到流量重新回到原来的大小。反之,p$_p$减小时,将逆向变化上述过程。由此可见泵的输出流量不会因负载的变化而改变,达到稳定速度的目的。

综上所述,此调速系统的速度-负载特性硬,速度稳定性好。为了保证可靠地控制变量泵定子相对转子的偏心量,压力差pj不可过小,一般须保持p$_j$$\approx$(3$\sim$4)$\times$10$^5$ Pa,即泵的控制缸的弹簧选择为F$_s$$\approx$A(3$\sim$4)$\times$10$^5$ N。

这种调速回路没有溢流损失,故回路效率较高。由于泵的供油压力随工作压力的增减而增减,故在轻载条件下工作时,其效率较高的特点尤为显著。

\section{快速运动回路}
为了缩短辅助时间提高生产率,合理利用功率,机床上的空行程一般都希望做快速运动,故机床液压系统中常常同时设置工作行程时的调速回路和空行程时的快速运动回路。两者相互联系,快速运动回路的选择必须使调速回路工作时的能量损耗尽可能小。

实现快速运动的方法一般有三种:①增加输入执行元件的流量;②减小执行元件在快速
