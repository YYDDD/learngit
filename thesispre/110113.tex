
$\rm p_1- p_2$可基本保持不变。当$\rm p_2$减小时,溢流阀阀芯$a$腔的压力亦减小,溢流阀阀芯受力平衡被破坏,向上移动,溢流阀溢流口加大,液阳减小,使液压泵出口压力$\rm p_1$相应地减小,同样使$\rm p_1-p_2$保持基本不变。溢流阀阀芯受力平衡方程式为\\ \vspace{-0.7cm}
$${\rm p_1A_y=p_2+F_s+G+F_f}\eqno(4-19)$$ \vspace{-0.9cm}

\noindent 式中 \quad
\setlength{\tabcolsep}{0.2mm}{
\begin{tabular}[t]{lll}
$\rm p_1$&——&节流阀人端压力,即液压泵供油压力;\\
$\rm p_2$&——&节流阀出端压力,即由外载荷决定的压力;\\
$\rm A_y$&——&溢流阀阀芯的大端面积,也就是阀芯肩部$b$与下端$c$的有效面积之和;\\
$\rm F_s$&——&溢流阀阀芯大端$a$腔的弹簧作用力;\\
$\rm G$&——&阀芯自重(垂直安装时考虑);\\
$\rm F_f$&——&阀芯移动时的摩擦力。
\end{tabular}}

如略去$\rm G$和$\rm F_f$的影响,式$(4-19)$可写成
 $$\rm p_1- p_2=\frac{F_s}{ A_y}\eqno(4-20)$$

式$(4-20)$表明,当溢流阀阀芯移动量较小,且弹簧的刚度又很小时,$\rm F_s$可基本维持是个常数,亦即节流阀前后压差 $\rm (p_1- p_2)$基本为一常数,这就保证了通过节流阀的流量的稳定。

安全阀2用以防止系统过载,它相当先导式溢流阀的先导部分。

调速阀与溢流节流阀都可用来调节并稳定流量,功能相似,但其使用性能不完全相同。调速阀是在保持液压泵供油压力基本不变(由溢流阀调定)情况下工作的.此压力要满足系统的最大载荷,因此消耗功率较大。而溢流节流阀的供油压力是随负载而变的。当负载小时,节流阀后的压力降低,液压泵供油压力也随着下降,这样就可减小驱动液压泵所需的功率,并减少液压系统的发热。但溢流节流阀中流过的流量是液压泵的全流量,调芯运动时的阻力较大,因此溢流阀上的弹簧一般比调速阀的硬一些,这样就加大了节流阀前后的压差波动,如考虑稳态液动力的影响,溢流节流阀入口压力的波动也影响节流阀前后压差的稳定,因此溢流节流阀的稳速性能稍差。

\section*{\centerline{4\;--\;5\quad 比例阀和逻辑阀}} \vspace{-0.7cm}

比例阀和逻辑阀的出现,扩大了液压系统的使用范围。所谓比例阀就是一种按输入的电信号连续地、按比例地控制液压系统的压力和流量的阀。在液压系统中常用的控制阀多具有开关控制的性质。它的作用是使一个液压元件接入液压系统或脱离液压系统,或者进行简单的油路切换等,而不能进行连续控制。如果要对液压系统的参数进行连续控制,则必须使用伺服阀(对于伺服阀,将在第九章介绍)。由于伺服阀的价格昂贵,维护保养要求严格,使用条件要求高,因而限制了它在一般液压系统中的广泛使用。比例阀可以对液压系统的参数进行连续、成比例的控制。而它与伺服阀相比,则结构简单,成本低,通用性好,并能简化液压系统的油路及减少元件的数量。比例阀的组成就是把普通的压力阀、流量阀和换向阀的控制部分换上比例电磁铁,用比例电磁铁的吸力来改变阀的参数以进行比例控制。根据用途和工作特点的不同,比例阀可分为比例压力阀、比例流量阀和比例方向阀等。

逻辑阀是以锥阀为基本单元,以芯子插入式为基本连接形式,配以不同的先导阀来满足各种动作要求的阀类,它实际上是一种液控单向阀,又叫嵌装式闸阀或插装式锥阀。这种受控单向阀的开启和闭合完全像一个受操纵的逻辑元件那样工作,所以又叫逻辑阀。它特别适用于高压、大流量的液压系统中。

下面分别简要地说明一下它们的工作原理。

\textbf{一、电磁比例压力阀}

图$4-49$所示为电磁比例溢流阀的结构原理图。

它是由普通先导式溢流阀和比例电磁铁组成,它的工作原理与先导式溢流阀相同。所不同的,普通溢流阀的调压多是用手调的,面电磁比例溢流阀的压力是由电磁铁产生的电磁力推动推杆,压缩弹簧作用在锥阀上,顶开锥阀的压力$p$.即是调整压力。其电磁推力的大小与通人比例电磁铁的电流成比例,因此改变电流的大小,即可调节溢流阀压力的大小。其关系式如下:

电磁力 \hspace{4.4cm}
$\rm F_D=K_1I$

弹簧压缩力 \hspace{3.7cm} $\rm F_s=pA$

\noindent 由于$\rm F_D=F_s$,所以$\rm pA=K_1I$
$$\rm p=\frac{K_1}{A}I=K_pI\eqno(4-21)$$
\noindent 式中 \quad
\setlength{\tabcolsep}{0.2mm}{
\begin{tabular}[t]{lll}
$\rm p$&——&溢流阀调整压力:;\\
$\rm K_p$&——&比例常数;\\
$\rm A$&——&锥阀在阀座上的受压面积;\\
$\rm I$&——&通入比例电磁铁中的电流大小。
\end{tabular}}

从式$(4-21)$中可以看出,若输人的电流是连续的或按-定程序变化,则比例阀所控制的压力也是与输人信号成比例的或按一定程序变化的。

图$4-50$所示为比例压力阀的$\rm p-I$特性曲线。根据式$(4-21)$,压力 P与电流I的关系应该是线性的,但由于磁性材料和运动部件的磁滞、摩擦影响,$\rm p-I$上升与下降曲线不重合。从图上可以看到,在电流上升到$\rm I_0$。时,输出压力为$p_A$,继续增大控制电流,压力将按比例增加,直到$\rm I_M$时,压力为$\rm p_M$。当控制电流减小时,压力不按原来的曲线下降,当控制电流为零时,输出压力为$\rm p_A$,而在控制电流从零到$\rm I_0$范围内,输出压力不变,出现不灵敏区。

\textbf{二、电磁比例流量阀}

图$4-51$所示为电磁比例调速阀的结构原理图,它是由普通调速阀与比例电磁铁组合而成的,是把普通调速阀的手柄换上了比例电磁铁。当外加电信号输人时,节流阀的阀芯在弹筑力与比例电磁铁的电磁力作用下保持平街,该位置对应节流阀定的开口量 $\rm x$,通过节流口的流量可按小孔流量特性方程决定,即
$$\rm Q=CA\Delta p^\varphi$$

因为减压阀保证了$\rm \Delta p$基本恒定,所以 \vspace{-0.1cm}
$$\rm Q\propto A=bx\eqno(4-22)$$

比例电磁铁的电磁力  \hspace{2.3cm} $\rm F_D=K_1I$

弹簧的作用力 \hspace{3.4cm} $\rm F_s=K_s x$

由于$\rm F_D=F_s$,所以\hspace{2.2cm} $\rm K_1I=K_s x$
$$\rm x=\frac{K_1}{K_s}I\eqno(4-23)$$

将式$\rm (4-23)$代入式$\rm (4-22)$得
$$\rm Q\propto\frac{K_1 b}{K_s}I\eqno (4-24)$$

\noindent 式中 \quad
\setlength{\tabcolsep}{0.2mm}{
\begin{tabular}[t]{lll}
$\rm K_1$&——&比例常数;\\
$\rm K_s$&——&弹簧刚度;\\
$\rm b$&——&节流口宽度;\\
$\rm x$&——&节流口开度。
\end{tabular}}

从式$(4-24)$可看出,只要改变输人电流信号的大小,就可控制调速阀的流量,其流量-电流特性曲线与图$4- 50$很相似。

\textbf{三、逻辑阀}

图$4-52$所示为逻辑阀锥阀式基本单元。它是由弹簧1、阀套2和阀芯(锥阀)3组成的。根据用途不同,逻辑阀又分为逻辑压力阀、逻辑流量阀和逻辑换向阀三种。


逻辑阀的工作原理:逻辑锥阀有两个管道连接口$A$,$B$和一个控制连接口$C$,压力油分别作用在锥阀的三个控制面$\rm A_a$,$\rm A_b$和
$\rm A_c$上。其中$A_a$而总是处在$A$口压力油的作用下,$A_b$面总是处在$B$口压力油的作用下。如果忽略锥阀的质量和阻尼的影响,作用在阀芯上的力平衡关系如下:
$$\rm F_s+F_w+p_cA_c-p_bA_b-p_aA_a=0\eqno (4-25)$$

\noindent 式中 \quad
\setlength{\tabcolsep}{0.2mm}{
\begin{tabular}[t]{lll}
$\rm F_s$&——&作用在阀芯上的弹簧力;\\
$\rm F_w$&——&阀口液流产生的稳态液动力;\\
$\rm p_c$&——&控制口$C$的压力;\\
$\rm P_b$&——&工作油口$B$的压力;\\
$\rm P_a$&——&工作油口$C$的压力;
\end{tabular}}

\noindent \quad $\rm A_a$,$\rm A_b$,$\rm A_c$分别为锥阀三个控制面的面积。

从式$(4- 25)$可以看出,锥阀的启、闭与控制压力$\rm p_c$以及
工作压力$\rm p_a$和$\rm p_b$的大小有关,同时还与弹簧力$\rm F_s$、液动力$\rm F_w$的大小有关。当锥阀开启时,油流的方向视$\rm p_a$与$\rm p_b$的具体情况而定,当$\rm p_a\textgreater p_b$时,油从$A$口流向$B$口;当$\rm p_b\textgreater p_a$时,油从$B$口流向$A$口;当锥阀关闭时,
