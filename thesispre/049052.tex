力一般都低于大气压力;同时部分压力油沿齿顶圆周缝隙由压油腔漏至吸油口,压力沿周向逐渐由高降低,致使沿齿轮径向的液压作用力不平衡,再加上齿轮啮合力的联合作用,因此在齿轮轴的轴承上受到一个很大的径向力。泵的工作压力愈高,该径向力愈大,使泵的工作条件变坏,不仅加速轴承的磨损,减低泵的寿命,而且会使轴变形,造成齿顶与壳本内表面之间的摩擦,使泵的总效率降低。为了解决齿轮泵径向受力不平衡的问题,有的泵在侧盖或座圈上开有平衡槽,如图2-8(a)所示。这种方法会增多泄漏的途径,使容积效率降低,压力上不去,此外加工较复杂。另一种方法是缩小压油口(见图2-8(b)),通过减小压力油作用在齿轮上的面积来减小径向力,虽然采用这种方法后径向力未得到完全平衡,轴仍受径向力的作用而产生弯曲变形,但可稍加大齿顶的径向间隙以减小摩擦,由于圆周密封带较长,漏油的增加并不显著。

\begin{figure}[htbp]
\centering
\ifOpenSource
\includegraphics[width=7cm]{cover.jpg}
\else
\includegraphics{fig0208.pdf}
\fi
\caption{径向力平衡的方法}
\label{fig:fig0208}
\end{figure}

\section{齿轮泵的泄漏问题和高压化措施}

对任何容积式液压泵来讲,为了提高其工作压力,必须使液压泵具有较好的密封性能,但为了实现密封容积的变化,相对运动的零件间又不得不具有一定的间隙,这就构成了一对矛盾。因此,提高容积式液压泵工作压力的途径就是要合理地解决这一矛盾。对齿轮泵来讲,漏油的途径有齿顶圆和壳体内孔之间的径向间隙;齿轮端面和侧盖之间的轴向间隙以及由于在齿宽方向上不能保证完全啮合而造成的齿面缝隙。而其中尤以齿轮端面的轴向间隙对泄漏的影响为最大,油压愈高,泄漏愈多。如果制造时减小此间隙,这不仅会给制造带来困难,而且将引起齿轮端面的很快磨损,容积效率仍不能提高。所以高压外啮合齿轮泵一般都采取利用液压力来补偿轴向间隙的方法。目前国内生产的外啮合齿轮泵,主要是采用浮动轴套或采用浮动侧板来自动补偿轴向间隙,这两种方法都是引人压力油使轴套或侧板贴紧齿轮端面,压力越高贴得越紧,便可自动补偿轴向磨损和间隙,这种泵结构紧凑,容积效率高,但是流量脉动较大。

\chapter{叶片泵}

由于普通齿轮泵的工作压力较低,流量脉动较大,且流量不能调节,在机床的中压系统中或要求运动平稳的机床上广泛采用了叶片泵。叶片泵又分为双作用叶片泵和单作用叶片泵两种形式,前者为定量泵,后者一般为变量泵。

\section{双作用叶片泵}

\subsection{双作用叶片泵的工作原理}

双作用叶片泵的工作原理可以用图2-9所示的简图来说明。该泵由转子1、定子2、叶片3、配油盘4以及泵体5等零件组成。定子2与泵体5固定在一起,其内表面类似椭圆形,是由与转子同心的四段圆弧($\wideparen{ab}$,$\wideparen{cd}$,$\wideparen{ef}$,$\wideparen{gh}$)和连接这些圆弧的四段过渡曲线(bc,de,fg,ha)所组成(见图2-10)所示。其中$\wideparen{ab}$和$\wideparen{ef}$圆弧段的半径为$R$,$\wideparen{cd}$和$\wideparen{gh}$圆弧段的半径为$r$,且$R>r$。叶片3可在转子径向叶片槽中灵活滑动,叶片槽的底部通过配油盘上的油槽(图中未表示出来)与压油窗口相连。当电机带动转子1按图示方向转动时,叶片在离心力和叶片底部压力油的双重作用下,向外伸出,其顶部紧贴在定子内表面上。处于圆弧上的四个叶片分别与转子外表面、定子内表面及两个配油盘组成四个密封工作油腔,这些密封工作油腔随着转子的转动,在图示2和4象限内,密封工作油腔的容积逐渐由小变大,通过配油盘的吸油窗口(与吸油口相连),将油液吸入。在图示1和3象限,密封工作油腔的容积由大变小,通过配油盘的压油窗口(与压油口相连),将油液压出。由于转子每转一转,每个工作腔完成两次吸油和压油,所以称为双作用叶片泵。由图不难看出,两个吸油区(低压)和两个压油区(高压)在径向上是对称分布的。作用在转子上的液压作用力互相平衡,使转子轴轴承的径向载荷得以平衡,故也称为卸荷式叶片泵。由于改善了机件的受力情况,所以双作用叶片泵可承受的工作压力比普通齿轮泵高。一般国产双作用叶片泵的公称压力为$63\times10^5$Pa。

\begin{figure}[htbp]
\centering
\ifOpenSource
\includegraphics[width=7cm]{cover.jpg}
\else
\includegraphics{fig0209.pdf}
\fi
\caption{双作用叶片泵的工作原理}
\label{fig:fig0209}
\end{figure}

\begin{figure}[htbp]
\centering
\ifOpenSource
\includegraphics[width=7cm]{cover.jpg}
\else
\includegraphics{fig0210.pdf}
\fi
\caption{定子内表面形状}
\label{fig:fig0210}
\end{figure}


\subsection{双作用叶片泵的结构特点}

(1)叶片倾角。在双作用叶片泵中,叶片在转子槽中的安装并不是沿转子半径方向,而是将叶片顶部朝转子旋转方向往前倾斜了一个角度如图2-11(a)。其理由如下:因为当叶片在压油区工作时,定子内表面将叶片向中心顶入,定子内表面给叶片的作用力其方向是沿内表面的法向,所以该力与叶片移动方向的夹角是$\alpha$,称为压力角。定子曲线坡度愈陡,压力角$\alpha$就愈大。若叶片沿径向放置(见图2-11(b)),则定子内表面对叶片的法向作用力$F_N$便与叶片成一个较大的角度$\beta$。$F_N$可分为两个分力,即沿叶片方向的分力$F_P$和垂直叶片的分力$F_T$,力$F_T$会使叶片弯曲,使叶片在槽中偏斜而引起磨损不均匀,滑动不灵活。当力$F_T$太大时(力$F_T$随$\beta$角和液压力的增大而增大),甚至会发生叶片折断和卡死现象。所以应将叶片相对转子半径倾斜一个角度$\theta$,尽量使力$F_N$的方向与叶片运动方向一致。最理想的情况是将叶片槽开在定子内表面的法线方向上,但是过渡曲线上各处的法线方向是不同的,所以只好根据理论分析和试验探索以选择适当的叶片倾角$\theta$。国产双作用叶片泵的叶片倾角取$\theta=13°$。目前,关于叶片倾角的问题仍有争议,可采用的范围为$0^{\circ}\sim 13^{\circ}$。

\begin{figure}[htbp]
\centering
\ifOpenSource
\includegraphics[width=7cm]{cover.jpg}
\else
\includegraphics{fig0211.pdf}
\fi
\caption{叶片倾角}
\label{fig:fig0211}
\end{figure}

\par (2)定子曲线。双作用叶片泵定子内表面轮廓形状如图2-10所示。四个圆弧段对应的中心角为$\beta$;四个过渡曲线段所对应的中心角为$\alpha$。长半径$R$和短半径$r$的差值($R-r$)称为曲线的升程,它的大小直接影响到泵的输出流量。($R-r$)愈大,流量就愈大,但($R-r$)过大时,压力角太大,叶片易折断和卡死。且叶片在吸油区由于径向伸出运动的速度跟不上,容易引起叶片顶部和定子内表面的脱空现象,所以长短半径之比值有一定的限制。
\par 叶片转过圆弧部分时,叶片沿径向槽的运动速度为零。一旦进入过渡曲线时,叶片的径向速度就不为零。在曲线的转接处,如果叶片径向速度有突变,则叶片径向加速度将会很大,叶片就会以很大的力冲击定子内表面引起噪声和严重磨损。为避免此种现象,国产双作用叶片泵的定子过渡曲线采用了等加速曲线(见图2-12(a))。叶片从$B$点转到$C$点,扫过前一半$\alpha$角时,叶片按等加速运动规律作径向运动,其径向速度由零逐渐增加到最大。叶片从$C$点转到$D$点,扫过后一半$\alpha$角时,叶片按等减速运动规律作径向运动,其径向速度又从最大逐渐减小到零。由于在与圆弧$\wideparen{DE}$和$\wideparen{AB}$的连接处($D$和$B$点),叶片的径向速度为零,所以速度不会产生突变,叶片对定子内表面也不会产生过大的冲击力。

\begin{figure}[htbp]
\centering
\ifOpenSource
\includegraphics[width=7cm]{cover.jpg}
\else
\includegraphics{fig0212.pdf}
\fi
\caption{等加速度过渡曲线的形状和运动特性}
\label{fig:fig0212}
\end{figure}

\par 等加速度过渡曲线的极坐标方程如下:
\begin{gather}
\rho =r+\frac{2(R-r)}{\alpha ^2} \varphi ^2 \qquad (0\leqslant \varphi \leqslant \frac{\alpha }{2} ) \\
\rho =2r-R+\frac{4(R-r)}{\alpha } (\varphi -\frac{\varphi ^2}{2\alpha } ) \qquad (\frac{\alpha }{2} \leqslant \varphi \leqslant \alpha  )
\end{gather}
\\式中 \quad $\rho$ ——曲线的极径;
\par \quad \ $\varphi $——叶片的转角;
\par \quad \ $R$——长半径;
\par \quad \ $r$——短半径;
\par \quad \ $\alpha $——过渡曲线段对应的中心角。
\par 设转子的角速度为常数$\omega $,则有$\varphi =\omega t$,代入式(2-19)和式(2-20)可得
\begin{gather}
\rho =r+\frac{2(R-r)}{\alpha ^2} (\omega t)^2 \qquad (0\leqslant \varphi \leqslant \frac{\alpha }{2} ) \notag
\end{gather}
\par  叶片的径向速度
\begin{gather}
 v=\frac{d \rho }{d t}  =\frac{4\omega (R-r)}{\alpha ^2} \varphi 
\end{gather}
\par  叶片的径向加速度
\begin{gather}
a=\frac{dv}{dt} =\frac{4\omega ^2(R-r)}{\alpha ^2} =\mbox{常数} \\
\rho =2r-R+\frac{4(R-r)}{\alpha } [ \omega t-\frac{(\omega t)^2}{2\alpha } ]  \qquad (\frac{\alpha }{2}\leqslant \varphi \leqslant \alpha  )
\end{gather}
\par  叶片的径向速度
\begin{gather}
v=\frac{d \rho }{d t} =\frac{4\omega (R-r)}{\alpha } -\frac{4\omega (R-r)}{\alpha ^2} \varphi 
\end{gather}
\par  叶片的径向加速度
\begin{gather}
a=\frac{dv}{dt} =-\frac{4\omega ^2(R-r)}{\alpha ^2} =\mbox{常数}
\end{gather}
\par 由式(2-21)~式(2-24)可画出叶片的运动特性(见图2-12(b)),图中可看出叶片径向速度是均匀变化的,不会产生刚性冲击,因而使定子内表面受力和磨损均匀。但叶片在两端接点$B$,$D$和过渡曲线的中点$C$三处径向加速度仍有突变,由于加速度$a$为有限值,故只产生柔性冲击。
\par (3)配油盘的三角槽。在双作用叶片泵的配油盘(见图2-13)上,有两个吸油窗口2,4和两个压油窗口1,3,窗口之间为封油区。通常应使封油区对应的中心角$\beta$稍大于或等于两个叶片之间的夹角,否则会使吸油腔和压油腔连通,造成泄漏。当两个叶片间的密封油液从吸油区过渡到封油区(长半径圆弧处)时,其压力基本上与吸油压力相同。但当转子再继续旋转一个微小角度时,该密封腔突然与压油腔连通,使其中油液压力突然升高,油液的体积突然收缩,压油腔中的油倒流进该腔,使油泵的瞬时流量突然减小,引起油泵的流量脉动、压力脉动和噪声。为此在配油盘的压油窗口靠叶片从封油区进入压油区的一边开有一个截面形状为三角形的三角槽(又称眉毛槽),使两叶片之间的封闭油液在未进入压油区之前就通过该三角槽与压力油相连,使其压力逐渐上升,因而减缓了流量和压力脉动并降低了噪声。