
$$\mid{\mu}\mid=\mid{\frac{\tau}{du/dy}}\mid$$
\noindent${\mu}$的国际单位是${\frac{N \cdot s}{m^2}}$或${Pa \cdot s}$。

(2)运动黏度。运动黏度以${\nu}$表示,它是动力黏度${\mu}$与密度${\rho}$的比值,没有什么特殊的物理    ${{ }^{\circ }}{{E}_{t}}$
意义,即
\begin{gather}
    {\nu}={\frac{\mu}{\rho}}
\end{gather}
   \noindent${\nu}$的国际单位是$m^2/s$,常用$mm^2/s$。

    我国目前常用运动黏度${\nu}$来表示油液的黏度,普通机械油的牌号就是用该油液在
50℃(323K)时运动黏度${\nu}(mm^2/s)$的平均值来标志的。例如,10号机械油就是该油的运动
黏度为$10mm^2/s$。

(3)相对黏度。液体的动力黏度及运动黏度都难以直接测量,一般多用于理论分析和计算。相对黏度是一种以被测液体的黏度相对于同
温度下水的黏度之比值来表示黏度的大小的。相对黏度按其测试方法的不同,有多种名称。我国习惯采用恩氏黏度,以符号,表示。它们是在某标定温度(如20℃或50℃)下将$200cm^3$的被测油
液在自重作用下从恩氏黏度计中直径为2.8mm的小孔流出的时间${{t}_{1}}{ (s) }$,与$200cm^3$蒸馏水在20℃时从恩氏黏度计中流出所需时间${{t}_{2}}{(s) }$之比,即
\begin{gather}
    {{ }^{\circ }}{{E}_{t}}=\frac{{{t}_{1}}}{{{t}_{2}}}
\end{gather}

恩氏黏度计只能用来测定比水黏度大的液体。恩氏黏度与运动黏度的换算关系如下:
\begin{gather}
    {\nu_{t}}= (7.31{{ }^{\circ }}{{E}_{t}}-{ \frac  {6.31}   {{ }^{\circ } {{E}_{t}} }} )×{10^{-6}}m^2/s
\end{gather}
\subsection*{2.温度和压力对黏度的影响}

液压油的黏度随温度的增加而减小,这是因为液体的黏性是由于分子之间的相互作用力而引起的,这种作用力随着温度升高引起分子间的距离增大而减小。油液黏度的变化
直接影响液压系统的工作性能,因此希望黏度随温度的变化越小越好。当其运动黏度不超过$76×{10^{-6}}m^2/s$,温度变化在$30\sim150℃$范围内时,可用下式计算温度为t时的运动黏度:
\begin{gather}
    {\nu_{t}}={\nu_{50}} ( {\frac{50}{t}}) ^{n} 
\end{gather}
    \noindent 式中\ 
    \begin{tabular}[t]{ll}
        ${\nu_{t}}$ & —— 温度为t时油液的运动黏度;\\
        ${\nu_{50}}$ & —— 温度为50℃时油液的运动黏度;\\
        ${n}$ & —— 根据油液种类而定的常数。其值可参考表1-1 ;
    \end{tabular}

我国常用液压油的黏度与温度的关系可参阅图1-3所示国产油黏度温度曲线。

液压油的黏度随压力的升高而变大,其原因是由于分子之间距离缩小,内聚力增大所致。
其关系可以表示为
\begin{gather}
    {\nu_{p}}={\nu_{0}}{e}^{bp} 
\end{gather}
    \newpage  
    \noindent 式中\ 
    \begin{tabular}[t]{ll}
        ${\nu_{0}}$ & —— 压力为$10^5Pa$时液体的运动黏度;\\
        ${\nu_{p}}$ & —— 压力为$p$(相对压力)时的运动黏度;\\
        $p$ & —— 油液的压力($10^5Pa$); \\
        $b$ & —— 根据液体种类不同而定的系数,一般$b=( 0.002\sim0.003)        {\frac {1}  { 10^{5} Pa} } $。
    \end{tabular}

    若压力变化不大(变化值在5MPa以下),液体的黏度变化甚微,可忽略不计。如果压力变化大于20MPa,则液体黏度的变化就不容忽视了。
          \section*{四、对液压油的要求和选用}
在液压传动中,液压油既是传递动力的介质,又是润滑剂,油液还可以将系统中的热量扩散出去。在这三点作用中前两点是主要的。

随着液压技术的日益广泛应用,液压系统的工作条件、周围环境以及所控制的对象也越来越复杂,因此,要保证液压系统工作可靠、性能优良,对液压油必须提出以下几项要求:

(1)应具有合适的黏度,且黏温性要好,即黏度随温度的变化要小。黏性过大,油液流动时阻力大,功率损失大,系统效率低。黏度过小,将引起泄漏增加,系统效率也要降低。

(2)可压缩性要小,即体积弹性模量要大,释放空气性能要好,这是由于油中混入空气时,将大大降低油的体积弹性模量,降低系统的动态性能指标。

(3)润滑性要好,保证在不同的压力、速度和温度条件下,都能形成足够的油膜强度。

(4)具有较好的化学稳定性,不易氧化和变质,以免造成元件或机件的损坏,影响系统的正常工作。

(5)质量应纯净,应尽量减小机械杂质、水分和灰尘的含量。水混入液压油中,会降低液
\newpage
压油的润滑性、防锈性;其他杂质混入液压油中,会堵塞节流小孔和缝隙或导致运动部件卡死这些都影响系统工作的可靠性和准确性。

(6)对密封材料的影响要小,液压油对密封材料的影响主要是使密封材料产生溶胀、软化或硬化,结果都会使密封装置密封性能降低,系统泄漏增加。

(7)抗乳化性要好,不易起泡沫。油中如果混入水则在泵及其他液压元件的作用下,会产生乳化液,引起油的变质、劣化,生成油泥和沉淀物,降低使用寿命。
        
(8)流动点和凝固点要低,闪点(明火能使油面上油蒸气闪燃,但油本身不燃烧的温度)和燃点应高。

 在机床液压系统中,目前使用最多的是矿物油,常用的像机械油、汽轮机油等。随着液压技术的发展,对液压油提出了更高的要求,油液经过精炼或在其中加入各种改善其性能的添加剂——抗氧化、抗泡沫、抗磨损、防锈等的添加剂,以提高其使用性能。如精密机床液压油、稠化液压油以及航空液压油等,其使用性能超过一般的机械油。
         
选用液压油时首先考虑的是它的黏度。在确定黏度时应考虑下列因素:工作压力的高低;环境温度的高低;工作部件运动速度的高低。例如,当系统工作压力较高、环境温度较高、工作部件运动速度较低时,为减少泄漏,宜采用黏度较高的液压油。此外,各类泵对液压油的黏度有一个许用范围,其最大黏度主要取决于该类泵的自吸能力,而其最小黏度则主要考虑润滑和泄漏。各类液压泵的许用黏度范围可查阅有关液压手册。

几种国产液压油的主要质量指标见表1-2


   \newpage    
   \chapter*{1-2 液体静力学}    
   
   本节主要讨论静止液体的平衡规律以及这些规律的应用。所谓“静止液体”是指液体内部质点与质点之间无相对运动,至于盛装液体的容器,不论它是静止的或是运动的,都没有关系。
   \section*{一、静压力(或称压力)及其性质}
         作用在液体上的力有表面力和质量力两类。单位面积上作用的表面力称为应力,它有法向应力和切向应力,当液体静止时,液体质点间没有相对运动,不存在摩擦力,不呈现黏性,因而静止液体表面力只有法向力。因为液体质点间的内聚力非常小,不能受拉,所以法向力总是向着液体表面的内法线方向作用的。习惯上即称它为压力(或压强),用公式表示为
         \begin{gather}
            {p}={\frac{F}{A}}
        \end{gather}
\noindent 式中\ 
\begin{tabular}[t]{ll}
    $F$ & —— 作用在流体上的外力;\\
    $A$ & —— 外力作用的面积;\\
    $p$ & —— 压力(或压强)。 
   
\end{tabular}

        如果流体上各点的压力是不均匀的,则液体中某一点的压力可写为
$$    p=\lim_{{\Delta A}\to 0}  (\frac {\Delta F}  {\Delta A } )   $$


        此外,液体的压力还有如下性质,即静止液体内任意点处的压力在各个方向上都相等。

\section*{二、在重力作用下静止液体中的压力分布} 

         在重力作用下的静止液体,其受力情况如图1-4所示,如要求得液体内任意点A的压力,可从自由液面向下取一微小圆柱体,其高度为h,底面积为$\Delta{A}$,这微小圆柱体在重力及周围压力作用下处于平衡状态,于是有
$${p\Delta{A}}={p_{0}\Delta{A}}+{F_{G}}$$
式中 ${F_{G}}$ 为液柱重力,即${F_{G}}={{\rho}gh}{\Delta{A}}$,代入上式并化简
\begin{gather}
    {p}={p_{0}}+{{\rho}gh}
\end{gather}
式中 ${p_{0}}$ 为作用于流体表面上的压力。
 由(1-14)可以看出:

(1)静压力由两部分组成:一是液面上的压力${p_{0}}$;二是液柱质量产生的压力${{\rho}gh}$。当液面上只有大气压力${p_{a}}$作用时,则A点处静压力为
\begin{gather}
    {p}={p}_{a}+{{\rho}gh}
\end{gather}

(2)静止液体内的压力沿深度呈直线规律分布

(3)离液面深度相同处各点的压力都相等。压力相等的所有点组成的面叫做等压面。在