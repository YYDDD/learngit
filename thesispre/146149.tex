运动时的有效面积;\ding{174}以上两种方法的联合使用。

下面介绍几种机床上常见的快速运动回路。

\subsection{差动连接回路}

\begin{spacing}{2.0}
图6-22所示为一差动连接回路。图示位置时,若二位三通阀通电,液压缸差动连接,活塞便获得快速运动,其速度为非差动连接时的$ \displaystyle\frac{A_{1}}{A_{1}-A_{2}}$倍。如欲使快进与快退速度相等,则需使$A_{1}=2A_{2}$,此时快进(退)速度为工进速度的2倍。
\end{spacing}

差动连接时,油缸右腔的回油$Q_{2}$经二位三通阀后与液压泵供给的油液$Q_{1}$一起进入液压缸左腔,相当于增大了供油量。此时,进油路上的某些管路与阀的通过流量增大,其规格必须按差动时的流量选择,以免压力损失与功耗过大。

这种回路方法简单、经济,但由于差动时的推力减小,差动速度愈大,执行元件输出的推力愈小,故快速运动的速度不能太高。如欲获得较大的运动速度,常与双泵供油或限压式变量泵供油等方法联合使用。

\subsection{双泵供油回路}

图6-23为采用双泵供油实现快速运动的回路。1是小流量泵,2是大流量泵。快速运动时,系统压力小于卸荷阀3(起卸荷作用的液控顺序阀)的调整压力,阀3关闭,两泵同时向系统供油。工作进给时,系统压力升高,阀3打开,泵2卸荷,单向阀4关闭,系统由小泵1单独供油,系统最大工作压力由溢流阀5调节。

这种方法和单泵供油方式相比,效率较高,功率损失较小,故得到广泛应用。但须设置两个油泵或采用双联泵,泵站结构较为复杂。

\subsection{采用增速缸与限压式变量泵组合的快速运动回路}

如图6-24所示,快速运动时(轻载),顺序阀2关闭,限压式变量泵供给的低压油经固定在缸体上的柱塞4中心孔而进入活塞5内的增速腔\uppercase\expandafter{\romannumeral2}。由于增速腔\uppercase\expandafter{\romannumeral2}的有效工作面积较小,且系统压力较低,变量泵处于最大输油量状态,故活塞5获快速向右运动,左腔\uppercase\expandafter{\romannumeral1}通过液控单向阀3从油箱补油。当进入工作行程时,系统压力升高,限压式变量泵输油量减少,同时顺序阀2打开,压力油同时进入油腔\uppercase\expandafter{\romannumeral1}与\uppercase\expandafter{\romannumeral2},活塞获得低速工作运动。快速退回时,液压泵供油进入缸右腔\uppercase\expandafter{\romannumeral3},同时打开液控单向阀,使左腔\uppercase\expandafter{\romannumeral1}的油液流回油箱。

这种回路由于增速缸内的增速腔有效工作面积可以做得远比活塞面积小,加上限压式变量泵又能在系统压力上升时自动减小输出流量,故可使系统在空行程时获得远比工作速度为高的快速运动,功率利用较合理。它在压力机的液压系统中应用较多,但液压缸结构与油路较复杂。


\subsection{采用蓄能器的快速运动回路}

如图6-25所示,在图示位置,液压缸停止工作时,泵经单向阀向蓄能器充液,使蓄能器储存能量。当蓄能器压力达到某一调定值时,卸荷阀打开,使泵卸荷,单向阀使蓄能器保压。 当电磁换向阀通电使左位或右位接通回路时,泵和蓄能器同时给液压缸供油,使活塞获得快速运动。卸荷阀的调整压力应高于系统最高工作压力。

这种回路可采用较小流量的液压泵,而在短时间内能获得较大的快速运动速度。但系统在整个工作循环内需有足够的停歇时间,以使液压泵能完成对蓄能器的充液工作。

\section{速度换接回路}

机床在做自动循环的过程中,工作部件往往需要有不同的运动速度,经常进行不同速度的变换,如快速趋近工件变换到慢进工作速度,从第一种工作进给速度变换到第二种工作进给速度,等等。这就需要系统具有速度换接回路。对于加工精度要求高的机床,在速度换接过程中(特别是在两种工作进给速度的变换过程中)要求换接平稳,不允许出现前冲现象(速度换接时速度突然增大使工作部件出现跳跃式前冲)。

\subsection{快速运动和工作进给运动的换接回路}

\subsubsection{利用电磁阀或行程阀实现快速运动和工作进给运动的换接回路}

如图6-26所示,在图示位置,泵输出的压力油经二位四通阀进入液压缸左腔,右腔回油经二位二通电磁阀(或行程阀),再经二位四通阀流回油箱,获快速运动。在活塞杆上挡块压住行程开关$X$,控制二位二通电磁阀通电,使通道切断(或直接压下二位二通行程阀,把通道切断)后,回油必须经调速阀流回油箱,实现慢速工作进给运动。二位四通换向阀切换后,压力油经单向阀进入液压缸右腔,实现快速退回。

采用电磁阀的换接回路,安装比较方便,除行程开关须装在床身上外,其他液压元件均可集中安装在靠近液压泵的液压柜中,但速度换接时平稳性较差。采用行程阀的换接回路,由于行程阀通道的关闭与切换是逐渐进行的,故换接时速度平稳,但行程阀必须安装在床身上,管道的连接较长,较不方便。

调节活塞杆上挡块与行程开关(或行程阀)间的距离及挡块的长度,便可调节快速运动行程及工作进给行程的长度,调整比较方便,结构比较简单。

\subsubsection{利用液压缸本身结构实现快慢速度换接回路}

图6-27所示是一种利用特殊结构的液压缸速度换接回路。在图示位置,缸右腔回油经油路1和二位四通换向阀流回油箱,活塞获快速运动。当活塞移动到封盖住油路1的通口处时,右腔回油必须经节流阀3,然后经换向阀流回油箱,活塞转变为工作进给。二位四通换向阀切换后,压力油经单向阀2进入缸右腔,使活塞快速退回。

这种回路结构简单,速度换接位置准确,但不能调节。工作行程长度由活塞宽度决定,是固定的,故行程一般不宜太长。 此外,由于油路1的存在,活塞上不能使用密封件,只能采用间隙密封,故这种回路只宜用于压力不高、工作进给的行程不长、工作状况固定的场合。

图6-28所示是采用另一种特殊结构的双活塞液压缸速度换接回路。在活塞杆上浮动的活塞7与主活塞9之间的最大距离$l_{1}$可以通过螺母6加以调节。在图示位置,压力油经换向阀进入缸左腔,两个活塞(其间充满油液并保持$l_{1}$距离)一起向右做快速运动,右腔油液经油路5和换向阀流回油箱。当浮动活塞7越过油口a到达端点时,两活塞之间的油液从油口a经节流阀4和换向阀流回油箱,主活塞及活塞杆便以慢速向右继续运动,直至碰到浮动活塞时为止。当换向阀换向时,压力油进入缸右腔.先通过浮动活塞上的单向阀8.使主活塞连同活塞杆向左快速退回。直至螺母6碰到浮动活塞时,带动后者一起向左运动。此时,两个活塞间又充满了油液。

这种回路亦可获得准确的速度换接位置,工作运动行程可以调节,但液压缸结构比较复杂。

\subsection{两种工作速度的换接回路}

\subsubsection{两个调速阀并联式速度换接回路}

图6-29为两个调速阀并联实现两种工作进给速度换接的回路。在图示位置,液压泵输出的压力油经调速阀3和电磁阀5进入液压缸,当需要第二种工作速度时,电磁阀5通电切换,使调速阀4接入回路,压力油经调速阀4和电磁阀5进入液压缸。这种回路,当一个调速阀停止工作没有油流通过时,它的减压阀处于完全打开的位置。当它被突然接入回路时,会使工作部件出现突然前冲的现象,这在某种工作场合下是不允许的。

图6-30所示为另-种调速阀并联的两种工进速度换接回路。这里,两个调速阀始终处于工作状态,故一种工作速度转换为另一种工作速度时,不会出现执行部件突然前冲的现象。但系统在工作时,总有一部分油液通过其中一个不起调速作用的调速阀流回油箱,造成能量损耗。故对于工作速度较大(调速阀开口大,能量损耗亦大)的系统,不宜采用这种回路。

\subsubsection{两个调速阀串联式速度换接回路}

图6-31所示为两个调速阀串联的速度换接回路。在图示位置,压力油经调速阀3和电磁阀5进入液压缸,工作部件的运动速度由调速阀3控制。当电磁阀5通电切换时,调速阀4接入回路,压力油经调速阀3和4进入液压缸,工作部件的运动速度由调速阀4控制。调速阀4的开口量应调得比阀3为小,否则将不起作用。这种回路的能量损失比图6-29的大,但比图6-30的小。由于速度换接的瞬间,调速阀3仍在工作,可限制通过调速阀4的流量突然增加,故其换接平稳性亦比图6-29所示的回路好。
