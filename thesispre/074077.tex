

这两部分在组装后用四根长拉杆6串起来,并用螺母固紧。为了保证液压缸具有可靠的密封
性,在前、后端盖和缸体之间,缸体和活塞之间,活塞杆和后端盖之间以及活塞和活塞杆之间都分别设置了相应的密封件12,2,7等。活塞杆的伸出端由装有刮油、防尘装置9的导向套10支撑。为了防止活塞在两端对端盖的撞击,在前、后缸盖中都设置了由单向阀14和节流阀13组成的缓冲装置,其工作原理将在本节后面详细介绍。在液压缸工作前,应先放出缸内积聚的空气,为此在缸体的最上方开设排气装置(图中未表示出来) ,本节后面将介绍。

\indent 从以上对液压缸典型结构的分析可看出,液压缸是由缸体组件、活塞组件以及密封装置、缓冲装置、排气装置等所组成。它们的结构和性能直接影响到液压缸的工作质量和制造成本。下面分别做一介绍。

\indent 一、缸体组件

\indent 图3-6所示为几种常用的缸体组件的结构,设计时,主要应根据液压缸的工作压力、缸体材料和具体工作条件来选用不同的结构。一般工作压力低的地方,常采用铸铁缸体,它的端盖多用法兰连接,如图3-6(a)所示。这种结构易于加工和装拆,但外形尺寸大。工作压力较高时,可采用无缝钢管的缸体,它与端盖的连接方式如图3-6(b),(c),(d)所示。采用半环连接(见图3-6(b)),装拆方便,但缸壁上开了槽,会减弱缸体的强度。采用螺纹连接(见图3 -6(c)),外形尺寸小,但是缸体端部需加工螺纹,使结构复杂,加工和装拆不方便。图3 -6(d)所示为焊接结构,构造简单,容易加工,尺寸小。缺点是易产生焊接变形。图3-5所示缸体和端盖的连接是采用四根拉杆固紧的方法,缸体的加工和装拆都方便,只是尺寸较大。

\indent 二、活塞组件

\indent 最简单的形式是把活塞和活塞杆做成一体,这种结构虽然简单,工作可靠,但是当活塞直径大,活塞杆较长时,加工较费事。

\indent 图3-7所示为几种常用的活塞组件结构形式,其中图3-7(a)所示为活塞和活塞杆之间采用螺纹连接的方式,它适用于负载较小受力较平稳的液压缸中。当液压缸工作压力较高或负载较大时,由于活塞杆上车有螺纹,强度有所削弱。另外工作机构振动较大时,因必须设置螺母防松装置而使结构复杂,这时可采用非螺纹连接的方式,如图3- 7(b),(c),(d)所示。图3- 7(b)中所示活塞杆5.上开有一个环形槽,槽内装有两个半圆环3以夹紧活塞4,半圆环3用轴套2套住。弹簧圈1用来轴向固定轴套2。图3-7(c) 所示的活塞杆1使用了两个半圆环4,它们分别由两个密封圈座2套住,然后在两个密封圈座之间塞人两个半圆环形的活塞3。图3-7(d)中,则是用锥销1把活塞2固定在活塞杆3上。

\indent 由于活塞组件在液压缸中是-一个支撑件,必须有足够的耐磨性能,所以活塞一般都是铸铁的,而活塞杆通常都是用钢做的。

\indent 三、密封装置

\indent 液压缸中的密封主要指活塞和缸体之间,活塞杆和端盖之间的密封,它是用来防止内、外泄漏的,液压缸中密封性能的好坏,直接影响到液压缸的工作性能和效率,因此设计时应根据液压缸不同的工作条件来选用相应的密封方式.-般对密封装置的要求是:

\indent (1)在一-定工作压力下,具有良好的密封性能。最好是随压力的增加能自动提高密封性能,使泄漏不致因压力升高而显著增加。

\indent (2)相对运动表面之间的摩擦力要小,且稳定。

\indent (3)要耐磨,工作寿命长,或磨损后能自动补偿。

\indent (4)使用维护简单,制造容易,成本低。

\indent 液压缸中常见的密封形式有下述几种。

\indent 1.间隙密封

\indent 间隙密封是靠相对运动件配合表面间微小间隙来防止泄漏的(见图3-8)。它的密封性能与间隙大小、压力差、配合表面长度、直径以及加工质量有关。为了提高它的密封性能,在活塞上常开有深0.3 ~0.5 mm的截面为三角形的环形槽(也称作平衡槽),在环形槽中形成等压区,使作用在活塞上的径向液压力得到平衡,有使活塞自动对中的作用,从而减小了活塞和油缸配合表面间的摩擦力,并减少泄漏量。关于环形槽的作用分析,在第四章中还将详细说明。间隙密封结构简单,摩擦力小,在滑阀中被广泛采用,但隙密封结构简单,摩擦力小,在滑阀中被广泛采用,但是其密封性能不能随压力的增大而提高,且磨损后不能自动补偿间隙,当活塞直径大时,配合表面很大,要保证缸体很高的加工精度有一定困难,且不经济,因此一般在液压缸中较少采用,而仅用于直径小、运动速度快的低压液压缸中。

\indent 2.活塞环密封

\indent 如图3- 9(a)所示,在活塞的环形槽中,嵌放有开口的金属活塞环,其形状如图3-9(b)所示。活塞环依靠其弹性变形所产生的张力紧贴在油缸内壁,从而实现密封,这种密封装置的密封效果较好,能适应较大的压力变化和速度变化,耐高温,使用寿命长,易于维护保养,并能使活塞有较长的支撑面。缺点是制造工艺复杂,因此只适用于高压、高速或密封性能要求较高的场合。

\indent 3.密封圈密封

\indent 密封圈密封是液压元件中应用最广的一种密封形式,它的优点在于:

\indent (1)结构简单,制造方便,是大量生产的标准模压件,所以成本低;

\indent (2)能自动补偿磨损;

\indent (3)油液的工作压力越高,密封圈在密封面上贴得越紧,其密封性能可随着压力的加大而提高,因而密封可靠;

\indent (4)被密封的部位,表面不直接接触,所以加工精度可以降低;

\indent (5)既可用于固定件,也可用于运动件。

\indent 密封圈的材料应具有较好的弹性,适当的机械强度,耐热耐磨性能好,摩擦系数小,与金属接触不互相黏着和腐蚀,与液压油有很好的“相容性”。目前用得最多的是耐油橡胶,其次是尼龙和聚氨酯,也有的为了增加耐磨性,在密封圈表面喷涂上一层聚四氟乙烯。密封件的形状应使密封可靠、耐久,摩擦阻力小,容易制造和拆装,特别是应能随压力的升高而提高密封能力和利于自动补偿磨损。

\indent 常用密封圈按其断面形状可分为O形密封圈和唇形密封圈,而唇形密封圈中又可分为Y形、V形等密封圈,现分述如下:

\indent 图3-10(a)所示为O形密封圈的形状,其外侧、内侧及端部都能起密封作用。0形密封圈装人沟槽时的情况如图(a)右部所示,图中δ1和δ2为O形圈装配后的预变形量,它们是保证间隙的密封性所必须具备的,预变形量的大小应选择适当,过小时会由于安装部位的偏心、公差波动等而漏油,过大时对运动件上用的O形密封圈来说,摩擦阻力会增加,所以固定件上O形圈的预变形量通常取大些,而运动件上O形圈的预变形量应取小些,由安装沟槽的尺寸来保证。用于各种情况下的0形圈尺寸,连同安装它们的沟槽的形状、尺寸和加工精度等可从设计手册中查到。0形密封圈一般适用于低于100X 105 Pa的工作压力下,当压力过高时,可设置多道密封圈,并应加用密封挡圈,以防止O形圈从密封槽的间隙中被挤出。使用0形圈的优点是简单,可靠,体积小,动摩擦阻力小,安装方便,价格低,故应用极为广泛。

\indent 图3-10(b)所示为Y形密封圈,一般用耐油橡胶制成,它在工作时受液压力作用使唇张开,分别贴在轴表面和孔壁上,起到密封作用。为此,在装配时应注意使唇边面对有压力的油腔。这种密封圈因摩擦力小,在相对运动速度较高的密封面处也能应用,其密封能力可随压力的加大而提高,并能自动补偿磨损。

\indent 图3-10(c)所示为V形密封圈,它是用多层涂胶织物压制而成的,并由三个不同截面的支撑环、密封环和压环组成,其中密封环的数量由工作压力大小而定。当工作压力小于$100*10^5 Pa$时,使用三件一套已足够保证密封。压力更高时,可以增加中间密封环的数量。它与Y形密封圈一样,在装配时也必须使唇边开口面对压力油作用方向。V形密封圈的接触面较长,密封性好,但摩擦力较大,在相对速度不高的活塞杆与端盖的密封处应用较多。

\indent 图3-7中表示了O形和Y形密封圈在液压缸和活塞密封处的应用情况。图3-11中的(a),(b),(c)分别表示了O形、Y形和V形密封圈在活塞杆和端盖密封处的应用情况。对于工作环境较脏的液压缸来说,为了防止脏物被活塞杆带进液压缸,使油污染,加速密封件的磨损,需在活塞




