
先导式溢流阀的主阀弹簧14比较软,刚度很小,在很小的外力作用下即可被压缩,主阀芯的位移量大小,对系统的压力影响较小。
先导阀7的结构尺寸较小,其锥阀6的承压面积亦较小,调压弹簧8不必选用刚度较强的弹簧,因而使调节压力比较轻便。阻尼孔3起到增加主阀芯上下移动的阻尼,可以起稳定主阀芯的作用。%\newline
\par 由图4-22可以列出溢流阀阀芯受力的平衡方程式
\begin{equation}
   pA_{v}=F_{s}+G+F_{w}+F_{f}+p^{'}A_{v}\tag{4-1} %行间公式双$符号 
\end{equation}
\\式中\
\begin{tabular}[t]{lp{8mm}l}
   p &—— &液压力;\\
   $F_{s}$ &—— &主阀弹簧作用力;\\
   $F_{w}$ &——&稳态轴向液动力;\\
  G &——&阀芯自重;\\
  $F_{f}$&——&阀芯与阀体之间的摩擦力;\\
 $A_{v}$&——&阀芯截面积;\\
  $p^{'}$&——&p经过阻尼孔后的压力,$p^{'}=\frac{F_{s8}}{A_6}$;\\
\end{tabular}
%\\$p^{'}=F_{s8}/A_6$   %分式的两种表示
%\\$p^{'}=\frac{F_{s8}}{A_6}$   
\par $p^{'}$ 由先导阀调定,保持基本不变,$F_{s8}$是先导阀弹簧8的弹簧力,$A_{6}$是锥阀6的承压面积。如将G,$F_f,F_w$略去不计,则上式可写成
\begin{equation}
   p=p^{'}+\frac{F_{s}}{A_v}\tag{4-2}
\end{equation}
  
   %  $p=p^{'}+\frac{F_{s}}{A_v}$\ tag{4-2}  
\par 由前所述可知,溢流阀的进口压力p可以保证基本是一个恒值。
\par 2.溢流阀的特性
\par 溢流阀的工作性能分为静态特性与动态特性两部分。
\par (1)静态特性。静态特性主要有压力稳定性、启闭特性和黏滞特性等。
\par 1) 压力稳定性。压力稳定性是指溢流阀在调定压力下长期工作的性能。压力稳定性的好坏一般用压力脉动、压力偏移和噪声等的大小来衡量。
它们的大小与阀的结构、阀芯移动阻尼的大小、加工精度、油液性质和油温的变化等因素有关。一般溢流阀的压力脉动与压力偏移要求不大于$\pm2×10^5Pa$.
\par 2)启闭特性。启闭特性通常用流量-压力曲线表示,是静态特性中的重要特性。它表示溢流阀从开启到闭合的过程中,通过阀的流量与控制压力之间的关系。
\par 图4-23所示为溢流阀的启闭特性。理想的溢流阀其特性曲线最好是一条在 $p_t$处平行于纵坐标的直线。它表示溢流阀进口处压力p低于$P_t$时不溢流,仅在p到达$p_t$时才溢流,而且不管溢流量的多少,其压力始终保持在$p_t$值上。
图4-23所示溢流阀的实际特性曲线说明阀的工作压力是随溢流量的变化而变化的。这组曲线可以通过理论分析和实验得出。下面以图4-24所示的直动型溢流阀的原理图为例,来分析其启闭特性。
\par 当系统的初始压力为$p_0$时,滑阀尚未开启,但已经处在液压力与弹簧力相平衡的状态,弹簧的预压缩量为$x_0$,滑阀进油口的直径为d,此时
\begin{equation}
   \frac{\pi}{4}d^2p_0=K_sx_0\tag{4-3}
\end{equation}
\\式中\ 
\begin{tabular}[t]{lp{8mm}l}
   d &—— &滑阀进油口直径;\\
   $p_{0}$ &—— & 系统的初始油液压力;\\
   $K_s$ &——&弹簧刚度;\\
  $x_0$&——&弹簧预压缩量。\\
\end{tabular}
\par 当油液压力$p_0$上升为$p_1$时,阀门开口量为x,则弹簧总压缩量为$x_0$+x,此时阀芯平衡方程式为
\begin{equation}
   \frac{\pi}{4}{d_1}^2p_1=K_s(x_0+x)\tag{4-4}
\end{equation}
\\式中\ 
\begin{tabular}[t]{lp{8mm}l}
   $p_{1}$ &—— &溢流阀进口油液压力;\\
   $d_{1}$ &—— & 阀芯直径;\\
   x &——&开口量。\\
\end{tabular}
\par  若设d $\approx$ $d_1$,将式(4-4)减去式(4-3)得
\begin{equation}
   x=\frac{\pi}{4K_s}{d_1}^2(p_1-p_0)\tag{4-5}
\end{equation}
\qquad 流过阀口缝院的流量可依据下式计算 %空两格
\begin{equation}
      Q=C_dA\sqrt{\frac{2}{\rho}\Delta p}=C_d\pi d_1x\sqrt{\frac{2}{\rho}\Delta p}\tag{4-6}
\end{equation}
\\式中\ 
\begin{tabular}[t]{lp{8mm}l}
   Q &—— &流过阀口的流量;\\
   A &—— & 滑阀开口后所形成的环形过流面积;\\
   $\Delta p$ &——&阀门节流口前后两端的油液压力差;\\
   $C_d$ &——&流量系数;\\
   $\rho $ &——&油液的密度。\\
\end{tabular}
\par 将式(4-5)代人式(4-6),可得
\begin{equation}
   Q=C_d\pi d_1\frac{\pi}{K_s}{d_1}^2(p_1-p_0)\sqrt{\frac{2}{\rho}\Delta p}
=\frac{C_d\pi^2d_1^3}{4K_s}(p_1-p_0)\sqrt{\frac{2}{\rho}\Delta p}\tag{4-7}  
\end{equation}
\par 因为$\Delta p$=$p_1$-$p_2$,溢流阀流出口的压力$p_2$一般都是接通油箱的,可以认为其压力为零,
所以$\Delta p$=$p_1$,将$\Delta p$=$p_1$代人上式可得
\begin{equation}
   Q=\frac{C_d\pi^2d_1^3}{4K_s}\sqrt{\frac{2}{\rho}}(p_1^{\frac{3}{2}}-p_0p_1^\frac{1}{2})\tag{4-8}
\end{equation}
\qquad  从式(4-8)中可以看出,流量-压力特性曲线如图4-25所示。
不同的初始压力,对应着不同的曲线(改变弹簧的预压缩量$x_0$,就可调节初始压力$p_0$的大小),当$p_0$=0时,$X_0$=0,
即是曲线1。当$p_0$增加时,则曲线右移为2,3,4。$p_0$越大,曲线离原点越远,溢流阀所控制的压力值p越大。

从图4-23的曲线可看出,直动型溢流阀由于阀芯弹簧刚度较硬,一定的开口量变化对应的压力变化量就比先
导式溢流阀的压力变化量大,所以先导式溢流阀的流量-
压力特性就好。从溢流阀的使用情况考虑,我们总希望开启和关闭过程中压力的变化要小,由于在开启和关闭时滑
阀摩擦力方向的不同,就使得两个曲线不重合。又因先导式溢流阀有主阀芯上的和先导阀上的两部分摩擦力,故它
的启闭曲线不重合更加显著。开启时,一般要求被试阀溢流口溢流量为额定流量的1\%时所对应的压力值与调定
压力值之比应在90\%以上,此对应的压力值称为阀的开启压力。闭合时,
被试阀溢流口溢流量为额定流量的1\%时所对应的压力值与调定压力值之比应在85\%以上,而此对应的压力值称为阀的闭合压力。

3) 黏滞特性。溢流阀阀芯工作时,由于受摩擦阻力的作用,因此就产生了黏滞现象。黏滞现象将使溢流阀工作特性曲线出现不灵敏区,
这个区间的存在增大了溢流阀所控制系统压力的波动范围。

4)其他特性。如溢流阀作为卸荷阀使用,则对它的卸荷压力,即在全部溢流时的压力损失有一定的要求,
一般规定卸荷压力为$3X10^5 Pa$.卸荷压力越小,油液的发热越小,表示阀的性能越好。

其他如内部泄漏、密封性能等都影响溢流阀的静态性能。这一些要求,在其他阀中也是重要的,在此就不再叙述了。

(2)动态特性。溢流阀的动态特性通常是指溢流阀由关闭(此时压力为$p_0$)到开启,再关闭的突然变化时,
溢流阀所控制的压力随时间变化的过渡过程品质。由于阀内流动和受力情况比较复杂,因而动态特性的理论分析就比较困难。
实践中往往采用计算机仿真和实测的方法来进行分析。实测曲线如图4-26所示。

1)压力超调量$\Delta p$。当压力从$p_0$突然上升到某一调定压力$p_t$时,液压系统将出现最大压力冲击峰值$p_{max}$。
压力超调量$\Delta p=p_{max}-p_t$要小,否则会发生元件损坏、管道破裂以及使一些以压力
作为控制信号的元件错误动作。

2)压力回升时间$\Delta t_2$。又称过渡过程时间或调整时间。当溢流阀从初始压力$p_0$开始升压并
稳定到调定压力$p_t$时所需时间为$\Delta t_2$,一般要求$\Delta t_2=0.1\sim 0.5s$。

3)卸荷时间$\Delta t_1$。当溢流阀从调定压力$p_t$开始下降至卸荷压力$p_0$时所需时间为$\Delta t_1$,一般要求$\Delta t_1=0.03\sim 0.1s$。

压力回升时间$\Delta t_2$与卸荷时间$\Delta t_1$,反映溢流阀在工作中从一个稳定状态转变到另一个稳定状态所需要的过渡时间的大小,过渡时间短,溢流阀的动态性能好。

从溢流阀的静、动态特性可以看到,我们既希望溢流阀的启闭特性好,也希望溢流阀的压力超调量小,显然这是矛盾的。而实际设计溢流阀时,是综合考虑的。

3.溢流阀在系统中的应用

溢流阀是定量泵供油液压系统中必不可少的元件。溢流网在液压系统中的应用大致可分为溢流恒压、安全限压防止过载、远程调压、造成背压和使系统卸荷等。

图4-27所示为溢流阀用于定量泵液压系统。溢流阀常开,随着执行元件所需油量的不同,阀的溢流量时大时小,使系统压力保持恒定。节溢流阀弹簧的弹力,即可调节系统的供油压力。

4-28所示为溢流阀用于变量泵系统以限制系统压力超过最大允许值,防止系统过载。在正常情况下,阀口关闭,当超载时,系统油压达到最大允许值(益流阀调定压力),
阀口打开,压力油通过阀口回油箱,油压便不再升高。在此情况下,溢流阀起安全限压保护作用,故又称安全阀。
