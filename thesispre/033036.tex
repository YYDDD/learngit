
$$ \frac{p_1}{\varrho g}+\frac{\alpha _1v_{1}^{2}}{2g}=\frac{p_2}{\varrho g}+\frac{a_2v_{2}^{2}}{2g}=\varSigma h_{\zeta} $$ 

\noindent 取$ \alpha _1=\alpha _2=1 $ ,并且$ v_1=v_2 $ ,则上式简化为

$$ \frac{p_1}{\varrho g}=\frac{p_2}{\varrho g}+\varSigma h_{\xi} $$ 

\noindent 式中$ \varSigma h_{\xi} $ 为液体流经小孔的局部能量损失,它包括两部分:液体流经截面突然缩小时的局部损失$ h_{\xi 1} $ 和突然扩大时的$ h_{\xi 2} $ 当收缩截面上的平均流速为$ v_e $ 时,即可写成
$$ h_{\xi}=\left( \zeta _1+\zeta _2 \right) \frac{v^2}{2g} $$
\noindent 带入上式,有
$$ \frac{p_1-p_2}{\varrho g}=\left( \zeta _1+\zeta _2 \right) \frac{v_{e}^{2}}{2g} $$ 

\noindent 由上式求出

$$ v_e=\frac{1}{\sqrt{\zeta _1-\zeta _2}}\sqrt{\frac{2}{\varrho}\left( p_1-p_2 \right)}=C_v\sqrt{\frac{2}{\varrho}\varDelta p} $$ 

\noindent 又因$ \zeta _2=\left( 1-\frac{A_e}{A_2} \right) ^2 $ 而$ \frac{A_e}{A_2}\ll 1 $ ,故$ \zeta _2=1 $ ,因此

$$ C_v=\frac{1}{\sqrt{1+\zeta _1}} $$ 
\noindent 称$ C_v $ 为速度系数,$ \varDelta p $ 为小孔前后的压力差$ \varDelta p=p_1-p_2 $ ,由此得流经小孔的流量为
$$ Q=A_ev_e=C_cC_vA_0\sqrt{\frac{2}{\varrho}\varDelta p}=C_dA_o\sqrt{\frac{2}{\varrho}\varDelta p} $$ 

\noindent 式中$ C_d $ 为流量系数,$ C_d=C_cC_v $ 。

流量系数的值由实验条件确定。图1-26给出了在液流完全收缩的情况下,当$ R_e\le 10^5 $ 时,$ C_d $ ,$ C_c $ ,$ C_v $ 与$ \text{Re} $ 之间的关系。当$ \text{Re}>10^5 $ 时,$ C_d $ 可以认为是不变的常数,计算时取平均值$ C_d $ 为0.60~0.62。

从图1-26看出,当$ R_e $ 较小时,$ C_d $ 随$ R_e $ 的增大而迅速增大,这是由于粘性起主导作用的结果。它对收缩系数影响较小,而对速度系数$ C_v $ 影响较大,此时$ C_d $ 主要受$ C_v $ 影响,随$ \text{Re} $ 增加而迅速增加。当$ \text{Re} $ 进一步增大时 ,$ C_d $ 随$ \text{Re} $ 增加而缓慢增加,这是因为此时黏性作用相对减小而惯性作用增大,直到惯性作用起主导作用时,它对收缩系数影响较大,而对$ C_v $ 影响较小。在$ \text{Re} $ 增大到一定值后,黏性作用可以忽略,此时$ C_v $ 趋近1,$ C_d $ 也趋于某一常数。
当液流不完全收缩时,管壁离小孔较近,此时管壁对液流起导向作用,流量系数可增大到0.7~ 0.8。

从以上对薄壁小孔的流量公式推导可以看出:流经薄壁小孔的流量$ Q $ 与小孔前后压差$ \varDelta P $ 的1/2次方成正比;摩擦阻力作用极小,流量受黏度的影响也很小,因而油温变化对流量影响也很小;此外,薄壁小孔不易堵塞。这些都使得薄壁小孔(或近似薄壁小孔)在流量控制阀中表现出较好的性能。

(2)细长小孔的流量公式。细长小孔一般是指小孔的长径比$ l/d>4 $ 时的情况,如液压系统中的导管、某些阻尼孔、静压支承中的毛细管节流器等。

液流在细长孔中流动,一般都是层流,若不计管道起始段的影响,可以应用前面推出的圆管层流的公式(1-41),即

$$ Q=\frac{\pi d^4}{128ul}\varDelta p=\frac{d^2}{32ul}A_o\varDelta p=CA_o\varDelta p $$
\noindent 式中$ A_o=\pi d^2/4 $ 即细长小孔截面积;$ C=d^2/\left( 32ul \right)  $ ;其他符号同前。

从式(1-51)可知:油液流经细长小孔的流量$ Q $ 与小孔前后压差$ \varDelta P $ 的一次方成正比;流量受油液黏性($ u $ )变化的影响较大,即油温变化引起黏度的变化,从而引起流过细长小孔的流量变化;此外,细长小孔较易堵塞。这些特点都和薄壁小孔不同。

介于薄壁小孔与细长小孔之间的孔,即$ 1/2<l/d\le 4 $ 时,称为厚壁小孔或称为短孔。这时的过流情况,除流束在入口处有收缩作用外,且收缩结束后,流束要扩大,致使扩大后有一段沿程损失,以后才流出。所以,能量损失应为收缩、扩大和沿程三个部分的能量损失之和。应该指出的是,这里的收缩仅发生在孔的内部,液流一旦流出短孔就不再收缩。一般液压系统中的圆柱形外伸管嘴的流出情况,均属此类。

厚壁孔加工起来比薄壁孔容易得多,因此特别适合于作固定节流孔用。流量计算也可采用薄壁小孔的公式,但流量系数$ C_d $ 应根据短管的形状和安装方式不同而作具体计算或查表,关于这方面的深入了解,可参考有关的流体力学专著。

2.缝隙的流量公式

在液压传动的元件中,适当的缝隙(间隙)是零件间正常相对运动所必需的。间隙对液压元件的性能影响极大。液压系统的泄漏主要是由于间隙和压力差决定的,泄漏的增加使系统油温升高效率降低,系统性能受影响。因此应尽可能减少泄漏以提高系统的性能,保证系统正常工作。

缝隙的大小相对于它的长度和宽度小得很多,因此,液体在缝隙中的流动受固体壁的影响很大,其流动状态一般均为层流。缝隙的流量公式不再推导,现列于表1- 4,可作为计算各种缝隙流量时选用。

表1-4公式中各符号的意义为:

$ Q $  —— 通过缝隙的流量(L/min);
$ b $  —— 缝隙的宽度(m);
$ \delta  $  —— 缝隙的高度(m);
$ \varDelta p $  —— 缝隙前后压力差(Pa);
$ u $  —— 油液的动力黏度(Pa. s);
$ l $  ——  缝隙的长度(m);
$ d $  —— 环形缝隙的直径或圆盘的中心孔径(m);
$ \varepsilon  $  —— 缝隙的相对偏心率,即内圆柱中心与外圆筒中心的偏心距离$ e $ 对缝隙$ \delta  $ 的比值,即$ \varepsilon =\frac{e}{\delta} $ 
$ D $  —— 圆盘外圆直径(m)。

当偏心环形缝隙的偏心率达到最大值,即$ \varepsilon =\frac{e}{\delta}=1 $ 时,偏心环形缝隙的流量增加为同心环形缝隙的2.5倍。

\section{液压冲击和气穴现象}

\noindent 一、液压冲击

在液压系统的工作过程中,因执行部件的突然换向或阀门突然关闭以及外负载的急剧变化而引起压力急剧变化,出现压力交替升降的波动过程,这种现象称为液压冲击。液压冲击常伴随着很大的噪音和振动,它的压力峰值有时会大到正常工作压力的几倍至几十倍.甚至足以使管道和某些液压元件产生破坏的程度。因此,弄清液压冲击的本质,估算出它的压力峰值,并研究抑制措施,是十分必要的。

液压冲击是一种非恒定流动,它的瞬态过程相当复杂,本节只是简单分析产生冲击的原因及压力峰值的计算方法。

通常所说的压力冲击主要有两种情况:

一种是阀门突然打开或关闭,以及系统中某些元件反应的滞后,使液流突然停止运动。由于管路中液流的惯性及油液的可压缩性等原因,将流体的动能转变为压力能,并迅速逐层形成压力流,在阀门前出现高压波,阀门后出现低压波(从而产生空穴),这种压力波在水力学中称为“水击”现象或“水锤”现象。由于油液的黏性作用,经过一段时间以后这种压力波逐渐衰碱而停止。

另一种情况是运动部件(如机床工作台)突然启动或停止,由于运动部件的惯性使液压缸和相连管道内的压力产生急剧的变化而形成压力波,产生液压冲击。

以上两种情况本质上都是相同的,产生的后果也是相类似的。

1. 液流突然停止时的液压冲击

设有如图1-27所示的一根等径直管,其上游与一固定水面的大水池相连,出口经一快速




