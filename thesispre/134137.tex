
对于工作中负载变化很大的回路,效率和有效功率$P_1$一样,随$p_p$和$A_j$而变,当$p_1=\frac{2}{3}p_p$时,回路有最佳效率。由式(6-17)和式(6-19)可知
$$
\eta \le \frac{0.385p_pQ'_1}{p_pQ_p}=0.385\frac{Q'_1}{Q_p}
$$
此式表明,$\frac{Q'_1}{Q_p}$越小,溢流损失越大,效率越低。因$Q'_p$恒小于$Q_p$,故此情况下的回路效率恒低于0.385。

由以上分析可知,这种进口节流调速回路是不宜在负载变化大的工作状态下使用的。负载变化大带来执行元件的速度变化大v速度稳定性差,回路效率也低。只有在负载恒定(或变化很小)、凋速范围不大的工作情况下,才能获得较好的速度稳定性与回路效率。

(2)出口节流调速回路(见图6-2)。

l)工作原理与回路参数。节流阀装在执行元件的回油路上﹐控制从执行元件回油腔流出的流量$Q_2$ ,从而也就控制了进入执行元件工作腔的流量$Q_1$,因为这两者有固定的比例关系,即
$$
v=\frac{Q_2}{A_2}=\frac{Q_1}{A_1}
$$

液压泵输出流量除流入执行元件的流量$Q_1$外,其余由溢流阀流回油箱﹐即
$$
Q_p=Q_1+\varDelta Q_y
$$
当忽略管路压力损失时,由活塞平衡方程可得
$$
p_1A_1=p_pA_1=F+p_2A_2
$$
或
$$
p_2=\frac{p_pA_1-F}{A_2}=\varDelta p_j
$$
可见,负载F越小,回油腔压力$p_2$越大,当$A_{2<}A_1$,且负载很小时,回油腔压力$p_2$可比工作腔压力大得多,甚至超过泵的供油压力很多,节流阀的两端将承受很大的压力差。

执行元件的运动速度﹐由通过节流阀从执行元件回油腔排出的流量$Q_2$决定﹐即
$$
v=\frac{Q_2}{A_2}=\frac{CA_jp_{2}^{\varphi}}{A_2}=\frac{CA_j\left( p_pA_1-F \right) ^{\varphi}}{A_{2}^{\varphi +1}}
$$

2)速度-负载特性。由式(6-24)可求得出口节流调速回路的速度刚性为
$$
k_v=-\frac{\partial F}{\partial v}=\frac{A_{2}^{\varphi +1}}{\varphi CA_j\left( p_pA_1-F \right) ^{\varphi -1}}
$$
由式(6- 24)、式(6- 25)可写成
$$
k_v=\frac{A_1}{\varphi v}\left( p_p-\frac{F}{A_1} \right) 
$$

比较式(6-26)和式(6-11),其形式完全相同.在供油压力$p_p$.执行元件的运动速度v及节流阀的结构形式与液压缸尺寸相同的情况下,出口节流调速回路的速度刚性和进口节流调速回路完全相同,其速度-负载特性曲线与特性分析也完全一样。比较式(6-24)和式(6-8),在其他条件相同的情况下,因为$A_{2<}A_1$,故进口节流调速能获得较低的工作速度。若为双出杆液压缸, $A_1=A_2$,则两者的速度范围完全相同。

3)功率特性与回路效率。泵的输出功率为
$$
P_p=p_pQ_p
$$

执行元件的有效功率
$$
P_1=Fv=\left( p_1A_1-p_2A_2 \right) v=p_pQ_1-p_2Q_2
$$

功率损失为
$$
\varDelta P=P_p-P_1=p_p\varDelta Q_y+p_2Q_2
$$
而$$
p_2Q_2=\varDelta p_j\left( \frac{A_2}{A_1}Q_1 \right) =\varDelta p'_jQ_1
$$
这里$\varDelta p'_j$为折算到进油路上的节流阀压力损失,故
$$
\varDelta P=p_p\varDelta Q_y+\varDelta p'_jQ_1
$$
说明出口节流调速回路的功率损失和进口节流调速回路相同.也是由溢流损失($p_p\varDelta Q_y$)和节流损失($\varDelta p'_jQ_1$)两部分组成,两者的功率特性和回路效率也相同。

(3)旁路节流调速回路(见图6-3)。

1)工作原理和回路参数。节流阀装在与执行元件并联的旁支油路上,定量泵输出的流量部分($Q_1$)直接进人执行元件,另一部分($Q_2$)通过节流阀流回油箱。不计泄漏时﹐由连续方程
$$
Q_p=Q_1+Q_2
$$
当不考虑管路的压力损失时,液压泵供油压力等于执行元件的工作压力,亦等于节流阀两端压力差,其大小决定于负载F和工作腔有效工作面积$A_1$,即
$$
p_p=p_1=\varDelta p_j=\frac{F}{A_1}
$$

溢流阀调定压力必须大于克服最大负载所需压力,故在工作时溢流阀处于关闭状态,仅回路过载时才打开,起安全保护作用。

调节节流阀通流面积,改变通过节流阀的流量$Q_2$﹐也就改变了进入执行元件的流量$Q_1$ ,从而调节执行元件的工作速度v,即
$$
v=\frac{Q_1}{A_1}=\frac{Q_p-Q_2}{A_1}=\frac{Q_p-CA_jp_{1}^{\varphi}}{A_1}=\frac{Q_p-CA_j\left( \frac{F}{A_1} \right) ^{\varphi}}{A_1}
$$
式中$Q_p$是指泵的出口流量﹐随压力的变化,泵的泄漏量也变化,即
$$
Q_p=Q_o-\varDelta Q_1=Q_o-C_1p
$$
其中 \quad
\setlength{\tabcolsep}{0.2mm}{
\begin{tabular}[t]{lll}
$\rm Q_o$&——&泵的理论流量;\\
$\rm \varDelta Q_1$&——&泵的泄露量,随压力的增大而增大;\\

\end{tabular}}

2)速度-负载特性。由式(6-32)可求得旁路节流调速回路的速度刚性为
$$
k_v=\frac{A_{1}^{2}}{\varphi CA_j}\left( \frac{F}{A_1} \right) ^{1-\varphi}
$$
按式(6-32)可得旁路节流调速回路的速度-负载特性曲线,如图6一7所示。

由式(6 -32)、式(6-33)及图6-7可知:a)随着负载的增加,运动速度下降很快,其速度——负载特性比进、出口节流调速回路更软;b)在节流阀通流截面积一定时,负载愈大,速度刚性愈大;c)负载-定时,节流阀通流面积愈小(即执行元件运动速度愈高)速度刚性愈好;d)增大执行元件有效工作面积,减小节流阀指数,可以提高速度刚性;e)执行元件工作速度愈低(即节流阀通流面积愈大),则其能承受的最大负载愈小,即低速时的最大承载能力变小,故节流阀的开度不能太大,这种回路只能在小流量范围内进行调节,调速范围较小。

3)功率特性和回路效率。液压泵输出功率随负载增大而增大,即
$$
P_p=p_pQ_p=p_1Q_p=\frac{F}{A_1}Q_p
$$

旁路节流无溢流损失,只有油液通过节流阀时所产生的节流损失和液压泵泄漏损失,即
$$
\varDelta P_j=p_1Q_2CA_jp_{1}^{\varphi +1}
$$

执行元件的有效功率为
$$
P_1=p_1Q_1=p_1\left( Q_p-CA_jp_{1}^{\varphi} \right) =P_p-CA_jp_{1}^{\varphi +1}
$$

当负载一定时,有效功率随工作速度增加而线性上升,功率损失则随之线性下降,如图6-8所示。

当负载变化时,泵功率随负载的增加而线性上升,有效功率则与进、出口节流调速回路相似,与负载呈曲线变化关系。旁路节流调速回路的效率为
$$
\eta =\frac{P_1}{P_p}=\frac{Q_1}{Q_p}=1-\frac{CA_jp_{1}^{\varphi}}{Q_p}
$$

由于泵的驱动功率随负载的增减而增减,故此种回路的效率比进、出口节流调速回路为高,工作速度愈大,效率愈高。

(4)三种节流调速方式的比较。三种节流阀节流调速回路主要性能的综合比较列于表6一1。

在进、出油路上同时安装节流阀的复合节流调速回路在生产实际中亦得到应用。由理论分析可知,复合节流调速回路的低速性能、速度刚性及调速范围均优于进口节流或出口节流调速回路,但由于回路上增加了一个节流元件,故功率损失与发热较大。

节流阀调速的共同优点是结构简单,能在较大范围内实现无级调速,速度随负载的变化而变化,机械特性软是普通节流阀调速的共同缺点,故多在负载变化不大的机床(如磨床工作台的传动系统)中应用.功率损耗大,尤其在低速.轻载时效率低,是这种调速方式的另--个共同缺点v故只限于用在功率不大的系统。

2.节流调速系统中的速度稳定

在节流阀调速回路中,负载的变化引起速度变化的原因在于负载变化引起节流阀两端的压力差变化,因而使通过节流阀进人执行元件的流量发生变化,执行元件的运动速度亦随之变化。要解决这--问题,必须使节流阀两端的压力差与负载的变化无关或关系很小。

(1)采用调速阀(速度稳定器)的节流调速回路。图6-9是调速阀装在进油路上的回路图。该图所示的工作原理与节流阀进口节流调速回路相同。调速阀的工作原理见第4-4节。调速阅中的减压阀a是-一种能自动调节开口量,进行压力补偿﹐保持节流阀b两端的压力差基本不变的定差式减压阀,减压阀阀芯两端的压力分别作用于节流阀b的进出口端,由式(4-18)可知,使节流阀两端的压力差亦基本保持不变,即
$$
\varDelta p_j=p_g-p_1=\frac{F_g}{A_g}\approx \text{常数}
$$
式中 \quad
\setlength{\tabcolsep}{0.2mm}{
\begin{tabular}[t]{lll}
$\rm P_g$&——&减压阀后,节流阀入口处压力;\\
$\rm \varDelta p_1$&——&节流阀出口处压力;\\
\rm 其余同式(4-18)。\\

\end{tabular}}

节流阀的开口量一定时,不管负载如何变化,由于$\varDelta p_j$基本不变,故其过流量亦基本保持
