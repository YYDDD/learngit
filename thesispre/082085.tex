
\chapter{控制阀}

\section{概述}

阀在液压系统中起控制调节作用,它可对液压系统所需的压力大小、油液的流动方向、流量的多少进行控制调节,以满足工作部件克服外部载荷、改变运动方向和运动速度的要求。阀的类别根据用途不同,大致可分为三大类:方向控制阀、压力控制阀和流量控制阀;如果依据操纵动力划分,则有手动、机动、电动、液动、气动及电-液动等类型;如果按照连接方式分,则有管式、板式、法兰连接式和集成块式等形式。

尽管阀的类别和品种繁多,但它们都具有以下共性:

\begin{enumerate}[(1)]
\item 从阀的结构来看,均由阀体、阀芯和控制动力三大部分组成。

\item 从阀的工作原理来看,都是利用阀芯和阀体的相对位移来改变通流面积,从而控制压力、流向和流量。

\item 各种阀都可以看成是油路中的一个液阻,只要有液体流过,都会产生压力降(有压力损失)和温度升高等现象。
\end{enumerate}

阀在液压系统中起着神经中枢作用,阀的质量优劣,直接影响液压系统工作的性能。为此,控制阀应具备如下要求:

\begin{enumerate}[(1)]
\item 动作灵敏、准确、可靠,工作平稳,冲击和振动要小。

\item 密封性好,油液流过时漏损少,压力损失小。

\item 结构紧凑,工艺性好,使用维护方便,通用性好。
\end{enumerate}

\section{方向控制阀}

方向控制阀在液压系统中起阻止和引导油液按规定的流向进出通道,即在油路中起控制油液流动方向的作用。

方向控制阀按工作职能可分为单向阀和换向阀两类。

\subsection{单向阀}

单向阀的作用是使油液只能向一个方向流动,而不能反向流动。常用的单向阀有普通单向阀与液控单向阀两种。

\subsubsection{普通单向阀}

图4-1所示为一种普通单向阀的结构和符号图。其工作原理是:压力为$p_\text{1}$的压力油从阀体的入口流入,推动阀芯压缩弹簧,油液则经阀芯的径向孔从阀体的出口流出,其压力降为$p_\text{2}$。如反向流人油液,则阀芯在液压力与弹簧力的共同作用下,堵死阀口,使油液无法流出。

单向阀的阀芯还有钢球式,如图4-2所示。由于它的对中性及密封性较差,多用在小流量及要求不高的场合。

在普通单向阀中,要求通油方向的液阻尽量小,一般选用的弹簧刚度较小,其开启压力为(0.35$\sim$0.5)x10$^5$Pa,全流量的压力损失为(1$\sim$3)x10$^5$Pa。如果单向阀作为背压阀使用,其弹簧刚度可取大一些,其开启压力为(2$\sim$6)x10$^5$Pa。

\subsubsection{液控单向阀}

图4-3所示为液控单向阀的结构和符号图。其工作原理是:当控制油口K不通压力油时,液控单向阀与普通单向阀的工作原理相同。当控制油口K通人控制油液时,活塞1推动顶杆2,进而顶开阀芯3,使$p_\text{1}$与$p_\text{2}$连通,油液可以从两个方向自由流动。控制油口的压力$p_\text{K}$一般取主油路压力的 30\%$\sim$40\%。

图4-4为液控单向阀的应用实例。当手动换向阀 3左移时,压力油经换向阀3,打开液控单向阀 4(此时单向阀4 的控制油口 K通油箱,其性能与普通单向阀相同)进入液压缸6的 A腔,与此同时,压力油进人液控单向阀5的控制油口,将阀5的阀芯顶开。液压缸 6上腔的油液经液控单向阀5、换向阀3与油箱连通。此时活塞在压力油的作用下运动。反之亦然。当换向阀处于中位时,液压缸6处于自锁状态。

图4-5所示是用液控单向阀的平衡回路。当换向阀左位接入回路时,压力油进人液压缸下腔,同时打开液控单向阀,工作部件向下运动。当换向阀处于中位时,液压缸下腔失压,液控单向阀关闭,工作部件立即停止运动。由于液控单向阀是锥面密封,泄漏量小,故锁闭性能好,可以防止工作部件因泄漏而缓慢下滑。

\subsection{换向阀}

换向阀在机床液压系统中用以改变液流的方向,实现运动换向及速度换接等。按结构可分为转阀式和滑阀式;按阀芯工作位置可分为二位、三位、多位;按阀的进出口通道数目可分为二通、三通、四通、五通等。

\subsubsection{转阀}

转阀是利用阀芯的转动,使阀芯与阀体相对位置发生变化来改变油流的方向。图4-6所示为转阀工作原理和符号图。

当转阀处在图4-6所示右位时,压力油从P口进人,经径向孔(实线所示)由A口流出,进入执行元件,而执行元件的回油由B口进入,经径向孔(虚线所示),由O口流出。当转阀阀芯转动到图4-6所示左位时,则P口和B口相通,A口和O口相通,使液流换向。

转阀由于结构尺寸较大,密封性能较差,易出现径向力不平衡,因而多用在流量较小、压力不高的场合,如用作先导阀及小型低压换向阀等。

\subsubsection{滑阀}

机床及其他各类液压系统中所使用的换向阀大部分是滑阀式结构。下面扼要介绍滑阀的结构、工作原理及其性能分析。

\begin{enumerate}[(1)]
\item 结构和工作原理。

\begin{enumerate}[1)]
\item 主体部分。滑阀阀芯与阀体是换向阀的主体,图4-7所示为阀体与阀芯结构示意图及相应的符号图。其工作原理是利用阀芯相对阀体的轴向位移以变换油液的流动方向。

图4-7(a)表示滑阀阀芯相对阀体处在左位,压力油由P口进人,经B口流出,回油从A口进入,经O口流回油箱。图4-7(b)表示滑阀阀芯相对阀体右移到右位时的油流走向。由于阀芯的移动,改变了油流的方向,因而也就改变了执行元件运动的方向。在结构示意图的下面画出了它们的符号图。

滑阀式换向阀,按阀芯工作位置数和进出阀的油口数目,可分为如图4-8所示的几种。

\item 操纵和定位部分。滑阀阀芯相对阀体的移动是靠操纵动力实现的。为了使滑阀可靠地工作,必须在实现操纵后将阀芯定位,使阀芯与阀体的相对位置处于给定状态。在机床液压传动与控制系统中常用的有以下几种类型:
\end{enumerate}

\begin{enumerate}[(i)]
\item 如图4-9所示,摆动手柄,即可改变阀芯与阀体的相对位置,从而使油路通断。阀芯定位靠钢珠、弹簧使其保持确定的位置。
\end{enumerate}
\end{enumerate}
