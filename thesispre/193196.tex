

\noindent 三种开口的滑阀的位移-流量特性曲线。
\begin{figure}
   \ifOpenSource
   \includegraphics{cover.jpg}
   \else
   \includegraphics{fig0907.pdf}
   \fi
   \caption{滑阀式液压放大器}
   \label{fig:fig0907}
\end{figure}
\begin{figure}
   \ifOpenSource
   \includegraphics{cover.jpg}
   \else
   \includegraphics{fig0908.pdf}
   \fi
   \caption{正开口、零开口和负开口滑阀}
   \label{fig:fig0908}
\end{figure}


从图9-8(a)可见,负开口滑阀在中间平衡位置时,四个节流口都被遮盖,完全断开了油源
与执行元件之间的通路。阀芯需左右移动一段距离$\emph{x}_0$以后,才能把相应的节
流口启开,才有油流输给执行元件,因此形成了没有流量输出的一段阀芯位移区,即死区,
造成不良的非线性的死区特性。

图9-8(c)所示为正开口滑阀。从图可见,阀芯在中间平衡位置时,由于各节流口都有一定的开口
量,因此造成一部分压力油经这些节流口直接流回油箱,成为无功损耗,并从以后的分析计算可知,
正开口滑阀增加了系统的静态误差。

图9-8(b)所示的零开口滑阀避免了正、负开口滑阀的缺点,所以应用最广。

为了保证良好的控制调节性能,制造时必须保证滑阀的四个节流口对称-致。为此形成节流口的四
个阀芯凸肩棱边的轴向位置尺寸与对应的阀套凹槽四个棱边轴向位置尺寸必须保持精密的配合,有
些甚至要求轴向配合精度在$2_{\mu m}$以内。此外,各棱边必须保持尖锐,四对棱边不得有轻微
擦伤或圆角,阀芯与阀套径向配合也要求十分精密,所有这一切都给制造带来困难,使制造成本增加。


\subsection{双边控制滑阀式液压放大器}
 
其工作原理在图9-1示例中已作了详细说明。滑阀所形成的两个节流口的控制调节作用可用图9-7(b)所
示的双臂电桥等效电路来描述。阀芯的偏移使-个节流口关小,液阻(电阻)加大,而另一个则开大,液
阻减小,从而使液压缸大腔压力力发生改变,使液压缸因此产生相应的伺服运动。

双边控制滑阀在中间平衡位置时,两节流口的初始开口量不同,同样有正开口、零开口和负开口之分。
和四边控制滑阀相似,以零开口的性能最好,应用最广。由于只有两个节流口起控制调节作用,液压缸
只一腔的压力随阀芯位移而变化,另一腔的压力恒定不变,不受控制。而四边控制滑阀则利用四个节流
口使油缸两腔压力同时变化,一腔增加则另一腔降低。因此,在相同的条件下,在液压缸上可获得的
推力和速度变化,四边控制的要比双边控制的大。由于双边控制滑阀只有两个节流口,要求精确配合
的轴向尺寸比四边控制的少,只有一个轴向尺寸要求精确配合,再加上棱边数目少,因此制造较容易。

\subsection{单边控制滑阀式液压放大器}

如图9-7(a)所示,具有恒压力为$\emph{p}_p$的压力油从油源直接进人液压缸有效面积小的一-腔,
并经阻尼孔1进人液压缸有效面积大的另一腔,且压力降为$\emph{p}_1$,再经滑阀上的节流口2流回油箱。当滑阀阀芯移动时,
改变节流口2的通流截面积的大小,从而改变液压缸大腔的压力$\emph{p}_1$,使液压缸两腔的油压作用力不平衡
而产生运动。在中间平衡位置时,节流口2有一个预开口量xo,它使得液压缸两腔油压作用力保持平
衡,即在油缸无负载时,$\emph{p}_1$A1=$\emph{p}_p$A2,因此油缸静止不动。当推动阀芯向左偏移$\emph{x}_v$距离时,节流口2开
大,使压力$\emph{p}_1$降低,使得$\emph{p}_1$A1\textless $\emph{p}_p$A2,则推动油缸向左移动。滑阀阀套是与液压缸固定在一起的,液压
缸左移,阀套也一起左移,当左移距离也为$\emph{x}_v$时,使节流口2恢复原来尺寸,压力$\emph{p}_1$也因此恢复到原值,
液压缸左右油压作用力又恢复平衡,液压缸运动停止。同样,阀芯向右偏移时,节流口2关小,$\emph{p}_1$上升,
液压缸将跟随右移并直到节流口2再回到原来状态为止。单边控制滑阀的工作原理可用图中的双臂电桥
的等效电路来描绘。阀芯偏移时,只改变电桥中的一个臂的阻值,另一臂是固定液阻(阻尼孔1)。因此,
在相同的条件下,它比双边控制滑阀所能获得的液压缸推力与运动速度的变化要小。而且在中间平衡
位置时,节流口2必须是开启的,因此不可避免地有无功流量损耗。但由于只有一个节流口,所以结构
最简单,制造最容易,成本亦最低。

综上所述,三种滑阀式液压放大器中,四边控制的控制调节性能最好,但结构最复杂,制造成本高,因此
主要用在控制要求严格的精密伺服系统中。而双边和单边控制的滑阀式液压放大器则用在一般系统中或
作为多级液压放大器的前置级。
\subsection{喷嘴挡板式液压放大器}
图9-9(a)所示为单喷嘴挡板式液压放大器的原理图。恒压力为$\emph{p}_p$的压力油经固定节流口a流人喷嘴前腔b,
且压力降为$\emph{p}_c$。压力油再由喷嘴前腔一路流人执行元件(液压缸)的工作腔,另一路经喷嘴c与喷嘴及挡板d
间的节流缝隙$\delta$流回油箱。显然,当挡板在输入信号作用下左右摆动,改变节流缝隙$\delta$时,将使前腔压
力$\emph{p}_c$变化,从而使执行元件运动。其工作原理类似单边控制滑阀式液压放大器,其等效电路如图9-9(b)所示。

 为了改善挡板受力情况和提高灵敏度更常采用的是如图9-10所示双喷嘴挡板式液压放大器。它实际上是将
 两个单喷嘴挡板式液压放大器连成推挽形式。当挡板处于两喷嘴之间的中间位置时,不难看出,两喷嘴的
 前腔及执行元件(液压缸)的两腔压力$\emph{p}_{c1}$和$\emph{p}_{c2}$相等,所以输出的压力差
 $\emph{p}_1$=$\emph{p}_{c1}$-$\emph{p}_{c2}$=0,
 液压缸不动。当挡板偏离中间位置时,例如向左偏移$\emph{x}_d$,使缝隙$\delta_1$减小,液流流经它
 时的液阻加大(参看图中的等效电路),则$\emph{p}_{c1}$增高。缝隙$\delta_2$加大,液阻减小,
 则$\emph{p}_{c2}$下降,因而有压力差$\emph{p}_1$=$\emph{p}_{c1}$-$\emph{p}_{c2}$产生,即可
 推动执行元件运动。喷嘴挡板式液压放大器更多的是作为多级液压放大器的前置级,在第9-4节将要讲到的
 电液伺服阀中就是用它来作为前置级放大器。
 \begin{figure}
   \ifOpenSource
   \includegraphics{cover.jpg}
   \else
   \includegraphics{fig0909.pdf}
   \fi
   \caption{单喷嘴挡板式液压放大器}
   \label{fig:fig0909}
\end{figure}
\begin{figure}
   \ifOpenSource
   \includegraphics{cover.jpg}
   \else
   \includegraphics{fig0910.pdf}
   \fi
   \caption{双喷嘴挡板式液压放大器}
   \label{fig:fig0910}
\end{figure}
 \subsection{射流管式液压放大器}
 图9-11所示为射流管式液压放大器的原理图。压力油经收缩型的射流管1将液体压力能变成动能,从射流口2
 高速射出,并为接收器5上呈扩散形的两接收孔3和4所接收,再将液体动能重新变成压力能来驱动执行元件。
 输人信号为射流管绕轴心$\emph{O}$的摆动,它使射流口2相对两接收口的重叠面积$\Delta$$\emph{f}_1$和
 $\Delta$$\emph{f}_2$改变,因而改变了两
 接收孔中接收到的液体动能分配比例。当射流口2在中间时,$\Delta$$\emph{f}_1$=$\Delta$$\emph{f}_2$,
 两接收口所接收到的液体动能相同,
 因此$\emph{p}_1$=$\emph{p}_2$,液压缸不动。
 
 当射流管摆动使射流口向右偏移$\emph{x}_f$距离
 (见图9-11(b))时,$\Delta$$\emph{f}_1$\textgreater $\Delta$$\emph{f}_2$,接收孔3接
 收到的液体动能就比接收孔4的大。因此$\emph{p}_1$\textgreater $\emph{p}_2$,液压缸的活塞将左移。
 当射流口向左偏移时,同理,活塞将右移。显然,活塞移动的速度以及产生的推力大小与输人信号(射流管)偏
 移量$\emph{x}_f$成比例。

 射流管式液压放大器是-种非节流式液压放大器,其工作原理与滑阀式、喷嘴挡板式液压放大器有根本区别。
 前者是改变液体动能分配比例来控制执行元件运动的,而后者是利用油液通过不同开口量的节流口造成不
 同的压力降来控制执行元件运动的。
 \begin{figure}
   \ifOpenSource
   \includegraphics{cover.jpg}
   \else
   \includegraphics{fig0911.pdf}
   \fi
   \caption{射流管式液压放大器}
   \label{fig:fig0911}
\end{figure}

 射流管式液压放大器的优点是射流口较大,因而对脏物不敏感,不容易出现堵塞或像滑阀式那样出现“卡死”
 故工作可靠性高结构简单,制造容易。然而至今对射流口与接收孔之间的液流状态的分析研究不够,
 还没有精确的分析和计算方法,对其性能也难预测,因此应用还不多,但有的国家在液压放大前置级
 中逐渐用它来代替喷嘴挡板式放大器的趋势。目前我国已有此类的系列产品。
 \section{机液伺服系统特性分析与计算}
 现在以图9-12所示四边控制滑阀式液压仿形刀架为例,对机液伺服系统的特性分析与计算做简要的介绍。
 \subsection{四边控制滑阀式液压放大器特性}
 系统是依靠四边控制滑阀式液压放大器实现伺服控制的,因此它的特性在系统中就起关键的作用。现在
 假定系统所使用的是理想零开口的四边控制滑阀,即认为阀芯与阀套相应的校边绝对锋锐,轴向尺寸完全
 一致 、对称,阀芯与阀套的径向问隐也为零。因此,这个理想的滑阀当其处于中间平街位置时,四个控
 制节流口完全关闭无油流。当阀芯相对阀套向右位移$\emph{x}_v$时,节流口1和4启开,而2和3关闭,
 从图中右上角的等效电路可见。1和4两臂是通的,其液阻大小取决于$\emph{x}_v$的大小,而2和3两臂
 则断开。如果滑阀所控制
 的液压缸两腔的有效面积一样,在不考虑油液的泄漏和可压缩性的情况下,流过节流口1和4以及液压缸的
 流量$\emph{Q}_1$,$\emph{Q}_4$和$\emph{Q}_1$是相等的,即
 \begin{equation}
    \emph{Q}_1=\emph{C}_d\emph{w}\emph{x}_v\sqrt{\frac{2}{\rho}(\emph{p}_p-\emph{p}_1)}
 \end{equation}
