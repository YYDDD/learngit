
\noindent 回油:

(2)减速加压。主缸活塞接触工件后,阻力增加,上腔油压升高,关闭液控单向阀14。这时只有泵3继续向主缸上腔供高压油,推动活塞慢速下行,对工件加压。主缸下腔排油将液控单向阀12封闭,经背压阀13回油箱。这样,当快速行程转为工作行程时,速度减低,从而避免了液压冲击。

系统中的远程调压阀9可使液压机在不同的压力下工作,安全阀8用于防止系统超载。

进油:

回油:

(3)保压延时。当主缸上腔的油压达到要求的数值时,由压力继电器17发信号,使电液换向阀10回复中位,将主缸上、下腔油路封闭。这时泵3也卸荷,而单向阀15被高压油自动关闭,主缸上腔进入保压状态。但这种实现保予的方法要求主缸活塞,单向阀(保医阀)及其间的管种具有很高的密封性能,若泄漏较大,压力会迅速下降,无法实现保压。在保压过程中变量泵3的压力油经换向阀10和4回油箱,使泵卸荷。

进油:

回油:

(4)泄压回程。保压时主缸上腔油液的压缩和管道膨胀储存了能量,而使其上腔的油压很高,再加上主缸为差动油缸﹐所以当电液换向阀10很快切换到回程位置,会使回程开始的短时间内泵3及主缸下腔的油压升得很高,比保压时主油路的压力还要高得多,以致引起冲击和振动。所以保压后必须先逐渐泄压然后再回程,以防冲击和振动发生。该液压系统保压完毕,压力继电器17控制时间继电器TS发信号(定程成形时,由挡铁压行程开关$XK_3$发信号),使各阀处于回程位置,回程开始。主缸上腔高压油打开泄压阀15,并且液控单向阀14也被打开,使泵3来的油经泄压阀15中的阻尼孔(形成一定阻力)回油箱,泵3成为低负荷运转。这时主缸活塞并不马上回程,待上腔压力降低,泄压阀被关闭后,泵3的油才能进入主缸下腔开始回程。

主油箱:

控制油路:

(5)回程停止。当主缸挡铁压行程开关$XK_1$时,使各阀处于停止位置,主缸活塞回程停止。变量泵3经电液换向阀10和4卸荷。

进袖:

回油:

2.顶出缸的运动

顶出缸的动作是在主缸停止时才能进行的,因为进入顶出缸的压力油,经过主缸油路的电液换向阀10后,才通入顶出缸油路的电液换向阀4的。电液换向阀10处在中间位置即主缸停止运动时,才能实现顶出和顶退运动,保证免除误动作。

(1)顶出。按下按钮$A_3$,使电液换向阀4在左位工作,从泵3来的压力油进入顶出缸下腔,顶出缸的活塞上升将工件顶出。

进油:

回油:

(2)顶退,按下按钮$A_4$,使电液换向阀4在右位工作,从泵3来的压力油进入顶出缸上腔,顶出缸的活塞向下退回。

进油:

回油:

(3)停止。按下按钮$A_5$,使各阀处于停止位置,顶出缸活塞停止运动。

进油:

回油:

(4)压边。作薄板拉伸时的压边动作,顶出缸停止在顶出位置。这时顶出缸下腔油液被电液换向阀4封闭,所以当主缸活塞下压时,顶出缸活塞被迫随之下行(此时阀4中位,泵3卸荷),顶出缸下腔的油液只能经固定节流器7和溢流阀6缓慢流回油箱,从而建立起所需的压边力。固定节流器7和溢流阀6用来调节压边压力;安全阀5是当固定节流器7阻塞时起安全作用。

3.静止时下滑问题

液压机主缸活塞及其所带的滑块往往很重,为防止活塞回程停止后,因泄漏或其他原因(如泵电机突然掉电)而自动下滑,回路中装有液控单向阀12和背压阀13来封闭主缸下腔的油液,起支撑平衡作用,保证主缸活塞可靠地停留在任何位置。但为防止因阀12失灵(不通)使主缸下腔产生超高压事故,背压阀13起安全作用。其背压所产生的抗力,足以支持活塞及其所带动的滑块的自重,即光靠自重无法顶开背压阀13,所以活塞不会自动下落。

4.液压机工作缸的换向及其低压控制

主液压缸和顶出液压缸的换向都由电液换向阀担当。为使两缸动作协调,两个电液换向阀4和10这样配置,即主缸油路的回油要经过顶出缸油路的电液换向阀4才能回油箱﹐从而保证了顶出缸停止动作时,主缸才能运动。而且顶出缸的进油要经过控制主缸油路的阀10,这就保证了主缸处于停止时,顶出缸才能运动。
当液压机系统压力高时,为避免换向冲击,电液换向阀由外控供油,必须有低压控制油路,不宜直接引用主油路的高压油。该系统采用单独的小流量辅助液压泵作为能源的低压控制油路,控制压力为$(10-15)×10^3Pa$,压力稳定,工作可靠。

三、液压系统的分析

(1)液压机工作循环中,压力、行程速度和流量变化较大,泵的输出功率也较大。如何满足液压机工作循环要求,又能使能量消耗最小,是液压机液压系统设计中要考虑的问题。

液压机液系统通常有两种供油方案:一种是采用高低压泵组,用一个高压小流量柱塞泵和一个低压大流量齿轮泵组合起来向系统供油;另一种是采用恒功率变量柱塞泵向系统供油,以满足低压快速行程和高压慢速行程的要求。

(2)在不增加主油泵功率的前提下提高快速行程速度以提高生产率,其基本方法是增加低压供油的流量或减小活塞面积,可采用自重充液﹑蓄能器强制充液、快速缸、辅助缸或差动回路来提高低压快速行程的速度,当快速行程转为慢速工作行程时,为了避免冲击,可通过减速回路减速。

(3)立式液压机为使滑块可靠地停留在任何位置,须采用平衡回路,可根据具体要求选用各种基本回路来组合。

(4)由于液压机主油路压力较高,为避免换向冲击,电被换向阀一般由低压,外控油路来控制,不宜直接引用主油路的高压油。

(5)液压机工作循环中的保压过程与制品质量密切相关,很多液压机均要求保压性能。保压后必须逐渐泄压,泄压过快,将引起液压系统猛烈的冲击、振动和噪声。因此保压和泄压是液压机系统必须考虑的两个同题。

(6)液压机的液压系统属高压,大流量和太功率系统。因此,合理利用功率以降低温升非

