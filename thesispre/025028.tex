(1- 33)可求得液流流入或流出阀腔时的稳态维动力为
\begin{equation}
F_\text{s}=\rho Qv\cos \theta
\end{equation}
\noindent 式中\ 
\begin{tabular}[t]{ll}
$\theta$——射流角,一般取$\theta$=$69^{\circ}$;\\
$v$——阀口处的平均流速。
\end{tabular}

稳态液动力的方向总是指向关闭阀口的方向,相当于一个回复力,使滑阀的工作趋于稳定。

\subsubsection*{瞬态波动力}

    瞬态液动力是滑阀在移动过程中(即开口大小发生变化时)阀腔中液流因加速或减速而作用在阀芯上的力。这个力只与阀芯移动速度有关(即与阀口开度的变化率有关),与阀口开度本身无关。

    图1 - 20表示了阀芯移动时出现瞬态液动力的情况。当阀口开度变化时,阀腔内长度为\l 那部分油液的轴向速度亦发生变化,也就出现了加速或减速,于是阀芯上就受到了一个轴向的反作用力$F_{a}$,这就是瞬态液动力。由式(1 - 37a)可知
\begin{equation*}
F_\text{a}=\rho l \frac{dQ}{dt}
\end{equation*}
当阀口前后的压差不变或变化不大时,流量的变化率$\frac{dQ}{dt}$与阀口开度的变化率$\frac{dx_{v}}{dt}$成正比。

    滑阀上瞬态液动力的方向,视油液流入还是流出阀口而定。图1 - 20(a)中油液流出阀口,当阀口开度加大时长度为$l$的那部分油液加速,开度减小时油液减速,这两种情况下瞬态液动力作用方向都与阀芯移动方向相反,起着阻止阀芯移动的作用,相当于一个阻尼力,并将$l$称之为“正阻尼长度”。反之,图1 - 20(b)的情况油液流入阀口,阀口开度变化时引起液流流速变化的结果,都是使瞬态液动力的作用方向与阀芯移动方向相同,起着帮助阀芯移动的作用,相当于一个负的阻尼力。这种情况下$l$称为“负阻尼长度"。


\section{流动液体的流量-压力特性}

前一节叙述了液体运动最普遍适用的基本规律,并未涉及具体装置中(如管路、孔口等)液体运动的物理本质,因而有些问题,例如伯努利方程中的能量损失($h_{w}$)等并未解决。每一具体的流动都有其相应的流量-压力特性,下面分别加以叙述。

\subsection*{压力损失}

在密封管道中流动的液体存在两种损失:一种是液体在圆管中流动因黏性产生的沿程损失;另一种是由于管道截面突然变化、液流速度大小和方向突然改变等而引起的局部损失。两种能量损失均可用压力损失来表示。压力损失大小与流动状态有关,下面将分别进行讨论。


\subsubsection*{沿程损失}

当液流为层流状态时,其流量及沿程压力损失均可由理论公式计算。

图1 - 21所示为液体在等径(半径为$R$)水平圆管中作恒定层流时的情况。在图中的管内取出一段半径为$r$,长度为$l$,与管轴相重合的微小圆柱体,作用在其两端面上的压力为$p_{1}$和$p_{2}$,作用在侧面上的内摩擦力为$F_{f}$。根据力的平衡,有
\begin{equation*}
(p_{1}-p_{2})\pi r^{2}=F_{f}
\end{equation*}
内摩擦力按式(1 - 6)计算为
\begin{equation*}
F_{f}=-2\pi \mu rl \frac{du}{dr}
\end{equation*}

图1 - 21所示坐标轴中速度梯度为负值,故式中加一负号以使摩擦力为正值。令$\Delta p$=$p_{1}$-$p_{2}$,将这些关系代入上式,则有
\begin{equation}
\frac{du}{dr}=-\frac{\Delta p}{2\mu l}r
\end{equation}

\noindent 对式(1 - 38)进行积分得
\begin{equation*}
u=-\frac{\Delta p}{4\mu l}r^{2}+C
\end{equation*}

\noindent 积分常数$C$由边界条件确定,即$r$=$R$时,$u$=0,则有
\begin{equation*}
C=\frac{\Delta p}{4\mu l}R^{2}
\end{equation*}
从而求得速度分布表达式为
\begin{equation}
u=\frac{\Delta p}{4\mu l}(R^{2}-r^{2})
\end{equation}

\noindent 式(1 - 39)是一抛物面方程。最大速度发生在轴线上,即$r$=0处,速度最大,有
\begin{equation}
u_{max}=\frac{\Delta p}{4\mu l}R^{2}=\frac{\Delta p}{16\mu l}d^{2}
\end{equation}

由式(1 - 39)看出,液体在圆管中作层流流动时,速度按对称于管轴的抛物线规律分布。由于速度分布不均匀,为了计算流量,在半径r处取一层厚为$dr$的微小圆环面积(见图1 - 21),通过此环形面积的流量为
\begin{equation*}
dQ=2\pi urdr
\end{equation*}
对此式积分
\begin{equation}
Q=\int_{0} ^{R}2\pi urdr=\frac{\pi R^{4}}{8\mu l}\Delta p=\frac{\pi d^{4}}{128\mu l}\Delta p
\end{equation}
或
\begin{equation*}
\frac{\Delta p}{l}=\frac{8\mu Q}{\pi R^{4}}
\end{equation*}

式(1-41)表明,液体在圆管中作层流流动时,流量与管径的四次方成比例,压力差(压力损失)则与管径的四次方成反比,可见管径对流量及压力损失的影响是很大的。这个公式又叫泊肃叶公式。

管中平均流速$v$可表示为
\begin{equation}
v=\frac{Q}{A}=\frac{4Q}{\pi d^{2}}=\frac{\frac{\pi R^{4}}{8\mu l}\Delta p}{\pi R^{2}}=\frac{1}{2}\frac{\Delta p}{4\mu l}R^{2}=\frac{1}{2}u_{max}
\end{equation}
由式(1 - 42)可知,通流截面上的平均流速为管子中心线上最大流速之半。

由速度分布规律,可计算出通流截面上的实际动能和实际动量,则可进一步求出动能修正系数$\alpha$(式(1-28))及动量修正系数$\beta$(式(1-34))。
\begin{equation*}
\alpha =\frac{\int_{A}u^{3}dA}{v^{3}A}=\frac{\int_{0} ^{R}[\frac{\Delta p(R^{2}-r^{2})}{4\mu l}]^{3}2\pi rdr}{[\frac{\Delta pR^{2}}{8\mu l}]^{3}\pi R^{2}}=2
\end{equation*}
\begin{equation*}
\beta =\frac{\int_{A}u^{2}dA}{v^{2}A}=\frac{\int_{0} ^{R}[\frac{\Delta p(R^{2}-r^{2})}{4\mu l}]^{2}2\pi rdr}{[\frac{\Delta pR^{2}}{8\mu l}]^{2}\pi R^{2}}=\frac{4}{3}\approx 1.33
\end{equation*}

伯努利方程中$h_{w}$一项,若仅考虑沿程损失,管径不变并水平安放,则可按式(1 - 29)求出
\begin{equation}
h_{\lambda}=\frac{p_{1}-p_{2}}{\rho g}=\frac{\Delta p}{\rho g}
\end{equation}

若管中是层流流动,由式(1-41)可得到
\begin{equation}
\Delta p=\frac{128\mu l}{\pi d^{4}}Q=\frac{32\mu l}{\pi d^{2}}v
\end{equation}
将上式代入式(1- 43)中,并经适当变换可得到
\begin{equation}
h_{\lambda}=\frac{\Delta p}{\rho g}=\frac{64}{\rho \frac{ud}{\mu}}=\frac{l}{d} \frac{v^{2}}{2g}=\frac{64}{ld}\frac{1}{d}\frac{v^{2}}{2g}=\lambda \frac{1}{d}\frac{v^{2}}{2g}
\end{equation}

\noindent 式中$\lambda = \frac{64}{Re}$为沿程阻力损失系数。在机床液压传动系统中,$\lambda$和$Re$间的关系曲线如图1 - 22所示。 

由式(1 - 45)可见,流体在管道中流动的能量损失表现为流体的压力损失,即流体下游的压力要小于上游的压力,这个压力差值用来克服流动中的摩擦阻力。

在实际情况下,由于管壁附近的流体层因冷却作用而引起局部黏性系数增多,从而使摩擦阻力加大,因此在液压技术中流体为油时取
\begin{equation*}
\lambda= \frac{75}{Re}
\end{equation*}
如果管道是橡胶软管,由于管中流动状况易受扰动,常取
\begin{equation*}
\lambda= \frac{80}{Re}
\end{equation*}